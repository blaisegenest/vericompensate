Deep neural networks have demonstrated remarkable capabilities, achieving \newline human-like or even superior performance across a wide range of tasks. However, their robustness is often compromised by their susceptibility to input perturbations. This vulnerability has catalyzed the verification community to develop various methodologies, each presenting a unique balance between completeness and computational efficiency \cite{katz2019marabou,Reluplex,deeppoly}. This surge in innovation has also led to the inception of competitions such as VNNComp \cite{VNNcomp}, which aim to systematically evaluate the performance of neural network verification tools. While the verification engines are generic, the benchmarks usually focus on local robustness, i.e. given a DNN, an image and a small neighbourhood around this image, 
is it the case that all the images in the neighbourhood are classified in the same way.
While some quite large DNNs (e.g. ResNet with tens of thousands of neurons) can be verified very efficiently (tens of seconds per input) \cite{crown}, with all inputs either certified robust or an attack on robustness is found; some smaller DNNs (with hundreds of neurons, only using the simpler ReLU activation function) cannot be analysed fully, with $12-20\%$ of {\em undecided} inputs where neither of the decisions can be reached (\cite{crown} and Table \ref{tab:example}). Actually, DNNs which are trained to be robust (using DiffAI \cite{DiffAI} or PGD \cite{PGD}) are easier to verify, while the DNNs trained in a {\em natural} way are harder to verify.

In this work, we concentrate on DNNs that are trained naturally, given the prevalence of techniques for {\em easier} DNNs. Our analysis is confined to DNNs employing the standard ReLU activation function, though our findings potentially extend to more sophisticated architectures. Our investigation delves into the core abstraction mechanisms integral to several prominent algorithms, such as Eran-DeepPoly \cite{deeppoly}, the Linear Programming approximation \cite{MILP}, PRIMA \cite{prima}, and various implementations of ($\alpha$)($\beta$)-CROWN \cite{crown,xu2020fast}\footnote{
	\cite{VNNcomp} reports that $\alpha,\beta$-Crown \cite{crown,xu2020fast} often surpasses other competing techniques;  It maintains completeness for smaller DNNs \cite{xu2020fast}, and showcases impressive efficiency for larger networks, benefiting from branch-and-bound based methodology \cite{cutting,BaB}.
}. Perhaps, it is worth highlighting that these tools are  sound but not necessarily complete, i.e., when these tools certify a DNN to be robust for a particular image $I$, then the corresponding DNN Is indeed robust but it may happen that the tool is unable to certify a DNN to be robust even though the DNN is indeed robust. To dig further into their incompleteness,we first remark that high-level approach followed by all these techniques is to compute lower or/and upper bounds for the values of neurons (abstraction on values) for inputs in the considered input region, and then finally conclude based on the bounds of neurons in the output layer. Therefore, inaccurate bounds leads to incompletness. 


{\color{red}
As a starting point, we sought to understand properties of DNNs that make them hard to verify.
Our key finding is a new notion of {\em compensations}, that leads to inaccuracies of bounds and thus explain the {\em hardness} of verifying some DNNs.} Formally, a compensating pair is a tuple of paths $(\pi,\pi')$ between a pair of neurons $(a,b)$, such that we have $w < 0 < w'$, for $w,w'$ the products of weight seen along $\pi$ and $\pi'$. Ignoring the (ReLU) activation functions, the weight of $b$ is loaded with $w \cdot weight(a)$ by $\pi$, while it is loaded with $w' \cdot weight(a)$ by $\pi'$. That is, it is loaded by $(w+w') weight(a)$. As $w,w'$ have opposite sign, they will compensate (partly) each other. The compensation is only partial due to the ReLU activation seen along the way of $\pi$ which can "clip" a part of $w \cdot weight(a)$, and similarly for $\pi'$. However, it is very hard to evaluate by how much without explicitly considering both phases of the ReLUs, which all the efficient tools try to avoid because it is very expansive (could be exponential in the number of such ReLU nodes opened).
%; for instance, what is a differentiating  feature between DNNs trained in natural way (i.e., without explicit concern for robustness) versus the DNNs trained to be robust. 



Our three main contributions address the challenges of inaccuracies in neural network computations due to compensating pairs of paths:
\begin{enumerate}
	\item  Our first main contribution establishes, through Theorem \ref{th1}, the foundational understanding that the compensating pairs are the 	%primary 
	source of {\color{red}all} inaccuracies, 
	{\color{red} as in their absence, even the simplest Box abstraction (and most abstractions under consideration) is fully accurate.}
	This revelation, while theoretically pivotal, has limited practical applicability as most networks inherently possess compensating pairs. Nonetheless, this insight lays the groundwork for our innovative approach, {\color{red}{\em relaxing} the usual but inefficient Mixed Integer Linear Program (MILP) formulation encoding exactly DNNs \cite{MILP}. To compute bounds for $z$, we encode as binary variables (the costly ones) only ReLU nodes on compensating paths {\em with target $z$}; other nodes, in particular those on compensating paths to neurons before $z$ are encoded using the much more efficient linear relaxation. To take into account indirectly the compensating paths to previous nodes, we inductively compute bounds for every neuron layer by layer, and use these bounds of previous neurons in the MILP encoding for $z$. This is more efficient, as the number of binary nodes is strongly reduced. However, the question remains how accurate is it?}
	
\item {\color{red} Our second main contribution, delineated in Theorem \ref{th2}, answers this question, showing that this abstraction is fully accurate, in that the bounds computed in such a way for neuron $z$ correspond exactly to the mininmal and maximal values reachable by inputs of the DNN. Again, the prevalence of compensating paths in actual DNNs renders this algorithm computationally intensive because the number of binary variables is still too high.}

\item Our third main contribution addresses this practical concern, introducing 
Algorithm \ref{algo1}, a pragmatic solution that leverages our theoretical insights. This approach ranks ReLU nodes based on {\color{red} 
the weights of the compensating paths they belong to,
and differentiates their treatment in the MILP model (binary for heavy, linear for light). 
%Notably, Algorithm \ref{algo1} is exceptionally amenable to parallelization, as all nodes within the same layer can be processed simultaneously. Furthermore, 
Our comprehensive empirical evaluation indicates that Algorithm \ref{algo1} provides favorable trade-offs between accuracy and completeness compared to existing methods, lowering the number of 
undecided inputs from $17.6\%$ ($\beta$-Crown \cite{crown}) to $10.4\%$ 
on DNN $8 \times 200$ (Table \ref{tab:example}), even verifying more than $20\%$ more images than $\alpha,\beta$-CROWN \cite{xu2020fast} (or PRIMA \cite{prima}) on the larger $6 \times 500$ DNN (Table \ref{tab:example3}).}

%We verify experimentally that the algorithm offers interesting trade-offs, by testing on local robustness for DNNs trained "naturally" (and thus difficult to verify).

%KSM: I think this is a distraction in intro, so I suggest moving to later part
% Overall, the worst case complexity of algorithm \ref{algo1} is lower than $O(N 2^K LP(N))$, where $N$ is the number of nodes of the DNN, $K$ the number of ReLU nodes selected as binary variable, and $LP(N)$ is the (polynomial time) complexity of solving a linear program representing a DNN with $N$ nodes. This complexity is an upper bound, as e.g. Gurobi is fairly efficient and never need to consider all of the $2^K$ ReLU configurations to compute the bounds. Keeping $K$ reasonably low thus provides an efficient algorithm. 
%By design, it will never run into a complexity wall (unlike the full MILP encoding), although it can take a while on large networks because of the linear factor $N$ in the number of nodes.

 


\end{enumerate}






%   
% 
%
%In this context, application of DNNs in safety critical applications is cautiously envisioned. For that to happen at a large scale, hard guarantees should be provided \cite{certification}, through e.g. incremental verification \cite{incremental}, so that to avoid dramatic consequences. It is the reason for the development of (hard) verification tools since 2016, with now many tools with different trade-offs from exact computation but slow (e.g. Marabou \cite{katz2019marabou}/Reluplex\cite{Reluplex}), up to very efficient but also incomplete (e.g. ERAN-DeepPoly \cite{deeppoly}). To benchmark these tools, a competition has been run since 2019, namely VNNcomp \cite{VNNcomp}. The current overall better performing verifier is $\alpha$-$\beta$-Crown \cite{crown}, a fairly sophisticatedly engineered tool based mainly on "branch and bound" (BaB) \cite{BaB}, and which can scale all the way from complete on smaller DNNs \cite{xu2020fast} up to very efficient on larger DNNs, constantly upgraded, e.g. \cite{cutting}. 
%
%While the verification engines are generic, the benchmarks usually focus on local robustness, i.e. given a DNN, an image and a small neighbourhood around this image, 
%is it the case that all the images in the neighbourhood are classified in the same way.
%While some quite large DNNs (e.g. ResNet with tens of thousands of neurons) can be verified very efficiently (tens of seconds per input) \cite{crown}, with all inputs either certified robust or an attack on robustness is found; some smaller DNNs (with hundreds of neurons, only using the simpler ReLU activation function) cannot be analysed fully, with $12-20\%$ of inputs where neither of the decisions can be reached (\cite{crown} and Table \ref{tab:example}). Actually, DNNs which are trained to be robust (using DiffAI \cite{DiffAI} or PGD \cite{PGD}) are easier to verify, while the DNNs trained in a "natural" way are harder to verify.
%
%
%In this paper, we focus on DNNs trained in a "natural" way,
%%uncovering what makes the DNNs trained in a natural way so hard to verify (
%because for "easier" DNNs, adequate methods already exist. 
%To do so, we analyse the abstraction mechanisms at the heart of several efficient algorithms, namely Eran-DeepPoly \cite{deeppoly}, the Linear Programming approximation \cite{MILP}, PRIMA \cite{prima}, and different versions of ($\alpha$)($\beta$)-CROWN \cite{crown}. All these algorithms compute lower or/and upper bounds for the values of neurons (abstraction on values) for inputs in the considered input region, and conclude based on such bounds. For instance, if for all image $I'$ in the neighbourhood of image $I$, we have $weight_{I'}(n'-n) < 0$ for $n$ the output neuron corresponding to the expected class, then we know that the DNN is robust in the neighbourhood of image $I$. We restrict the formal study to DNNs using only the standard ReLU activation function, although nothing specific prevents the results to be extended to more general architectures. We uncover that {\em compensations} 
%(see next paragraph) is the phenomenon creating inaccuracies. We verified experimentally that a DNN trained in a natural way has heavier compensating pairs than DNNs trained in a robust way.
%
%Formally, a compensating pair is a pair of paths $(\pi,\pi')$ between a pair of neurons $(a,b)$, such that we have $w < 0 < w'$, for $w,w'$ the products of weight seen along $\pi$ and $\pi'$. Ignoring the (ReLU) activation functions, the weight of $b$ is loaded with $w \cdot weight(a)$ by $\pi$, while it is loaded with $w' \cdot weight(a)$ by $\pi'$. That is, it is loaded by $(w+w') weight(a)$. As $w,w'$ have opposite sign, they will compensate (partly) each other. The compensation is only partial due to the ReLU activation seen along the way of $\pi$ which can "clip" a part of $w \cdot weight(a)$, and similarly for $\pi'$. However, it is very hard to evaluate by how much without explicitly considering both phases of the ReLUs, which all the efficient tools try to avoid because it is very expansive (could be exponential in the number of such ReLU nodes opened).

%Our first main contribution is to formally show, in Theorem \ref{th1}, that compensation is the sole reason for the inaccuracies as (most) efficient algorithms will compute exact bounds for all neurons if there is no compensating pair of paths at all.
%While this theorem is theoretically interesting, it is not usable in practice as (almost) all networks have some compensating pairs. However, this notion of compensating pairs opens a first interesting idea concerning an exact abstraction of the network using a Mixed Integer Linear Program \cite{MILP}, where the weight of each neuron is a linear variable, and ReLU node may be associated with binary variables (exact encoding) or linear variables (overapproximation). While LP tools can scale to thousands of linear variables, MILP encoding can only be solved for a limited number of binary variables. This suggests that a simpler encoding could be used for those ReLUs that are not on compensating pairs, as their precise outcome may not be necessary.

%Our second main contribution is to show formally in Theorem \ref{th2}, that 
%encoding all ReLU nodes on a pair of compensating paths with a binary variable,
%and using linear relaxation for the other ReLU nodes, will lead to exact bounds for (most) of the algorithms considered. This theorem allows to restrict the number of integer variables, and thus to obtain encodings that are faster to solve. Practically, however, (almost) all ReLU nodes are on some compensating path, and using this exact restricted MILP encoding will be too time consuming.

%Our third main contribution is more practical, proposing Algorithm \ref{algo1} based on this knowledge that compensating pair of paths are the reason for inaccuracy. The idea is thus to use this information to rank the ReLU nodes in terms of importance, and only keep the most important ones as binary variables, and use linear relaxation for the least important ones.
%%More precisely, the algorithm will, as DeepPoly, consider layers one by one and neurons $b$ %on this layer one by one, selecting the heaviest pairs of compensating paths ending in $b$
%%and associating these nodes with a binary variable. Then an MILP tool such as Gurobi is used %to compute the lower and upper bound for node $b$. 
%Overall, the worst case complexity of algorithm \ref{algo1} is lower than $O(N 2^K LP(N))$, where $N$ is the number of nodes of the DNN, $K$ the number of ReLU nodes selected as binary variable, and $LP(N)$ is the (polynomial time) complexity of solving a linear program representing a DNN with $N$ nodes. This complexity is an upper bound, as e.g. Gurobi is fairly efficient and never need to consider all of the $2^K$ ReLU configurations to compute the bounds. Keeping $K$ reasonably low thus provides an efficient algorithm. 
%By design, it will never run into a complexity wall (unlike the full MILP encoding), although it can take a while on large networks because of the linear factor $N$ in the number of nodes. An additional interesting point is that it is extremely easy to parallelize, as all the nodes in the same layer can be run in parallel. We verify experimentally that the algorithm offers interesting trade-offs, by testing on local robustness for DNNs trained "naturally" (and thus difficult to verify).


%KSM: I suggest we move this to experimental evaluation
%This paper does not focus on producing the most efficient tool, and we did not spend engineering efforts to optimize it. The focus is instead on the novel notion of compensation, the associated methodology and its evaluation. For instance, our implementation is fully in Python, with uncompetitive runtime for our DeepPoly implementation ($\approx 100$ slower than in CROWN). Still, evaluation of the methodology versus even the most efficient tools reveals a lot of potential for the notion of compensation, opening up several opportunities for applying it in different contexts of DNN verification (see Section \ref{Discussion}). 


\section{Comparison with Related Works} 

We compare with several (but not all) main verification tools for DNNs, to better explain our methodology and how it differs with the existing SOTA. Compared with the exact encoding of a DNN using MILP \cite{MILP}, 
we select a restricted number of ReLU nodes based on compensating strength to scale to larger DNNs while maintaining good accuracy. Further, compared to MIPplanet \cite{MIPplanet} which 
uses a different selection of ReLU nodes, we compute bounds for all neurons iteratively, reducing drastically the number of ReLU nodes represented exactly for a given neuron (ReLU nodes which were considered for neurons in previous layers are not represented exactly for that neuron, while MIPplanet would need to represent all those ReLU nodes exactly), using many fast computations, more efficient than one large computation.

Compared with the linear relaxation of the MILP encoding \cite{MILP}, our algorithm is strictly more accurate by design, but it will also be slower.

Compared with ERAN-Deeppoly \cite{deeppoly}, which compute bounds on the value in a very efficient way, we prove that the LP encoding is strictly more accurate, which as far as we know is another novel result. To be more precise, DeepPoly
abstract the weight of every node using two functions, one upper function and one lower function. While the upper function is fixed, there are 2 choices for the lower bound.
We prove in Proposition  \ref{LP} that the LP relaxation corresponds exactly to the intersection of both choices. It is thus more accurate than DeepPoly, but also not as efficient. Therefore, our algorithm will also be (much) more accurate than DeepPoly, but also not as efficient.

Concerning PRIMA \cite{prima}, the idea is to keep explicitly dependencies between neurons, computing bounds layer by layer (as we do). This allows to keep very efficiently dependencies from potentially many layers beforehand. We take care of dependencies between neurons in a different way, as compensation is the reason why there are dependencies between neurons. 
Our method is more accurate locally, but we will tend to lose precision for dependencies created many layers ago. Experimental results tend to show that most of the dependencies are local, in the few last layers (because ReLU nodes will likely clip those that happened many layers ago). Also, the dependencies between nodes limit the parallelism, unlike in our method, which explains why we obtain both faster and more accurate results than PRIMA.

For comparison, $\alpha$-$\beta$ CROWN \cite{crown} (and other Branch and Bound algorithms, such as BaBSR \cite{BaB}) will run few instances of branch and bound (one per output neuron), in worst case considering all the possible ReLU configurations (although the branch and bound algorithm avoids most of the possibilities). On simple networks, such as those trained robust, this is particularly efficient because branch and bound can find very efficiently the bounds focusing on the actual question, considering the important branches, while our algorithm will be less efficient as it has to consider each node one by one from the start. However, branch and bound faces a complexity wall when the network is hard to verify, such as the DNNs trained naturally, as there are too many branches to consider.

On such complex DNNs, PRIMA and $\alpha$-$\beta$ CROWN resort to a "refined" path, where the bounds {\em on the first few layers} are refined \cite{MILP2} using an exact MILP encoding. In our algorithm, we do not use an exact encoding but a partial one with the most important ReLU nodes obtained by considering the compensation strength. As it is more efficient, this can be pushed to all layers. This would be infeasible without the selection based on the compensation. The refined version of $\alpha$-$\beta$ CROWN is particularly accurate on small DNNs. As the depth grows, the more work is left to BaB and our algorithm is more accurate, lowering the gap of verified images to the upper bound \cite{attack} down from 
$20\%$ to $16.2\%$ and from $17.6\%$ to $10.9\%$ (depth 8, Table \ref{tab:example}). On larger naturally learnt DNNs (3000 neurons), BaB only verifies $18.5\%$ more images than DeepPoly (PRIMA only $6.5\%$), while our algorithm verifies $39\%$ more images than DeepPoly (Table \ref{tab:example3}).

Finally, algorithms abstracting the network (e.g. Reluplex / Marabou \cite{Reluplex,katz2019marabou}) are very different from algorithms abstracting the values (PRIMA, ($\alpha$)($\beta$)-CROWN)\cite{prima,crown}$\ldots$ These algorithms have been developed to be complete, so they are much slower but also more accurate than what we propose.
