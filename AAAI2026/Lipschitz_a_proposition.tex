\documentclass{llncs}
\usepackage{hyperref}
\usepackage{url}
\pagestyle{plain}
\usepackage{threeparttable}
\input{math_commands.tex}
%\usepackage[latin9]{inputenc}
%\usepackage[T1]{fontenc}
\usepackage{float}
\usepackage{wrapfig}
\usepackage{lineno}
\usepackage{amsmath}
\usepackage{amssymb}
\usepackage{graphicx}
\usepackage{subcaption}
\usepackage{tabularx}
\usepackage{cases}
\captionsetup{compatibility=false}
% \usepackage{esint}
\usepackage{array}
\usepackage{epstopdf}
\usepackage{placeins}
\usepackage{pgfplots}
\usepackage{url}
\usepackage{tikz}
\usepackage{calc}
\usepackage{array}
\usepackage[linesnumbered,ruled,vlined]{algorithm2e}
\usetikzlibrary{positioning, arrows.meta,calc}

%%%%%%%%%%%%%%%%%%%%%%%%%%%%%%%
%%%%%%%%%%%%%%%%%%%%%%%%%%%%%%%


\newcommand{\vW}{\boldsymbol{W}}


\newcommand{\val}{{\textrm{value}}}
\newcommand{\Val}{{\textrm{value}}}
\newcommand{\MILP}{{\textrm{MILP}}}
\newcommand{\LP}{{\textrm{LP}}}
\newcommand{\Improve}{\mathrm{Improve}}
\newcommand{\Utility}{\mathrm{SAS}}
\newcommand{\Sol}{\mathrm{Sol}}
\newcommand{\sol}{\mathrm{sol}}

\newcommand{\UB}{\mathrm{UB}}
\newcommand{\LB}{\mathrm{LB}}
\newcommand{\ub}{\mathrm{ub}}
\newcommand{\lb}{\mathrm{lb}}
\newcommand{\B}{\mathrm{B}}
\usepackage{amsmath, amssymb, amsfonts}


\newcommand{\ReLU}{\mathrm{ReLU}}

\newcommand{\CMP}{{\textrm{CMP}}\ }

\newcommand{\fix}{\marginpar{FIX}}
\newcommand{\new}{\marginpar{NEW}}
\newcommand{\toolname}{Hybrid MILP}


\title{Solution-aware vs global ReLU selection: \\
partial MILP strikes back for DNN verification}
\date{}


\begin{document}
	
	\begin{definition}[$\varepsilon$-diff bound]
	Suppose we have a function $f$ from $\mathbb{R}^n$ to $\mathbb{R}^m$ and $||$ is the $L_\infty$ norm. 
			
	For a number $\varepsilon\in\mathbb{R}$, an $\varepsilon$-diff bound $D_\varepsilon$ is a number such that for any inputs $x,y$: \begin{align*}
	|x-y|\leq \varepsilon \implies |f(x)-f(y)| \leq D_\varepsilon \cdot \varepsilon
\end{align*}
			
	\end{definition}
	
	
\begin{definition}[Lipschitz above $\varepsilon$ constant ]
	Suppose we have a function $f$ from $\mathbb{R}^n$ to $\mathbb{R}^m$ and $||$ is the $L_\infty$ norm. 
	
	For a number $\varepsilon\in\mathbb{R}$, a Lipschitz above $\varepsilon$ constant  $K_\varepsilon$,  is a number such that for any inputs $x,y$: \begin{align*}
	|x-y|\geq \varepsilon &\implies |f(x)-f(y)| \leq K_\varepsilon \cdot |x-y|\\
	|x-y|<\varepsilon &\implies |f(x)-f(y)| \leq K_\varepsilon \cdot \varepsilon\\
	\end{align*}
	
\end{definition}


\begin{proposition}
	
	Suppose $D$ is an $\varepsilon$-diff bound for $f(x)$. Then for any $N\in\mathbb{Z}^+$, $D\frac{N+1}{N}$ is a Lipschitz about $N\varepsilon$ constant.
	
	In formula, for any two inputs $x,y$, if \begin{align*}
		|x-y|\leq \varepsilon \implies |f(x)-f(y)| \leq D \cdot \varepsilon,
	\end{align*} then 	 \begin{align*}
	|x-y|\geq N\varepsilon &\implies |f(x)-f(y)| \leq D\frac{N+1}{N} \cdot |x-y|\\
	|x-y|<N\varepsilon &\implies |f(x)-f(y)| \leq D\frac{N+1}{N} \cdot N\varepsilon\\
	\end{align*}
\end{proposition}

\begin{proof} We fix the number $N\in\mathbb{Z}^+$ and assume we have two inputs $x, y$.
	
	The first case, $|x-y|\geq N\varepsilon$. Then we assume $|x-y| \in [M\varepsilon ,  (M+1)\varepsilon]$ for another integer $M\geq N$. Then we can divide the line segment between $x, y$ into $M+1$ pieces: $x_0 = x, x_1, x_2, \cdots, x_{M+1} = y$ such that $|x_i-x_{i+1}| \leq \varepsilon$ and apply the definition of $\varepsilon$-diff bound $D$ for each pieces:\begin{align*}
		|f(x)-f(y)| &= |f(x_{M+1})-f(x_M)+\cdots+f(x_1)-f(x_0)|\\
		&\leq |f(x_{M+1})-f(x_M)|+\cdots+|f(x_1)-f(x_0)|\\
		&\leq D\varepsilon + \cdots +D\varepsilon \\
		&= (M+1)D\varepsilon
	\end{align*}
	Hence\begin{align*}
		|f(x)-f(y)| &\leq (M+1)D\varepsilon \\
		&\leq D\cdot (M+1)\varepsilon \frac{|x-y|}{M\varepsilon}\\
		&= D\cdot\frac{M+1}{M} |x-y|\\
		&\leq   D\cdot\frac{N+1}{N} |x-y|
	\end{align*}
	The second case, $|x-y|< N\varepsilon$. Similarly we can divide the line segment between $x, y$ into $N$ pieces and then $|f(x)-f(y)|\leq D N\varepsilon\leq |f(x)-f(y)|\leq D \frac{N+1}{N} N\varepsilon$.
	
	This is what we want to show.
\end{proof}


\begin{proposition}
For any $N\in\mathbb{Z}^+$, suppose for any $a$ in $\{\frac{N}{N}\varepsilon,\frac{N+1}{N}\varepsilon,\cdots, \frac{2N-1}{N}\varepsilon\}$, $D$ is an $a$-diff bound for $f(x)$. 

Then $D\frac{N+1}{N}$ is a Lipschitz about $\varepsilon$ constant:\begin{align*}
	|x-y|\geq \varepsilon &\implies |f(x)-f(y)| \leq D\frac{N+1}{N} \cdot |x-y|\\
	|x-y|<\varepsilon &\implies |f(x)-f(y)| \leq D\frac{N+1}{N} \cdot \varepsilon\\
\end{align*}
\end{proposition}

\begin{proof}
We fix the number $N\in\mathbb{Z}^+$ and assume we have two inputs $x, y$.
	
For the case that $|x-y|<\varepsilon$, this is trivial by definition.

For the case that $|x-y|\geq \varepsilon$, there exists a sum $x_1+x_2+\cdots+x_n$ by numbers from (allowing repetitions) $\{\frac{N}{N}\varepsilon,\frac{N+1}{N}\varepsilon,\cdots, \frac{2N-1}{N}\varepsilon\}$ such that \begin{align*}
	 \varepsilon \leq x_1+x_2+\cdots+x_n -\frac{1}{N}\varepsilon \leq |x-y| \leq x_1+x_2+\cdots+x_n
	\end{align*}
	By assumption, divide the line segment from $x$ to $y$ into pieces according to $x_1, x_2,\cdots,x_n$, then we will have $$|f(x)-f(y)|\leq Dx_1+Dx_2+\cdots+Dx_n.$$
	
	Hence,\begin{align*}
		\dfrac{|f(x)-f(y)|}{|x-y|} &\leq \dfrac{Dx_1+Dx_2+\cdots+Dx_n}{x_1+x_2+\cdots+x_n -\frac{1}{N}\varepsilon}\\
		& = D\cdot( 1+  \dfrac{\frac{1}{N}\varepsilon}{x_1+x_2+\cdots+x_n -\frac{1}{N}\varepsilon})\\
		& \leq D\cdot( 1+  \dfrac{\frac{1}{N}\varepsilon}{\varepsilon})\\
		& = D \frac{N+1}{N} 
	\end{align*}
	
\end{proof}


\end{document}


