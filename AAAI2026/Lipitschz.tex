\documentclass[letterpaper]{article} % DO NOT CHANGE THIS
\usepackage[submission]{aaai2026}  % DO NOT CHANGE THIS
\usepackage{times}  % DO NOT CHANGE THIS
\usepackage{helvet}  % DO NOT CHANGE THIS
\usepackage{courier}  % DO NOT CHANGE THIS
\usepackage[hyphens]{url}  % DO NOT CHANGE THIS
\usepackage{graphicx} % DO NOT CHANGE THIS
\urlstyle{rm} % DO NOT CHANGE THIS
\def\UrlFont{\rm}  % DO NOT CHANGE THIS
\usepackage{natbib}  % DO NOT CHANGE THIS AND DO NOT ADD ANY OPTIONS TO IT
\usepackage{caption} % DO NOT CHANGE THIS AND DO NOT ADD ANY OPTIONS TO IT
\frenchspacing  % DO NOT CHANGE THIS
\setlength{\pdfpagewidth}{8.5in} % DO NOT CHANGE THIS
\setlength{\pdfpageheight}{11in} % DO NOT CHANGE THIS
%
% These are recommended to typeset algorithms but not required. See the subsubsection on algorithms. Remove them if you don't have algorithms in your paper.
\usepackage{algorithm}
\usepackage{algorithmic}
\pagestyle{plain}
\usepackage{threeparttable}
\input{math_commands.tex}
\usepackage{lineno}
\usepackage{subcaption}
\usepackage{tabularx}
\usepackage{cases}
\captionsetup{compatibility=false}
\usepackage{epstopdf}
\usepackage{placeins}
\usepackage{pgfplots}
\usepackage{tikz}
\usepackage{calc}
\usepackage{array}
%\usepackage[linesnumbered,ruled,vlined]{algorithm2e}
\usetikzlibrary{positioning, arrows.meta,calc}
\usepackage{newfloat}
\usepackage{listings}
\DeclareCaptionStyle{ruled}{labelfont=normalfont,labelsep=colon,strut=off} % DO NOT CHANGE THIS
\lstset{%
	basicstyle={\footnotesize\ttfamily},% footnotesize acceptable for monospace
	numbers=left,numberstyle=\footnotesize,xleftmargin=2em,% show line numbers, remove this entire line if you don't want the numbers.
	aboveskip=0pt,belowskip=0pt,%
	showstringspaces=false,tabsize=2,breaklines=true}
\floatstyle{ruled}
\newfloat{listing}{tb}{lst}{}
\floatname{listing}{Listing}
%
% Keep the \pdfinfo as shown here. There's no need
% for you to add the /Title and /Author tags.
\pdfinfo{
	/TemplateVersion (2026.1)
}

\title{Order reduction and partial MILP models \\
	for certifying Robustness of DNNs globally.}
\date{}
\author{
	%Authors
	% All authors must be in the same font size and format.
	Written by AAAI Press Staff\textsuperscript{\rm 1}\thanks{With help from the AAAI Publications Committee.}\\
	AAAI Style Contributions by Pater Patel Schneider,
	Sunil Issar,\\
	J. Scott Penberthy,
	George Ferguson,
	Hans Guesgen,
	Francisco Cruz\equalcontrib,
	Marc Pujol-Gonzalez\equalcontrib
}
\affiliations{
	%Afiliations
	\textsuperscript{\rm 1}Association for the Advancement of Artificial Intelligence\\
	% If you have multiple authors and multiple affiliations
	% use superscripts in text and roman font to identify them.
	% For example,
	
	% Sunil Issar\textsuperscript{\rm 2},
	% J. Scott Penberthy\textsuperscript{\rm 3},
	% George Ferguson\textsuperscript{\rm 4},
	% Hans Guesgen\textsuperscript{\rm 5}
	% Note that the comma should be placed after the superscript
	
	1101 Pennsylvania Ave, NW Suite 300\\
	Washington, DC 20004 USA\\
	% email address must be in roman text type, not monospace or sans serif
	proceedings-questions@aaai.org
	%
	% See more examples next
}

%Example, Single Author, ->> remove \iffalse,\fi and place them surrounding AAAI title to use it
\iffalse
\title{My Publication Title --- Single Author}
\author {
	Author Name
}
\affiliations{
	Affiliation\\
	Affiliation Line 2\\
	name@example.com
}
\fi

\iffalse
%Example, Multiple Authors, ->> remove \iffalse,\fi and place them surrounding AAAI title to use it
\title{My Publication Title --- Multiple Authors}
\author {
	% Authors
	First Author Name\textsuperscript{\rm 1},
	Second Author Name\textsuperscript{\rm 2},
	Third Author Name\textsuperscript{\rm 1}
}
\affiliations {
	% Affiliations
	\textsuperscript{\rm 1}Affiliation 1\\
	\textsuperscript{\rm 2}Affiliation 2\\
	firstAuthor@affiliation1.com, secondAuthor@affilation2.com, thirdAuthor@affiliation1.com
}
\fi


\newtheorem{proposition}{Proposition}
\newtheorem{definition}{Definition}
\newcommand{\vW}{\boldsymbol{W}}
\newcommand{\val}{{\textrm{value}}}
\newcommand{\Val}{{\textrm{value}}}
\newcommand{\MILP}{{\textrm{MILP}}}
\newcommand{\LP}{{\textrm{LP}}}
\newcommand{\Improve}{\mathrm{Improve}}
\newcommand{\Utility}{\mathrm{SAS}}
\newcommand{\Sol}{\mathrm{Sol}}
\newcommand{\sol}{\mathrm{sol}}
\newcommand{\UB}{\mathrm{UB}}
\newcommand{\LB}{\mathrm{LB}}
\newcommand{\ub}{\mathrm{ub}}
\newcommand{\lb}{\mathrm{lb}}
\newcommand{\B}{\mathrm{B}}
\usepackage{amsmath, amssymb, amsfonts}
\newcommand{\ReLU}{\mathrm{ReLU}}
\newcommand{\CMP}{{\textrm{CMP}}\ }
\newcommand{\fix}{\marginpar{FIX}}
\newcommand{\new}{\marginpar{NEW}}
\newcommand{\toolname}{Hybrid MILP}

% REMOVE THIS: bibentry
% This is only needed to show inline citations in the guidelines document. You should not need it and can safely delete it.
\usepackage{bibentry}
% END REMOVE bibentry

\begin{document}
	
	\maketitle
	
	\begin{abstract}
		Deep neural networks (DNNs) are often brittle to small perturbations, which has led to extensive research to verify their robustness. 
Most existing methods focus on {\em local} robustness, i.e., verification in the neighborhood of fixed specific inputs. Local robustness 
%does {\em not} provide guarantees on whether new specific incoming inputs are robust, e.g., real-time images in a video stream. Moreover, most verification 
techniques are impractical to guarantee robustness of {\em real-time} inputs
on embedded systems, due to excessive latency and computational intensivity.

In this paper, we consider {\em global} robustness, which is significantly more complex than local robustness, as the number of variables doubles (from the deviation image to the image and its deviation). 
%Further, the values each neuron can take are no longer limited to a small neighborhood. 
%, that is, guarantees not restricted to a set of local images. 
We focus on deriving {\em bounds} on the variation of output values across different decision classes of a DNN, given $L_\infty$- {\em and} $L_1$-perturbations. We develop novel {\em Diff} MILP models encoding the evolution of {\em diff}erential variables between the image and its perturbation, more efficient than the classical MILP encoding of variables independently. To obtain better bounds, we reduce the space dimension by using principal component analysis (PCA), focusing on {\em realistic} inputs. These bounds enable the {\em real-time} certification of robustness for $\geq 67\%$ 
of incoming images for an $L_1$-perturbation of $1,4,2$
over the MNIST, Fashion MNIST and CIFAR-10 datasets respectively,
adding only $0.5ms$ of latency using only a single CPU core.
		\iffalse
		Most DNNs are brittle to small perturbations. Extensive works have thus been performed to verify robustness for DNNs.
		However, these works mostly consider local robustness, i.e. in the neighborhood of an image.
		While local robustness is useful to have an idea how often non robust images happen, by repeating the verification on 1000 or 10000 pre-obtained images, the main shortcoming is that we have no guarantee that a specific new incoming image, e.g. in a video feed, is robust: The verification process takes too long and requires too much resources to be performed online on embedded systems.
		
		In this paper, we consider {\em global} robustness, that is, guarantees not restricted to a set of local images. For that, we consider {\em bounds} on the switch of values between the different decision classes of a DNN due to a given perturbation. 
		The verification question is much harder than local robustness, as the number of complex variables doubles (from the deviation image to the image and its deviation).
		Further, the values each neuron can take is no more in a small neighborhood.
		Therefore, the global verification process is very complex.
		To obtain useable bounds, we develop several novel partial MILP models for global robustness, with different trade-offs. Last, we use order reduction techniques to reduce the space of images considered, avoiding unrealistic inputs, by using linear PCA. 
		This results into usable bounds, allowing in real time to certify robustness for $87\%$ of incoming images in the MNIST benchmark for a L1-perturbation of $0.5$, as well as for a surrogate computing the hidden plastic strain associated to a deformation map of a pipe.
		\fi
	\end{abstract}
	
	
	\section{Introduction}
	
	Deep neural networks (DNNs for short) have demonstrated remarkable capabilities, achieving human-like or even superior performance across a wide range of tasks. However, their robustness is often compromised by their susceptibility to input perturbations \cite{szegedy}. This vulnerability has catalyzed the verification community to develop various methodologies, each presenting a unique balance between completeness and computational efficiency \cite{Marabou,Reluplex,deeppoly}. This surge in innovation has also led to the inception of competitions such as VNNComp \cite{VNNcomp}, which aim to systematically evaluate the performance of neural network verification tools. Among them, NNenum \cite{nnenum}, Marabou \cite{Marabou,Marabou2}, and PyRAT \cite{pyrat} MnBAB \cite{ferrari2022complete}, built upon ERAN \cite{deeppoly} and PRIMA \cite{prima}; and $\alpha,\beta$-CROWN \cite{crown,xu2020fast}, 
%the winner of the last 4 VNNcomp, benefiting from 
based on branch-and-bound based methodology \cite{cutting,BaB}.

These tools %benchmarks usually 
focus on {\em local} robustness, i.e. given a DNN, an image and a small neighborhood around this image, is it the case that all the images in the neighborhood are classified in the same way by the DNN? The neighborhood is provided by a maximal perturbation of the input image, often an 
$L_\infty$-perturbation, i.e. every subpixel of the input image can vary in a very small range, typically $\frac{2}{255}$ (that is 2 levels of grey/blue/red/green). 
Although it is not necessarily the most meaningful perturbation,
$L_\infty$ is the usual choice because it is perfectly linear and specifies 
subpixel perturbations independantly, which is easier to verify. 
Importantly, these verification tools for local robustness are too computationally-intensive to be used in a real-time decision making pipeline: considering an autonomous car with a video feed from the dashboard, 
images of the feed cannot be certified robust in few ms on embedded hardware to e.g. skip non-robust images and only consider certified robust images.
% for the decision-making process.

\smallskip

In this paper, we consider {\em global} robustness, that is we do not restrict the certification process to the neighborhood of a fix input. We follow a two steps procedure. 
The first step, performed offline once, computes global bounds on the shift between 
output values of different decision classes due to the perturbation, close to \cite{vhagar}. That is, considering decision classes $C$ and $D$ and perturbation $\varepsilon$, compute an 
upper bound $\bar{\beta}^\varepsilon_{C,D}$ of $\max_{I,I', |I-I'| \leq \epsilon}(value_{I}(C) - value_{I}(D) + value_{I'}(D) - value_{I'}(C))$, where $value_{J}(X)$ is the output value of class $X \in \{C,D\}$ for input image $J \in \{I,I'\}$. %Notice that as $I,I',C,D$ have symetrical roles, we can choose $\beta^\varepsilon_{D,C} = \beta^\varepsilon_{C,D}$. Also, as $I,I'$ have symetrcal role, the minimum value is exactly $-$ the maximum value. 
%Hence, if we have $n$ decision classes, we only have to compute $n (n-1)/2$ bounds. 
Bound $\bar{\beta}^\varepsilon_{C,D}$ is computed offline once for a DNN, 
and it is valid over the whole input space; compared with $k$ calls to 
{\em local} robustness, once for each of the $k$ input images, with results only valid for these $k$ images.

The second step is real-time, being performed with the DNN inference of the image $I$ considered: it suffices to consider the class $C$ with the highest output value $value_{I}(C)$, and check whether for every other class $D \neq C$, 
$value_{I}(C) - value_{I}(D) > \bar{\beta}^\varepsilon_{C,D}$. 
If this is the case, then we are certified that image $I$ is robust for perturbation $\varepsilon$, because $\varepsilon$-perturbed image $I'$ could at most get  $value_{I'}(D) \leq \bar{\beta}^\varepsilon_{C,D}  - (value_{I}(C) - value_{I}(D))  + value_{I'}(C) < value_{I'}(C)$, hence $C$ is also the predicted class for image $I'$. This typically needs a couple of tens CPU instructions, which can be performed under 1ms (mili-second) on a single CPU core. If the image is not certified robust, one could either skip image $I$ (in a video feed), or use safer degraded mode in the decision making process till a trustable robust image is received.


Our main contributions address the challenges to compute the {\em global bounds} $\bar{\beta}^\varepsilon_{C,D}$, for $C,D$ output neurons:
% for standard ReLU DNNs (e.g. \cite{vhagar}). Our findings can be extended to other activation functions, following similar extention by \cite{DivideAndSlide}:
%, with updated MILP models e.g. for maxpool:
\begin{enumerate}
	%\item  Our first contribution studies the {\em LP relaxation} of the exact MILP encoding of ReLUs. {\color{blue} We establish in Proposition \ref{LP} its equivalence with the so-called "triangular abstraction"}.
	
	\item We develop a novel {\em Diff MILP encoding} for the global robustness problem, %called the {\em "2v" model}, 
	where the variables are the values of the perturbed neurons, as well as the difference between the original and the perturbed neuron values (called the {\em diff variables}, introduced in \cite{diff}). We study and encode how the {\em diff variables} evolve after passing through a ReLU (Prop.~\ref{Prop2}), see Section \ref{s.diff}. 
	Compared with the {\em classical MILP model} \cite{MILP} employed in 
	\cite{vhagar,lipshitz,ITNE}, which considers the input and the perturbation but {\em diff variable}, we keep the same number of 2 binary variables per ReLU.
	%the results are more accurate for the same runtime.
	%the linear relaxation is much more accurate, as each {\em diff variable} can be bounded after a ReLU as a function of the value of the {\em diff variables} before the ReLU, whereas the linear relaxations of the classical model is extremely inaccurate, as the variables are independent of each other. This was observed in \cite{lipshitz,ITNE}, and constraints encoding
	%a part of the linear relaxation of our "2v" model were added explicitly. 
	%Experimentally, 
	However, our {\em Diff MILP model} is more efficient, one reason being 
	the accuracy of its linear relaxation.
	%The number of variables a priori doubles compared with local robustness, from each neuron value in the perturbed image to each neuron value in the perturbed image {\em and} in the original image, as the original image is no more fixed. 
	%Recall that the worst case complexity of MILP is exponential in the number of binary variables \cite{DivideAndSlide}. 
	%A straightforward MILP model would be to use the  for each of these variables, as in \cite{vhagar,lipshitz}. The main issue with the classical model is that its linear relaxations is extremely inaccurate, as the variables are independent of each other. Instead, we develop another exact MILP model, 

	\item Further, from the {\em Diff MILP model}, which is exact, 
	we develop two abstract MILP models, which are more efficient but also asymptotically less accurate than {\em Diff}. 
	Namely, the "2b+1" model, accurate on the {\em diff variables} but abstract on the perturbed variables; while the "1b" model has a unique binary variable per ReLU, only considering the {\em diff variables}.
	%\item We adapt the Solution Aware Scoring from \cite{ATVA25} to our novel MILP models, in order to select the most important ReLUs to be treated using complex binary variables, while less important variables are treated using linear relaxation. The chosen number of binary variables depends upon the complexity of the DNN as well as the targeted runtime.

   \item  In terms of perturbations, we consider conjunctions of $L_\infty$- and $L_1$-norms, which allow to accurately describe perturbations. For instance, "each subpixel is perturbed by at most $\frac{50}{255}$ ($L_\infty$) and the sum of the absolute value of perturbations over all subpixels is at most $1$" ($L_1$-perturbation). While $L_1$-perturbations are not linear (because of the absolute values), reason for which it is seldom used, we show in Section~\ref{s.L1} how to use it as a perturbation in the MILP model without incurring any expansive binary variables (only cheap linear variables are necessary). 



\item Bounds obtained on the full input space are particularly pessimistic, as all inputs, including {\em Out of Distribution (OOD \cite{OOD})} inputs far away from the training dataset, need to be accounted for. 
%As a result, the runtime to obtain the bound is particularly long. 
	%Finally, when computing worst-case 
	%pairs (image, perturbation), improbable images are generated, hence these  are not meaningful. 
To address this, we consider model order reduction techniques from engineering science \cite{Paco}. Specifically, we use Principal Component Analysis (PCA) to represent faithfully common inputs from the dataset. OOD	inputs may be represented unfaithfully, which is reasonable as the DNN is unlikely to provide reasonable answer on such inputs anyway.
	%	focus on reduce the space to a linear input space. We choose the number of dimensions of the space to equal the accuracy of the DNN on the reduced space. 
	%(using a projection to the reduced space then the inverse projection to obtain a very similar image understandable by the DNN). 
For instance, using just $20$ out of 784 dimensions suffices to represent faithfully MNIST inputs, without losing accuracy for the DNN on the MNIST dataset.
	%On the MNIST benchmark, this means 20 linear dimensions to match the $97\%$ accuracy of the DNN we considered, instead of the 784 dimensions of the full image space. 
	%Using PCA, which is linear, makes it easy to specify perturbations on the actual image (where it is meaningful) rather than on the reduced space.

\item Experimentally, the {\em Diff MILP model} computes upper bounds
$\bar{\beta}^{\varepsilon}_{C,D}$ which reduces the gap to the lower bound compared with optimized version of the classical MILP encoding; namely reducing the number of binary variables (Vhagar \cite{vhagar}); or adding linear constraints from the {\em diff variables} (ITNE \cite{ITNE}), see Table \ref{table.classical}.
The abstractions  "2b+1" and "1b" variant offer different trade-offs, 
reaching better bounds than the {\em Diff MILP} model when the instance is very complex, or when runtime is limited (Table \ref{table.L1}).  
Further, using PCA reduces the upper bound $\beta_{C,D}$ by $3$ to $20$ times.
Overall, using these different techniques ({\em Diff} MILP model, abstraction and PCA) enables the real-time certification of $> 67\%$ of fresh images for an $L1$-perturbation of $1,4,2$ over the MNIST, Fashion MNIST and CIFAR-10 datasets respectively, see Table \ref{table.cert}. The online process adds only 0.5ms of latency per image, and 2000 images/second can be treated per CPU core.
\end{enumerate}


%\newpage

%   
% 
%
%In this context, application of DNNs in safety critical applications is cautiously envisioned. For that to happen at a large scale, hard guarantees should be provided \cite{certification}, through e.g. incremental verification \cite{incremental}, so that to avoid dramatic consequences. It is the reason for the development of (hard) verification tools since 2016, with now many tools with different trade-offs from exact computation but slow (e.g. Marabou \cite{katz2019marabou}/Reluplex\cite{Reluplex}), up to very efficient but also incomplete (e.g. ERAN-DeepPoly \cite{deeppoly}). To benchmark these tools, a competition has been run since 2019, namely VNNcomp \cite{VNNcomp}. The current overall better performing verifier is $\alpha$-$\beta$-CROWN \cite{crown}, a fairly sophisticatedly engineered tool based mainly on "branch and bound" (BaB) \cite{BaB}, and which can scale all the way from complete on smaller DNNs \cite{xu2020fast} up to very efficient on larger DNNs, constantly upgraded, e.g. \cite{cutting}. 
%
%While the verification engines are generic, the benchmarks usually focus on local robustness, i.e. given a DNN, an image and a small neighbourhood around this image, 
%is it the case that all the images in the neighbourhood are classified in the same way.
%While some quite large DNNs (e.g. ResNet with tens of thousands of neurons) can be verified very efficiently (tens of seconds per input) \cite{crown}, with all inputs either certified robust or an attack on robustness is found; some smaller DNNs (with hundreds of neurons, only using the simpler ReLU activation function) cannot be analysed fully, with $12-20\%$ of inputs where neither of the decisions can be reached (\cite{crown} and Table \ref{tab:example}). Actually, DNNs which are trained to be robust (using DiffAI \cite{DiffAI} or PGD \cite{PGD}) are easier to verify, while the DNNs trained in a "natural" way are harder to verify.
%
%
%In this paper, we focus on DNNs trained in a "natural" way,
%%uncovering what makes the DNNs trained in a natural way so hard to verify (
%because for "easier" DNNs, adequate methods already exist. 
%To do so, we analyse the abstraction mechanisms at the heart of several efficient algorithms, namely Eran-DeepPoly \cite{deeppoly}, the Linear Programming approximation \cite{MILP}, PRIMA \cite{prima}, and different versions of ($\alpha$)($\beta$)-CROWN \cite{crown}. All these algorithms compute lower or/and upper bounds for the values of neurons (abstraction on values) for inputs in the considered input region, and conclude based on such bounds. For instance, if for all image $I'$ in the neighbourhood of image $I$, we have $weight_{I'}(n'-n) < 0$ for $n$ the output neuron corresponding to the expected class, then we know that the DNN is robust in the neighbourhood of image $I$. We restrict the formal study to DNNs using only the standard ReLU activation function, although nothing specific prevents the results to be extended to more general architectures. We uncover that {\em compensations} 
%(see next paragraph) is the phenomenon creating inaccuracies. We verified experimentally that a DNN trained in a natural way has heavier compensating pairs than DNNs trained in a robust way.
%
%Formally, a compensating pair is a pair of paths $(\pi,\pi')$ between a pair of neurons $(a,b)$, such that we have $w < 0 < w'$, for $w,w'$ the products of weight seen along $\pi$ and $\pi'$. Ignoring the (ReLU) activation functions, the weight of $b$ is loaded with $w \cdot weight(a)$ by $\pi$, while it is loaded with $w' \cdot weight(a)$ by $\pi'$. That is, it is loaded by $(w+w') weight(a)$. As $w,w'$ have opposite sign, they will compensate (partly) each other. The compensation is only partial due to the ReLU activation seen along the way of $\pi$ which can "clip" a part of $w \cdot weight(a)$, and similarly for $\pi'$. However, it is very hard to evaluate by how much without explicitly considering both phases of the ReLUs, which all the efficient tools try to avoid because it is very expansive (could be exponential in the number of such ReLU nodes opened).

%Our first main contribution is to formally show, in Theorem \ref{th1}, that compensation is the sole reason for the inaccuracies as (most) efficient algorithms will compute exact bounds for all neurons if there is no compensating pair of paths at all.
%While this theorem is theoretically interesting, it is not usable in practice as (almost) all networks have some compensating pairs. However, this notion of compensating pairs opens a first interesting idea concerning an exact abstraction of the network using a Mixed Integer Linear Program \cite{MILP}, where the weight of each neuron is a linear variable, and ReLU node may be associated with binary variables (exact encoding) or linear variables (overapproximation). While LP tools can scale to thousands of linear variables, MILP encoding can only be solved for a limited number of binary variables. This suggests that a simpler encoding could be used for those ReLUs that are not on compensating pairs, as their precise outcome may not be necessary.

%Our second main contribution is to show formally in Theorem \ref{th2}, that 
%encoding all ReLU nodes on a pair of compensating paths with a binary variable,
%and using linear relaxation for the other ReLU nodes, will lead to exact bounds for (most) of the algorithms considered. This theorem allows to restrict the number of integer variables, and thus to obtain encodings that are faster to solve. Practically, however, (almost) all ReLU nodes are on some compensating path, and using this exact restricted MILP encoding will be too time consuming.

%Our third main contribution is more practical, proposing Algorithm \ref{algo1} based on this knowledge that compensating pair of paths are the reason for inaccuracy. The idea is thus to use this information to rank the ReLU nodes in terms of importance, and only keep the most important ones as binary variables, and use linear relaxation for the least important ones.
%%More precisely, the algorithm will, as DeepPoly, consider layers one by one and neurons $b$ %on this layer one by one, selecting the heaviest pairs of compensating paths ending in $b$
%%and associating these nodes with a binary variable. Then an MILP tool such as Gurobi is used %to compute the lower and upper bound for node $b$. 
%Overall, the worst case complexity of algorithm \ref{algo1} is lower than $O(N 2^K LP(N))$, where $N$ is the number of nodes of the DNN, $K$ the number of ReLU nodes selected as binary variable, and $LP(N)$ is the (polynomial time) complexity of solving a linear program representing a DNN with $N$ nodes. This complexity is an upper bound, as e.g. Gurobi is fairly efficient and never need to consider all of the $2^K$ ReLU configurations to compute the bounds. Keeping $K$ reasonably low thus provides an efficient algorithm. 
%By design, it will never run into a complexity wall (unlike the full MILP encoding), although it can take a while on large networks because of the linear factor $N$ in the number of nodes. An additional interesting point is that it is extremely easy to parallelize, as all the nodes in the same layer can be run in parallel. We verify experimentally that the algorithm offers interesting trade-offs, by testing on local robustness for DNNs trained "naturally" (and thus difficult to verify).


%KSM: I suggest we move this to experimental evaluation
%This paper does not focus on producing the most efficient tool, and we did not spend engineering efforts to optimize it. The focus is instead on the novel notion of compensation, the associated methodology and its evaluation. For instance, our implementation is fully in Python, with uncompetitive runtime for our DeepPoly implementation ($\approx 100$ slower than in CROWN). Still, evaluation of the methodology versus even the most efficient tools reveals a lot of potential for the notion of compensation, opening up several opportunities for applying it in different contexts of DNN verification (see Section \ref{Discussion}). 


	
	\section{Notations and Preliminaries}
	
	In this paper, we will use lower case latin $a$ for scalars, bold $\boldsymbol{z}$ for vectors, 
	capitalized bold $\boldsymbol{W}$ for matrices, similar to notations in \cite{crown}.
	To simplify the notations, we restrict the presentation to feed-forward, 
	fully connected ReLU Deep Neural Networks (DNN for short), where the activation function is $\ReLU : \mathbb{R} \rightarrow \mathbb{R}$ with
	$\ReLU(x)=x$ for $x \geq 0$ and $\ReLU(x)=0$ for $x \leq 0$, which we extend componentwise on vectors.
	
	%In this paper, we will not use tensors with a dimension higher than matrices: those will be flattened.
	
	%\subsection{Neural Network and Verification}
	
	
	% testtesttesttest
	An $\ell$-layer DNN is provided by $\ell$ weight matrices 
	$\boldsymbol{W}^i \in \mathbb{R}^{d_i\times d_{i-1}}$
	and $\ell$ bias vectors $\vb^i \in \mathbb{R}^{d_i}$, for $i=1, \ldots, \ell$.
	We call $d_i$ the number of neurons of hidden layer 
	$i \in \{1, \ldots, \ell-1\}$,
	$d_0$ the input dimension, and $d_\ell$ the output dimension.
	
	Given an input vector $\boldsymbol{z}^0 \in \mathbb{R}^{d_0}$, 
	denoting $\hat{\boldsymbol{z}}^{0}={\boldsymbol{z}}^0$, we define inductively the value vectors $\boldsymbol{z}^i,\hat{\vz}^i$ at layer $1 \leq i \leq \ell$ with
	\begin{align*}
		\boldsymbol{z}^{i} = \boldsymbol{W}^i\cdot \hat{\boldsymbol{z}}^{i-1}+ \vb^i \qquad \, \qquad
		\hat{\boldsymbol{z}}^{i} = \ReLU({\boldsymbol{z}}^i).
	\end{align*} 
	
	The vector $\hat{\boldsymbol{z}}$ is called post-activation values, 
	$\boldsymbol{z}$ is called pre-activation values, 
	and $\boldsymbol{z}^{i}_j$ is used to call the $j$-th neuron in the $i$-th layer. 
	For $\boldsymbol{x}=\vz^0$ the (vector of) input, we denote by $f(\boldsymbol{x})=\vz^\ell$ the output. Finally, pre- and post-activation neurons are called \emph{nodes}.
	% and when we refer to a specific node/neuron, we use $a,b,c,d,n$ to denote them, and $W_{a,b} \in \mathbb{R}$ to denote the weight from neuron $a$ to $b$. Similarly, for input $\boldsymbol{x}$, we denote by $\val_{\boldsymbol{x}}(a)$ the value of neuron $a$ when the input is $\boldsymbol{x}$.	For convenience, we write $n < z$ if neuron $n$ is on a layer before $\ell_z$, and $n \leq z$ if $n< z$ or $n=z$.
	
    In this paper, we consider the {\em global} verification problem, where we optimize over all image $I$ and all perturbation $I'$ of $I$ with $|I-I'| \leq \varepsilon$. 	We will consider three kinds of variables: 
    \begin{itemize}
    \item $x_j,\hat{x}_j$, for nodes $j$ with input image $I$, 
    \item $x'_j,\hat{x}'_j$, for nodes $j$ with input the perturbed $I'$, 
    \item  $y_j = x_j - x'_j$ the {\em diff variable}, with 
    $\hat{y}_j = \hat{x}_j - \hat{x}'_j$ (and {\em not} 
    $\hat{y}_i = \ReLU(x_j-x'_j)$), similarly than in the 
	Interleaving Twin-Network Encoding (ITNE) model \cite{lipshitz}. 
    \end{itemize}
     
    %Concerning the verification problem, we focus on the global-robustness question. Global robustness asks to determine how the output of a neural network will be affected under a certain kind of small perturbations to any possible input. In this view, Lipschitz continuity is a good characterization of global robustness.
	
	
	
	\iffalse
	
	\section{Global robustness and Lipschitz constant}
	
	
	Recall the definition of Lipschitz continuity:
	under distance $d$, a function $f(x)$ is Lipschitz continuous with respect to constant $K$ if:
	\begin{align*}
		\forall \boldsymbol{x} \forall\boldsymbol{y} (|f(\boldsymbol{x}) -f(\boldsymbol{y}) |\leq K|\boldsymbol{x}-\boldsymbol{y}|)
	\end{align*} 
	In our practice, when we need global robustness, we will compute an optimization question respect to a certain number $\varepsilon$:	\begin{align}\label{global_robustness}
		\max_{|\boldsymbol{x}-\boldsymbol{y}| \leq \varepsilon} |f(\boldsymbol{x}) -f(\boldsymbol{y}) |
	\end{align} And this will lead to the following definition
	
	\begin{definition}[$\varepsilon$-diff bound]
		Suppose we have a function $f$ from $\mathbb{R}^n$ to $\mathbb{R}^m$ and $||$ is $L_\infty$ norm. 
		
		For a number $\varepsilon\in\mathbb{R}$, an $\varepsilon$-diff bound $D_\varepsilon$ is a number such that for any inputs $x,y$: \begin{align*}
			|x-y|\leq \varepsilon \implies |f(x)-f(y)| \leq D_\varepsilon \cdot \varepsilon
		\end{align*}
		
	\end{definition}
	
	From $\varepsilon$-diff bound, we cannot directly obtain a Lipschitz bound for the function, but we can get the following weaker bound:
	
	\begin{definition}[Lipschitz above $\varepsilon$ constant]
		Suppose we have a function $f$ from $\mathbb{R}^n$ to $\mathbb{R}^m$ and $||$ is $L_\infty$ norm. 
		
		For a number $\varepsilon\in\mathbb{R}$, a Lipschitz above $\varepsilon$ constant  $K_\varepsilon$,  is a number such that for any inputs $x,y$: \begin{align*}
			|x-y|\geq \varepsilon &\implies |f(x)-f(y)| \leq K_\varepsilon \cdot |x-y|\\
			|x-y|<\varepsilon &\implies |f(x)-f(y)| \leq K_\varepsilon \cdot \varepsilon\\
		\end{align*}		
	\end{definition}
	
	
	\begin{proposition}
		
		Suppose $D$ is an $\varepsilon$-diff bound for $f(x)$. Then for any $N\in\mathbb{Z}^+$, $D\frac{N+1}{N}$ is a Lipschitz about $N\varepsilon$ constant.
		
		That is, for any two inputs $x,y$, if \begin{align*}
			|x-y|\leq \varepsilon \implies |f(x)-f(y)| \leq D \cdot \varepsilon,
		\end{align*} then 	 \begin{align*}
			|x-y|\geq N\varepsilon &\implies |f(x)-f(y)| \leq D\frac{N+1}{N} \cdot |x-y|\\
			|x-y|<N\varepsilon &\implies |f(x)-f(y)| \leq D\frac{N+1}{N} \cdot N\varepsilon\\
		\end{align*}
	\end{proposition}
	
	\textbf{Proof.} We fix the number $N\in\mathbb{Z}^+$ and assume that we have two inputs $x, y$.
	
	The first case, $|x-y|\geq N\varepsilon$. Then we assume $|x-y| \in [M\varepsilon ,  (M+1)\varepsilon]$ for another integer $M\geq N$. Then we can divide the line segment between $x, y$ into $M+1$ pieces: $x_0 = x, x_1, x_2, \cdots, x_{M+1} = y$ such that $|x_i-x_{i+1}| \leq \varepsilon$ and apply the definition of $\varepsilon$-diff bound $D$ for each pieces:\begin{align*}
		|f(x)-f(y)| &= |f(x_{M+1})-f(x_M)+\cdots+f(x_1)-f(x_0)|\\
		&\leq |f(x_{M+1})-f(x_M)|+\cdots+|f(x_1)-f(x_0)|\\
		&\leq D\varepsilon + \cdots +D\varepsilon = (M+1)D\varepsilon
	\end{align*}
	Hence,\begin{align*}
		|f(x)-f(y)| &\leq (M+1)D\varepsilon \leq D\cdot (M+1)\varepsilon \frac{|x-y|}{M\varepsilon}\\
		&= D\cdot\frac{M+1}{M} |x-y|	\leq   D\cdot\frac{N+1}{N} |x-y|		
	\end{align*}
	The second case, $|x-y|< N\varepsilon$. Similarly we can divide the line segment between $x, y$ into $N$ pieces and then $|f(x)-f(y)|\leq D N\varepsilon\leq |f(x)-f(y)|\leq D \frac{N+1}{N} N\varepsilon$.
	
	This ends the proof.
	\hfill $\square$
	
	In practice, Lipschitz above $\varepsilon$ constant is already sufficient, since in most cases we care about the absolute difference under input perturbations, not the ratio. By combining the above proposition, the computation of $\varepsilon$-diff bound can satisfy our aim.
	
	
	Moreover, we have one more proposition connecting $\varepsilon$-diff bound and Lipschitz above $\varepsilon$ constant.
	
	
	\begin{proposition}
		For any $N\in\mathbb{Z}^+$, suppose for any $a$ in $\{\frac{N}{N}\varepsilon,\frac{N+1}{N}\varepsilon,\cdots, \frac{2N-1}{N}\varepsilon\}$, $D$ is an $a$-diff bound for $f(x)$. 
		
		Then $D\frac{N+1}{N}$ is a Lipschitz about $\varepsilon$ constant:\begin{align*}
			|x-y|\geq \varepsilon &\implies |f(x)-f(y)| \leq D\frac{N+1}{N} \cdot |x-y|\\
			|x-y|<\varepsilon &\implies |f(x)-f(y)| \leq D\frac{N+1}{N} \cdot \varepsilon\\
		\end{align*}
	\end{proposition}
	
	\textbf{Proof.}
	We fix the number $N\in\mathbb{Z}^+$ and assume we have two inputs $x, y$.
	
	For the case that $|x-y|<\varepsilon$, this is trivial by definition.
	
	For the case that $|x-y|\geq \varepsilon$, there exists a sum $x_1+x_2+\cdots+x_n$ by numbers from (allowing repetitions) $\{\frac{N}{N}\varepsilon,\frac{N+1}{N}\varepsilon,\cdots, \frac{2N-1}{N}\varepsilon\}$ such that \begin{align*}
		\varepsilon \leq x_1+x_2+\cdots+x_n -\frac{1}{N}\varepsilon \leq |x-y| \leq x_1+x_2+\cdots+x_n
	\end{align*}
	By assumption, divide the line segment from $x$ to $y$ into pieces according to $x_1, x_2,\cdots,x_n$, then we will have $$|f(x)-f(y)|\leq Dx_1+Dx_2+\cdots+Dx_n.$$
	
	Hence,\begin{align*}
		\dfrac{|f(x)-f(y)|}{|x-y|} &\leq \dfrac{Dx_1+Dx_2+\cdots+Dx_n}{x_1+x_2+\cdots+x_n -\frac{1}{N}\varepsilon}\\
		& \leq D\cdot( 1+  \dfrac{\frac{1}{N}\varepsilon}{\varepsilon})= D \frac{N+1}{N}\\
	\end{align*}
	This ends the proof.
	\hfill $\square$
	
	\fi

	
	%\section{MILP for local robustness}
	
	
	
	\subsection{MILP encoding for local ReLU}
	
	Mixed Integer Linear Programming (MILP) can encode faithfully ReLU DNNs:
	For an unstable neuron $n$, that is with values 
    $x \in [\LB(n),\UB(n)]$ with $\LB(n)<0<\UB(n)$, 
    the value $\hat{x}$ of $\ReLU(x)$ can be encoded exactly in an MILP formula with one binary / integer variable $a$ valued in $\{0,1\}$, using constants $\UB(n),\LB(n)$ with 4 constraints \cite{MILP}:
	
	\vspace{-0.4cm}
	\begin{equation} 
        \hat{x} \geq x \, \wedge \, \hat{x} \geq 0 \, \wedge \, \hat{x} \leq \UB(n) a \, \wedge \, \hat{x} \leq x-\LB(n) (1-a)
		\label{eq11}
	\end{equation}
	
\begin{proposition}
\cite{MILP}
\label{Prop1}
A solution $x,\hat{x},a$  of the above MILP program satisfies $\hat{x} = \ReLU(x)$,
and $a=1$ if $x> 0$ and $a=0$ if $x< 0$ (both are possible if $x=0$).
\end{proposition}

	%For all $x \in [\LB(n),\UB(n)] \setminus 0$, there exists a unique solution $(a,\hat{x})$ that meets these constraints, with $\hat{x}=\ReLU(x)$ \cite{MILP}. The value of $a$ is 0 if $x < 0$, and 1 if $x>0$, and can be either if $x=0$. This encoding approach can be applied to every (unstable) ReLU node, and optimizing its value can help getting more accurate bounds. However, for networks with hundreds of {\em unstable} nodes, the resulting MILP formulation will contain numerous integer variables and generally bounds obtained will not be accurate, even using powerful commercial solvers such as Gurobi.
	
    \iffalse
	The global structure is as follows, using Gurobi as an example:
	\begin{enumerate}
		\item For each input node, each output node, and each pre-activation and post-activation node in the hidden layers,  set one variable. 
		\item Set constraints for input nodes.
		\item For each pre-activation node in a hidden layer (and each output node), set linear constraints relating them to the post-activation or input nodes in the previous layer they connect to.
		\item Between pre- and post- activation nodes, set the MILP constraint described above.
	\end{enumerate} 
    
    \fi
    
    In the whole MILP model, each unstable ReLU is encoded in the above way with one integer variable. The encoding from $(\hat{x}_j)_{j \text{ in layer } i}$ to 
	$(x_{j'})_{j' \text{ in layer } i+1}$ variables is simply the linear combination 
	$\boldsymbol{z}^{i} = \boldsymbol{W}^i\cdot \hat{\boldsymbol{z}}^{i-1}+ \vb^i$.
	%This exact MILP encoding is often too computationally intensive, as the worst-case complexity of MILP is exponential in the number of integer variables \cite{DivideAndSlide}.
    
    \subsection{LP relaxation}

	MILP instances can be linearly relaxed into LP over-abstraction, where variables $a$ originally restricted to integers in $\{0,1\}$ (binary) are relaxed to real numbers in the interval $[0,1]$, while maintaining the same encoding. As solving LP instances is polynomial time, this optimization is significantly more efficient. However, this efficiency comes at the cost of precision, often resulting in less stringent bounds. This approach is termed the {\em LP relaxation}.
    % We invoke a folklore result on the LP relaxation of (\ref{eq11}), for which we provide a direct and explicit proof.
	
	
	%\subsection{partial MILP}
	
	Intermediate between these 2 extreme cases, there is {\em partial MILP} 
    (pMILP for short) to get trade-offs between accuracy and runtime
	\cite{DivideAndSlide}:
	Let $X$ be a subset of the set of unstable neurons, and $n$ a neuron for which we want to compute upper and lower bounds on values: the pMILP based on $X$ to compute neuron $n$ uses the MILP encpoding (\ref{eq11}), where variable $a$ is:
	\begin{itemize}
		\item binary for neurons in $X$ (exact encoding of the ReLU),
		\item linear for neurons not in $X$ (linear relaxation).
	\end{itemize}
	
	%We will denote the above model by MILP$_X$. We say that a node is {\em opened} if it is in $X$. 
	
	%To reduce the runtime, we will limit the size of subset $X$. This a priori hurts accuracy. To recover some of this accuracy, we use an iterative approach: computing lower and upper bounds $\LB,\UB$ for neurons $n$ of a each layer iteratively, that are used when computing values of the next layer.
	
	\iffalse
	\subsection{SAS}
	
	
	In pMILP, to decide the set $X$, we introduce the method {\em Solution-Aware Scoring} (SAS)
	to evaluate accurately how opening a ReLU impacts the accuracy. Again, here we use the definition from paper CITE. For details and explanation, see CITE.
	
	
	Assume that we want to compute an upper bound for neuron $z$ on layer $\ell_z$. For each node $n<z$, we denote ($\Sol\_\max_X^z(n))_{n \leq z}$ a solution of $\mathcal{M}_X$ maximizing $z$: $\Sol\_\max_X^z(z)$ is the maximum of $z$ under $\mathcal{M}_X$; and we denote $(\sol(n))_{n \leq z} = (\Sol\_\max_\emptyset^z(n))_{n \leq z}$ a solution maximizing the value for $z$ when all ReLU use the LP relaxation. Moreover,  we define the function
	$\Improve\_\max^z(n)=$ $\sol(z) - \Sol\_\max_{\{n\}}^z(z)$, 
	accurately represents how much opening neuron $n < z$ reduces the maximum computed for $z$
	compared with using only LP. 
	
	First, SAS will call solvers to compute the LP model to get a solution, which is reasonably fast as there is no binary variables. 
	
	Next, for a neuron $b$ on the layer before layer $\ell_z$, we define:
	
	
	\vspace{-0.4cm}
	$$\Utility\_\max\nolimits^z(b) = W_{bz} \times (\sol(\hat{b})- \ReLU(\sol(b)))$$
	\vspace{-0.4cm}
	
	
	And for a neuron $a$ two layers before $\ell_z$, 
	$b$ denoting neurons in the layer $\ell$ just before $\ell_z$.
	Recall the rate $r(b)=\frac{\max(0,\UB(b))}{\max(0,\UB(b))-\min(0,\LB(b))} \in [0,1]$.
	We define:
	
	
	\begin{flalign*}
		\Delta(\hat{a}) &= \ReLU(\sol(a))-\sol(\hat{a})&&\\
		\forall b \in \ell, \Delta(b) &= W_{ab}\Delta(\hat{a})&&\\	
	\end{flalign*}
	
	\vspace{-1.2cm}
	

		\begin{subnumcases}{\forall b \in \ell, \Delta(\hat{b}) =}
			r(b)\Delta(b),&for $W_{bz} > 0$ \\
			\max(\Delta(b),-\sol(b)),&for $W_{bz} < 0$ and $\sol(b)\geq0$\\
			\max(0,\Delta(b)+\sol(b)),&for $W_{bz} < 0$ and $\sol(b)<0$ \quad \, \quad \, \quad		 
		\end{subnumcases}

	
	
	\begin{flalign*}
		\Utility\_\max\nolimits^z(a) &= \Delta(z) = -\sum_{b \in \ell} W_{bz} \Delta(\hat{b})&&
	\end{flalign*}
	
	From paper CITE, we know that $\Utility$ is a safe overapproximation in the sense of following proposition:
	
	\begin{proposition}
		$0 \leq \Improve\max^z(a) \leq \Utility\max^z(a)$. 
	\end{proposition}
	\fi
	
		\section{MILP Models for Global Robustness}

Let $N$ the number of neurons of a DNN. Compared with local robustness which considers only variables $(x'_j)_{j \leq N}$ for the perturbed image $I'$, global robustness considers also variables $(x_j)_{j \leq N}$ for the ({\em non}-fixed) input image $I$, and thus necessitates double the number of (binary) variables (for $I$ and for $I'$). Worst-case complexity being exponential in the number of binary variables, global robustness (e.g. ITNE and VHAGaR \cite{ITNE,vhagar}) is much harder than local robustness.

\subsection{Encoding $L_\infty$- and $L_1$-perturbations for input variables}
\label{s.L1}

To encode $L_\infty$-perturbations bounded by $\varepsilon$,
the linear constraint $|x_j-x'_j| \leq \varepsilon$ is used for each {\em input} neuron $j$. 

\smallskip

We now explain how to encode $L_1$-perturbations bounded by $\varepsilon$,
using additional variables that are linear only. Let $n$ be the number of input neurons. 
We use $n$ additional linear variables $(A_i)_{i \leq n}$, with the following  constrains:
\begin{align}\label{L1constraint}\begin{cases}
	A_j \geq x_j - x'_j &\text{ for all }j \leq n\\ 
	A_j \geq x'_j - x_j &\text{ for all }j \leq n\\
	\sum_{j\leq n} A_j \leq \varepsilon	&\end{cases}
\end{align}

\begin{proof}
%We prove that (\ref{L1constraint}) are equivalent with asking $||\vec{x}||_{L_1} = \sum_{j\leq n} |x_j-x'_j| \leq c$.

The first two lines of constraints (\ref{L1constraint}) are equivalent with $A_j \geq |x_j - x'_j|$. 
So the whole constraint (\ref{L1constraint}) 
implies that $\sum_{j\leq n} |x_j - x'_j| \leq \varepsilon$.

Conversely, for every instance that satisfies $\sum_{j\leq n} |x_j - x'_j| \leq \varepsilon$, then there is also an instance which satisfies constraints (\ref{L1constraint}),  by defining $A_j = |x_j - x'_j|$ for all $j \leq n$.

This shows that (\ref{L1constraint}) is equivalent with 
$||\vec{x} - \vec{x}'||_{L_1} = \sum_{j\leq n} |x_j-x'_j| \leq \varepsilon$.
\qed
\end{proof}


\subsection{Constraints for propagation in the DNN}


There are two kinds of propagation of values for variables: 
for a neuron $j$ of layer $\ell$, the value of variable $x_j$ is defined by the constraint $x_j = \sum_{i \in \ell-1} \alpha_i \hat{x_i}$, which is a simple linear operation from the output $(\hat{x_i})_{i \in \ell-1}$ of neurons $i$ in layer $\ell-1$.

In the same way, $x'_j = \sum_{i \in \ell-1} \alpha_i \hat{x}'_i$ 
for variable $x'_j$ associated with the perturbed image $I'$, using the same $\alpha_i$ associated with the DNN weights. Similarly, 
if the {\em Diff} variables $y_j = x'_j - x_j$ are considered as defined in Section 2, with $\hat{y}_j = \hat{x}_j - \hat{x}'_j$, then we simply have 
$y_j = \sum_{i \in \ell-1} \alpha_i \hat{y}_i$.

It remains to consider the (non-linear) ReLU functions from $x_j,x'_j$ and/or $y_j$ to $\hat{x_j},\hat{x}'_j$ and/or $\hat{y}_j$, which is the complex part, with different possible solutions explored in the following with different set of variables considered. 



		%(see later how we encode $L_1$ perturbations).


	
\subsection{The classical encoding of ReLUs for global robustness}
	
    The classical encoding is used (with variants) by VHAGaR \cite{vhagar} and ITNE \cite{ITNE}. For each neuron $j$, variables $x_j,\hat{x}_j$ and $x'_j,\hat{x}'_j$ are considered. For each ReLU $j$, a binary variable $a_j$ and related constraints from the classical encoding (\ref{eq11}) \cite{MILP} are used to set $\hat{x}_j$, and an {\em independent} binary variable $a'_j$ and related constraints from the classical encoding are used to set $\hat{x}_j$. We call this encoding $\mathcal{M}^{classical}$.
	
	As explained in the section \ref{s.notation}, the standard way to make the MILP model more efficient is to linearly relax some of the binary variables into linear variables. The problem is that the LP relaxation of this classical encoding is extremely inaccurate, as variables $x_j,x'_j$ dependencies are due solely on the function definition through common ancestor variables in previous layers - as this is approximated with LP, most of the dependencies will be lost. On top of being an issue with partial MILP models, LP relaxation is also used during the Branch and Bound process internal to MILP solvers to provide bounds for each branch. So even when 
	no LP relaxation is used, the classical MILP encoding of \cite{MILP}, while asymptotically accurate, will result into looser bounds in practice. 
	
	To improve this, ITNE adds some linear constraints to the model (Eq. (3) in \cite{ITNE}, see also Section \ref{s.diff}), while VHAGaR reduces the number of binary variables depending on the perturbation used \cite{vhagar}.
	
	
	%We will compare the different methods in Table \ref{table.classical}.


	
	
	


	
	%It contains twice as many variables as for local robustness. 
	%As the worst-case complexity of solving an MILP model is exponential in the number of binary variables, it will be computationally intractable to compute $\beta^\varepsilon_{i,j}$ in this way, but for very small DNNs. 


	\subsection{The novel {\em Diff} MILP encoding for global robustness}
	\label{s.diff}

	We now present our main methodological contribution, the {\em Diff} MILP encoding. It considers variables $x'_j,\hat{x}'_j, y_j, \hat{y}_j$, where $y_j, \hat{y}_j$ are the {\em Diff variables} \cite{diff}, with 
	$y_j = x_j - x'_j$ and $\hat{y}_j= \ReLU({x}_j) - \ReLU(x'_j)$. 
	A first important observation is that the image $I$ and the deformation $I'$ having perfectly symmetrical roles, assuming that $\gamma_j$ is an {\em upper} bound of $y_j$, then $-\gamma_j$ is a {\em lower} bound for $y_j$.


	To encode the equation describing $\hat{x}'_j$, we use 
	the MILP constraints from the classical encoding (\ref{eq11}), including
	one binary variable $a'_j$, similar to ITNE.

    However, unlike ITNE, 
	%Interleaving Twin-Network Encoding (ITNE) of \cite{lipshitz,ITNE} which relies on the classical MILP encoding (\ref{eq11}) \cite{MILP} to compute $\hat{y}_j$ from the difference between 
	%$\ReLU(x'_j + y_j) = \ReLU(x_j) = \hat{x_j}$
	%and $\hat{x}'_j$, 
	the {\em Diff} MILP model encodes the equation for $\hat{y}_j$ directly, without resorting to the classical encoding from $x_j(=y_j+x'_j)$ to 
	$\ReLU(x_j)$ and using $\hat{y}_j= \ReLU(x_j) - \ReLU(x'_j)$, which is correct but does not encode explicitly that $\hat{y}_j$ is small when $y_j$ is small, since both ReLUs can be large and they are encoded independently, see also \cite{diff}. Instead, we study the function from $(x'_j,y_j)$ to $\hat{y}_j$, and encode it into the novel MILP {\em Diff} model.
    


	%	(This model has the same binary set, although the meaning of binary variable for $y_i$ is somehow different.)
	
	%	The exact constraints for $$ \begin{align*}
		%		&\hat{y}_i \geq -\hat{x}'_i \hspace*{1ex} \wedge \hspace*{1ex} \hat{y}_i \leq -\hat{x}'_i+a\beta_i  \hspace*{1ex}\wedge\hspace*{1ex} x_i'+y_i \leq a\beta_i \hspace*{1ex}\wedge\hspace*{1ex}  x_i'+y_i \geq (1-a)\alpha_i \\
		%		&\hat{y}_i \geq -\hat{x}'_i+(x_i'+y_i) \hspace*{1ex}\wedge\hspace*{1ex} \hat{y}_i \leq -\hat{x}'_i+(x_i'+y_i) +(a-1)\alpha_i \\
		%	\end{align*} 
	%	
	%	
	%	Moreover, we can add two more natural constraints: $x_i'+y_i \geq \alpha_i \hspace*{1ex}\wedge\hspace*{1ex}  x_i'+y_i \leq \beta_i.$
	
	We first display on Figure \ref{fig.2v} the function $f(x'_j,y_j)=\hat{y}_j$ for $y_j \in [- \gamma_j, \gamma_j]$. 
	When $x'_j \leq 0$ and $x_j = x'_j+y_j \leq 0$, we have $\hat{y}_j = \ReLU(x_j) - \ReLU(x'_j) = 0-0=0$. For $x'_j \geq 0$ and $x_j = x'_j+y_j \geq 0$, we have $\hat{y}_j= \ReLU(x_j)-\ReLU(x'_j)=x_j - x'_j = y_j$. 
    Between these two linear planes, we have two more linear planes presenting intermediate cases, so 4 linear planes in total.
	
	\begin{figure}[b!]
		\centering
	\hspace*{-5ex}
	\begin{tikzpicture}[scale=0.65]
		\begin{axis}[	axis on top, xlabel = \(x'_j\),
			ylabel = {\(y_j\)}, zlabel = \(\hat{y}_j\),
			set layers=default,
			xmax = 4, xmin = -4,
			ymax = 1, ymin = -1,		
			zmax = 1, zmin = -1,
			unit vector ratio=1 1 1, scale=2.5,  ytick   = {-1,0,1},
			yticklabels = {$-\gamma_j$,$0$,$\gamma_j$}, xtick = {0},
			xticklabels = {$0$}, ztick   = {-1,0,1},
			zticklabels = {$-\gamma_j$,$0$,$\gamma_j$},
			view={35}{14},
			]
			\addplot3[ fill=blue,opacity=0.1, fill opacity=0.4] 
			coordinates {
				(0,0,0) (-1,1,0) (-4,1,0) (-4,-1,0) (0,-1,0) (0,0,0)
			};
			
			\addplot3[	fill=blue,opacity=0.1, fill opacity=0.4] 
			coordinates { (0,0,0) (0,1,1) (4, 1, 1) (4, -1, -1) (1,-1,-1) (0,0,0)
			};
			
			\addplot3[	fill=blue,opacity=0.1, fill opacity=0.4	] 
			coordinates { (0,0,0)  (-1,1,0) (0,1,1) (0,0,0)
			};
			
			\addplot3[	fill=blue,opacity=0.1, fill opacity=0.4	] 
			coordinates { (0,0,0)  (0,-1,0) (1,-1,-1) (0,0,0)
			};
			
			\addplot3[only marks, mark=*, mark size=2pt, blue] coordinates {(1,-1,-1)};
			\node[label={$(\gamma_j,-\gamma_j, -\gamma_j)$}] at (axis cs: 1.2, -0.5 ,-1) {};
			
			\addplot3[only marks, mark=*, mark size=2pt, blue] coordinates {(-1,1,0)};
			\node[label={$(-\gamma_j,\gamma_j, 0)$}] at (axis cs: -1, 0.8 ,0) {};			
		\end{axis}
	\end{tikzpicture}
	\caption{The function computing $\hat{y}_j=\ReLU(x_j=y_j+x'_j)-\ReLU(x'_j)$ 
	depending on $y_j$ and on $x'_j$.}
	\label{fig.2v}
\end{figure}
	

Recall first that $a'_j$ is the binary variables used in the 
 classical MILP encoding for $\hat{x}'_j=\ReLU(x'_j)$.
We set $a'=a'_j$ (whose value is already settled, see Proposition \ref{Prop1}),
and use it in the MILP encoding of $\hat{y}_j = \ReLU(x_j)-\ReLU(x'_j)$:

\newpage

\begin{proposition}
    \label{Prop2}
Assuming $y_j \in [-\gamma_j, \gamma_j]$ and $x'_j \in [\alpha_j,\beta_j]$, we have that $\hat{y}_j = \ReLU(x_j=y_j + x'_j) - \ReLU(x'_j)$ is the solution of the MILP program:
	\begin{align*}
		& \begin{aligned}
			y_j + x'_j &\leq a\beta_j        &
			y_j &+ x'_j \geq (1-a)\alpha_j \\
			x'_j       &\leq a'\beta_j       & 
			x'_j       &\geq (1-a')\alpha_j \\
			\hat{y}_j  &\leq a\gamma_j       &
			\hat{y}_j  &\geq -a'\gamma_j \\
			\hat{y}_j  &\leq y_j + x'_j + (1-a)(-\alpha_j) &
			\hat{y}_j  &\geq y_j + x'_j + a'(-\beta_j) \\
			\hat{y}_j  &\leq -x'_j + a\beta_j &
			\hat{y}_j  &\geq -x'_j + (1-a')\alpha_j \\
			\hat{y}_j  &\leq y_j + (1-a)\gamma_j  &
			\hat{y}_j  &\geq y_j - (1-a')\gamma_j
		\end{aligned}
	\end{align*} 
    where $a,a' \in \{0,1\}$ are binary variables, 
    and $a'$ is shared with the classical MILP
    encoding for $\hat{x}'_j=\ReLU(x'_j)$.
\end{proposition}

So to encode both $\hat{y}_j$ and $\hat{x}'_j$, the {\em Diff} encoding needs 2 binary variables, similar to the classical encoding for $\hat{x}_j$ and $\hat{x}'_j$. 

\begin{proof}
First, we know the constraints for $\hat{x}_j'=\ReLU(x_j')$ are exact, by Prop \ref{Prop1}. We now do a case analysis depending on the value of both binary variables. 
%We have 2 binary variables and 4 cases in total.
 We check that in all 4 cases $\hat{y}_j = \ReLU(y_j+x'_j)-\ReLU(x'_j)$.

\textbf{$\mathbf{a = 0, a' = 0}$:}  In this case $y_j+x'_j\leq 0$ and $x'_j \leq 0 $ (lines 1\&2). We thus have to show that 
$\hat{y}_j = 0$. This is true thanks to inequalities in line 3.

\textbf{$\mathbf{a = 1, a' = 0}$:}  In this case $y_j+x'_j\geq 0$ and $x'_j \leq 0$ (lines 1\&2). We thus have to show that 
$\hat{y}_j = y_j+x'_j$. This is true thanks to inequalities in line 4.

\textbf{$\mathbf{a = 0, a' = 1}$:}  In this case $y_j+x'_j\leq 0$ and
$x'_j \geq 0 $ (lines 1\&2). We thus have to show that 
$\hat{y}_j = -x'_j$. This is true thanks to inequalities in line 5.

\textbf{$\mathbf{a = 1, a' = 1}$:} 
In this case, $y_j+x'_j\geq 0$ and $x'_j \geq 0 $ (lines 1\&2). We thus have to show that $\hat{y}_j = y_j$. This is true thanks to inequalities in line 6.

Notice that in each case, equations not mentioned are moot.
	\qed
\end{proof}


%The proof of Prop. \ref{Prop2} can be found in the supplementary materials.

Intuitively, the {\em diff values} will be small as they will be carrying the difference between the image $I$ and its slight perturbation $I'$.
The LP relaxation of ${\cal M}^{classical}$ will miss this totally, while the LP relaxation of the Diff model contains such important bounds, including (see 
section \ref{s.1b} and Fig.~\ref{fig.1v}):
\begin{align}
	\label{eq.lpr}
\frac{y_j-\gamma_j}{2} \leq \hat{y}_j \leq \frac{y_j+\gamma_j}{2}
\end{align}
%which is not the case of ${\cal M}^{classical}$. 
This is a simplification (due to the noticed symmetry) of the additional linear constraint (3) of \cite{ITNE} used explicitly in ITNE.
	

	\iffalse
\paragraph{A proof of Prop.~\ref{Prop2}}

We do a case analysis depending on the value of both binary variables. 
First, we know the constraints for $\hat{x_i}'=\ReLU(x_i')$ are exact (see 
Prop.~\ref{Prop1}). 

We have 2 binary variables and 4 cases in total. We only need to check that, in all 4 cases, $$\hat{y}_i = \ReLU(x'_i+y_i)-\ReLU(x'_i).$$

Case 1: if $a = 1$ and $a' = 1$, then $x'_i \geq 0 $ and $y_i+x'_i\geq 0$, then we need to show $\hat{y}_i = y_i$ based on  $\hat{x'_i} = x'_i$. This is true by the two inequalities in line 4.

Case 2: if $a = 1$ and $a' = 0$, then $x'_i \leq 0 $ and $y_i+x'_i\geq 0$, then we need to show $\hat{y}_i = y_i+x'_i$ based on  $\hat{x'_i} = 0$. This is true by the two inequalities in line 6.

Case 3: if $a = 0$ and $a' = 0$, then $x'_i \leq 0 $ and $y_i+x'_i\leq 0$, then we need to show $\hat{y}_i = 0$ based on  $\hat{x'_i} = 0$. This is true by the two inequalities in line 3.


Case 4: if $a = 0$ and $a' = 1$, then $x'_i \geq 0 $ and $y_i+x'_i\leq 0$, then we need to show $\hat{y}_i = -x'_i$ based on  $\hat{x'_i} = x'_i$. This is true by the two inequalities in line 5.

\fi

%\begin{proof}

%\end{proof}
	
	%	\begin{align*}
		%		& y_i+x'_i \leq a\beta_i \quad\wedge \quad y_i+x'_i\geq (1-a)\alpha_i\\	
		%		& x_i' \leq a'\beta_i \quad\wedge \quad x_i'\geq (1-a')\alpha_i\\
		%		&\hat{y}_i \leq a\gamma_i \quad\wedge \quad	\hat{y}_i \geq -a'\gamma_i \\
		%		&	\hat{y}_i \leq y_i+(1-a)\gamma_i \quad\wedge \quad	\hat{y}_i \geq y_i - (1-a')\gamma_i \\
		%		&	\hat{y}_i \leq -x'_i+a\beta_i \quad\wedge \quad	\hat{y}_i \geq -x'_i+(1-a')\alpha_i \\
		%		&	\hat{y}_i \leq y_i+x'_i+(1-a)(-\alpha_i)\quad\wedge \quad	\hat{y}_i \geq y_i+x'_i+a'(-\beta_i) \\
		%	\end{align*} 
	
	

	%	The relaxation of this model is similar: let $a$s and $a'$s be continuous variables instead of binary/integer variables. Unlike the first model in this section, relaxing a few nodes does not lose too much accuracy.


	\iffalse

\subsection{Comparison between classical and Diff MILP models}


Intuitively, the {\em diff values} will be small as they will be carrying the difference between both the image $I$ and its perturbation $I'$.
    

	As explained in the previous section, the standard way to make the model more efficient is to rely on linearly relaxing some of the binary variables into linear variables.
    The problem is that the LP relaxation of this classical encoding is extremely inaccurate, as variables $x_j,x'_j$ dependencies are due solely on the function definition through common ancestor variables in previous layers - as this is approximated with LP, most of the dependencies will be lost. On top of being an issue with partial MILP models, LP relaxation is also used during the Branch and Bound process internal to MILP solvers to provide bounds for each branch. So even when all variables are binary, the classical MILP encoding 
	of \cite{MILP}, while fully accurate, will likely result into looser bounds as well. Notice that this was already witnessed in \cite{ITNE}, and some additional constraints were added to make the classical model more accurate:
	the {\em diff variables} were used as specific constraints added to the classical encoding (Eq. (3) in \cite{ITNE}). 
	%We will compare the different methods in Table \ref{table.classical}.

	\fi
	

 


	
    
    
    


    \subsection{A decoupled {\em 2b+1} model}

	We propose a variant of the {\em Diff} encoding: 
	the binary variables $a'$ appearing in 
    Prop. \ref{Prop2} can be decoupled from the binary variable $a'_j$
	appearing in the classical encoding of $\hat{x}'_j = \ReLU(x'_j)$. 
	This would mean having 3 binary variables, which would be too costly. 
	The variables $a'_j$ used for $x'_j$ is thus linearly relaxed, while
	both variables $a,a'$ used in the encoding of $\hat{y}_j$ are kept binary. We call this the {\em 2b+1} model, which uses 2 binary variables per neuron, 
	as the classical and {\em Diff} encodings.
	The decoupling between $a'$ and $a_j$ removes complicated constraints in the MILP encoding, which helps the solver. This however also means that  the {\em 2b+1} model is not as accurate as the {\em Diff} model asymptotically.


	\subsection{The efficient {\em 1b} model}
	\label{s.1b}

		\begin{figure}[t!]
		\centering
	\hspace*{10ex}\begin{tikzpicture}
		\begin{axis}[
			xlabel={$y_i$},
			ylabel={$\hat{y}_i$},
			xmin=-2, xmax=2,
			ymin=-2, ymax=2,
			axis lines=center,
			samples=100, 
			unit vector ratio=1 1 1, scale=1, xtick   = {-2,2},
			xticklabels = {$-\gamma_i$,$\gamma_i$},
			yticklabels = {},
			]
			\addplot[blue, thick, fill=blue, fill opacity=0.4] {x} \closedcycle; 
			\addplot[blue, thick] {0}; 
			
			\addplot[only marks, mark=*, mark size=2pt, blue] coordinates {(-2,-2)};
			\node[label={above:$(-\gamma_i,-\gamma_i)$}] at (axis cs: -1.24, -2.2) {};
			
			\addplot[only marks, mark=*, mark size=2pt, blue] coordinates {(2,2)};
			\node[label={above:$(\gamma_i,\gamma_i)$}] at (axis cs: 1.4, 1.65) {};
		\addplot[dashed, thick, blue] coordinates {(-2,-2) (2,0)};
			\addplot[dashed, thick, blue] coordinates {(-2,0) (2,2)};
			\node[label={above:\scriptsize $LP\ relaxation$}] at (axis cs: -0.7, 0.8) {};
			\node[label={above:\scriptsize $LP\ relaxation$}] at (axis cs: 0.5, -1.4) {};
		\end{axis}
	\end{tikzpicture}
\caption{Possible values of $\hat{y}_i$ depending on $y_i$ for the {\em 1b} model.}
	\label{fig.1v}
\end{figure}



	To simplify further and limit the number of (binary) variables, 
    we introduce an efficient {\em 1b} model which only uses the diff variables $y_i,\hat{y}_i$. Again, we study the range of values $\hat{y}_i$ can take according to the value of $y_i$, as depicted in Fig.~\ref{fig.1v}. This is a projection of Fig.~\ref{fig.2v} onto $y_i,\hat{y}_i$.
    

    \begin{proposition}
    \label{prop3}
    Given $y_i \in [-\gamma_i,\gamma_i]$, 
    $\hat{y}_i$ is a solution of the MILP program :\begin{align*}
		\hat{y}_i &\leq a \gamma_i               &\quad \hat{y}_i &\geq y_i - a \gamma_i \\
		\hat{y}_i &\geq (a-1) \gamma_i           &\quad \hat{y}_i &\leq y_i + (1-a) \gamma_i
	\end{align*} where $a \in \{0,1\}$ is a binary variable.
	\end{proposition}

    We can easily check that the LP relaxation of this MILP program,
	depicted in broken lines on Fig.~\ref{fig.1v}, is exactly (\ref{eq.lpr}).
    
	As the {\em 1b} model is a projection of the {\em Diff} model and also of the {\em 2b+1} models, 	the LP relaxations of the {\em Diff} and the {\em 2b+1} models contain (\ref{eq.lpr}). 
	%Notice that (\ref{eq.lpr}) corresponds to Eq. (3) in \cite{ITNE}, 
	%which is added explicitly to the classical model in their ITNE model.
	
    
	
	

	\iffalse
	Based on above constraints, we can sketch this simplified model:
	\begin{enumerate}
		\item For each input node, each output node, and each pre-activation and post-activation node in the hidden layers,  set one variable $y_i$. 
		\item Set constraints for input nodes.
		\item Set linear constraints . In this case, since the meaning of $y_i$ is $x_i-x'_i$, this constraints will not use the bias.
		\item Between pre- and post- activation nodes, set the MILP constraint described above.
	\end{enumerate}
	
	The key point is that, although this model sets 3 variables (and their binary variables) for each node in the network, only $y_i$  contributes to the final results, and we can ignore $x_i,x_i'$ (and their binary variables) during the optimization.
	
	As a result, we can relax the binary variables used to $\hat{x}_i = \ReLU(x_i)$ and $\hat{x}'_i = \ReLU(x'_i)$.
	\fi


\newpage	
	
\section{Real-time verification}

%As explained in the introduction, 
The objective we want to compute upper and lower bounds on is 
$\beta^{\varepsilon}_{C,D} = \max_{|I-I'| \leq \varepsilon}((x_C - x'_C)- 
(x_D - x'_D)) = \max_{|I-I'| \leq \varepsilon}(y_C - y_D)$, for $x_{C,D},x'_{C,D}, y_{C,D}$ the value of the output neuron associated with class $C,D$ from input image $I$, perturbed image $I'$ or the differential between $I,I'$ respectively.

Upper bounds  $\bar{\beta}^{\varepsilon}_{C,D} \geq \beta^{\varepsilon}_{C,D}$ will allow us to verify in real-time whether a new incoming image can be certified to be robust for the perturbation. Lower bounds
$\underline{\beta}^{\varepsilon}_{C,D} \leq \beta^{\varepsilon}_{C,D}$
will allow us to evaluate how large is the gap= 
$\frac{(\bar{\beta}^{\varepsilon}_{C,D} - \underline{\beta}^{\varepsilon}_{C,D})} {\underline{\beta}^{\varepsilon}_{C,D}}$. MILP solvers, e.g. Gurobi, provide both an upper bound ("bound") $\bar{\beta}^{\varepsilon}_{C,D}$
as well as a lower bound ("solution") $\underline{\beta}^{\varepsilon}_{C,D}$.


Related bounds were proposed in \cite{vhagar}, for an evaluation purpose only (how narrow is the gap) rather than to verify robustness in real-time. 
Namely, VHAGaR \cite{vhagar} defines $\alpha^{\varepsilon}_{C,D}$, computing the perturbation necessary to switch an image classified as class $C$ into an image classified as class $D$:
\begin{align*}
	\alpha^{\varepsilon}_{C,D} &= \max_{|I-I'| \leq \varepsilon} (x_C-\max_{E\neq C}x_E) &\text{s.t. }  x'_D \geq \max_{E\neq D}x'_E \wedge x_C \geq \max_{E\neq C}x_E 
\end{align*}

Notice that  
$\alpha^{\varepsilon}_{C,D}$ is not symmetrical, while 
$\beta^{\varepsilon}_{C,D}$ is.


    
Assume we have computed (offline) $\bar{\beta}^{\varepsilon}_{C,D}$ for each $C,D$. Now in real-time:
\begin{enumerate}  
  \item[0] Run inference on the DNN to obtain the values $x_D$ of output neuron for each class D. Let $C$ be the class with maximal $x_{C}$. This is the predicted class from the DNN. This is the basic operation commonly done for DNN classification, without any change.
  \item[1] Compute for each class $D \neq C$
  a bit $b_D=1$ iff $x_D +\bar{\beta}^{\varepsilon}_{C,D} > x_C$.
  \item[2] Return "certified robust" iff $\bigvee_{D \neq C}(b_D)=0$.
\end{enumerate}

\begin{proposition}
  If "certified robust" is returned over input $I$, 
  then for all perturbed input $I'$ with  $|I - I'| \leq \varepsilon$,
  the DNN has the same decision on $I$ and on $I'$, i.e. the DNN is robust around $I$ for perturbation $\varepsilon$.
\end{proposition}

\begin{proof}
  Let $C$ be the decision of the DNN on $I$.
Assume by contradiction that $D \neq C$ is the decision on $I'$. 
  We know that 
  $x_D +\bar{\beta}^{\varepsilon}_{C,D} < x_C$
  because "certified robust" is returned over input $I$.
  That  is, $x_D - x_C +\bar{\beta}^{\varepsilon}_{C,D} < 0$

  By definition of $\bar{\beta}^{\varepsilon}_{C,D}$,
  we have $x'_D - x'_C + x_C - x_D \leq \bar{\beta}^{\varepsilon}_{C,D}$.
  ($I$ and $I'$ have symmetrical roles), 
  that is 
  $x'_D - x'_C \leq x_D - x_C + \bar{\beta}^{\varepsilon}_{C,D} < 0$.
  That is $x'_D < x'_C$, a contradiction with $D$ is the decision on $I'$.
  \qed
\end{proof}

The overhead (1,2 above) after 0 amounts to $2k-1$ operations, 
where $k=\#$Class. 
More precisely,
$(k-1)$ ADD(+), 
$(k-1)$ COMP(<) and $1$ OR($\bigvee$) of ($k-1$) bits.
That is, 19 operations in total for $k=10$ classes. This can be realized extremely fast, even on limited hardware.
	
	\subsection*{A trick of $L_1$ norm constraints on input}

A trick to set $L_1$ norm constraints on the input space into linear constraints is described below.

Suppose the dimension of input space is $n$, and $x_i,i\leq n$ are the input variables, $\vec{x}$ is the input vector, and we hope to set a constraints of $\vec{x}$ that $\|\vec{x}\| \leq c$,  then we can set the constraints by add variables $A_i,i<n$ with the only constraints \begin{align}\label{L1constraint}\begin{cases}
	A_i \geq x_i &\text{ for }i \leq n\\ 
	A_i\geq -x_i &\text{ for }i \leq n\\
	\Sigma_{i\leq n} A_i \leq c	&\end{cases}
\end{align}

We show that this constraint is equivalent to $\|\vec{x}\| \leq c$.

The first two lines of constraints \ref{L1constraint} is equivalent to that $A_i\geq |x_i|$. So the whole constraints is non-weaker than $\|\vec{x}\| \leq c$.

Next, for every instance that satisfy $\|\vec{x}\| \leq c$, then there is also an instance which satisfy constraints \ref{L1constraint}: simply let all $A_i = |x_i|$. And this shows that $\|\vec{x}\| \leq c$ is non-weaker than constraints \ref{L1constraint}. Therefore they are equivalent.




\subsection*{A proof of constraints}


We will show that the constraints corresponding to \ref{fig.2v} is exact.

We only need to consider all cases of binary variable. First, we know the constraints for $\hat{x_i}'=\ReLU(x_i')$ are exact. 

We have 2 binary variables and 4 cases in total. We only need to check that, in all 4 cases, $$\hat{y}_i = \ReLU(x'_i+y_i)-\ReLU(x'_i).$$

Case 1: if $a = 1$ and $a' = 1$, then $x'_i \geq 0 $ and $y_i+x'_i\geq 0$, then we need to show $\hat{y}_i = y_i$ based on  $\hat{x'_i} = x'_i$. This is true by the two inequalities in line 4.

Case 2: if $a = 1$ and $a' = 0$, then $x'_i \leq 0 $ and $y_i+x'_i\geq 0$, then we need to show $\hat{y}_i = y_i+x'_i$ based on  $\hat{x'_i} = 0$. This is true by the two inequalities in line 6.

Case 3: if $a = 0$ and $a' = 0$, then $x'_i \leq 0 $ and $y_i+x'_i\leq 0$, then we need to show $\hat{y}_i = 0$ based on  $\hat{x'_i} = 0$. This is true by the two inequalities in line 3.


Case 4: if $a = 0$ and $a' = 1$, then $x'_i \geq 0 $ and $y_i+x'_i\leq 0$, then we need to show $\hat{y}_i = -x'_i$ based on  $\hat{x'_i} = x'_i$. This is true by the two inequalities in line 5.

	
	
		\section{Experimental Evaluation}
	
	We implemented our code in Python 3.8.
	Gurobi 9.52 was used for solving the MILP problems. We conducted our evaluation on an AMD Threadripper 7970X ($32$ cores$@4.0$GHz) with 256 GB of main memory and 2 NVIDIA RTX 4090. 
	
	We consider 4 pretrained DNNs on 3 different datasets (MNIST, Fashion MNIST and CIFAR10). Namely,  MNIST-FC, a fully connected DNN available at 
	\url{https://github.com/eth-sri/eran} ("5x100"). We also considered 3 convolutional DNNs: MNIST-Conv, FMNIST-Conv, and CIFAR10-Conv, 
	from \cite{vhagar} ("conv1"). We will also consider 3 PCA-DNNs, as detailled in Section~\ref{s62}
	%The pipe considers a conjunction of $L_1 \leq 3.9$ and $L_\infty \leq .02$ perturbations, physically pertinent dimensions, and maximizes the sum of the difference in output of 10 selected points in the mesh between input and its deformation.

\iffalse	
For each benchmark, we report three values: the bound obtained (lower value is better), the solution obtained (distance to the bound depicts how close the model has converged), and worst-case (higher is better), that is the value reached when considering the solution as input to the DNN. For a fully accurate model ("2v" or classical and all variables binary), the worst-case will equal the solution, otherwise, it will be smaller due to abstraction.
\fi

%\subsection{Classical vs our "2v" model vs ITNE}
%
%
%\begin{table}[b!]
%	\centering
%	\begin{tabular}{||l|c|c|c||}\hline\hline
%		model, nbr binary var &        Bound $\downarrow$ &  Sol. &      Worst-Case $\uparrow$ \\\hline \hline
%	Classical, $100$ &    $.320$ &  $.320$ & $.017$ 
%    \\\hline
%	ITNE, $100$ &    $.042$ &  $.037$ & $.022$
%	\\ \hline
%    2v model, $100$ &    {\bf .040} &  $.037$ &  $.018$ 
%    \\\hline \hline
%     2v model New, $100$ &    {\bf .039} &  $.029$ &  $.020$ 
%    \\\hline \hline
%    Classical, $200$ &  .186  &  $.022$ & $.022$ 
%    \\\hline
%	ITNE, $200$ &    $.045$ &  $.023$ & .023
%    \\\hline
%	2v model, $200$&     {\bf .042} &  $.023$ &   .023 
%    \\\hline \hline
%	\end{tabular}
%	\caption{Comparison of the classical encoding, ITNE and our "2v" model on the pipe system with a fixed timeout of 1000s, with either $100$, 
%    or the full $200$ binary variables.}
%    \label{table.classical}
%\end{table}
%
%We start by evaluating how accurate the 2v model is vs the classical encoding vs the interleaving twin-network encoding (ITNE) from \cite{lipshitz,ITNE}, that we reimplement by considering the classical encoding plus the explicit constraints $\frac{x_i-x_i' - \gamma}{2} \leq \hat{x}_i-\hat{x}_i' \leq \frac{x_i-x_i' + \gamma}{2}$, corresponding to Eq. (3) of \cite{ITNE} (and also to our (\ref{eq.lpr})). We compare when half the ReLUs are approximated with the LP relaxation, and when all the ReLUs are encoded exactly. We consider the pipe surrogate as it is easier to verify, and set the time-out at 1000s for all models.
%
%
%Table \ref{table.classical}, $100$, confirms that the LP relaxation of the classical model is extremely poor, with bounds $8$ times worse than using our "2v" model, when 50 variables use LP relaxation. Adding explicitly LP relaxation constraints of Eq. (3) from \cite{ITNE} recovers most but not all the bound of the "2v" model.
%
%Further, even using a fully accurate model with all $100$ variables encoded as binary variables, the "2v" model produces bounds $>4$ times better than the classical model, due to the internal Branch and Bound process to compute bounds, which uses linear relaxation. Again, 
%Eq. (3) from \cite{ITNE} recovers most but not all the accuracy of the "2v" model.
%Overall, the "2v" model produces better bounds, even with short time-outs, despite its higher complexity.






\subsection{Comparison for global robustness with VHAGaR and ITNE}

We use VHAGaR benchmarks to be able to compare: 
bound $\alpha$, perturbation $L_\infty=0.05$, VHAGaR convolutional DNNs, for which the VHAGaR reduction rules of binary variables are implemented. We tested with 1000s and 14400s (4h) timeout. We report the upper bound $\bar{\alpha}$ each MILP model find. For each DNN, we report the best lower bound $\underline{\alpha}$ among the MILP models, and report the gap between $\bar{\alpha}$ and  $\underline{\alpha}$ as $\%$ of increase over $\underline{\alpha}$. We report results in Table \ref{table.classical}. Recall most robustness tools, e.g. $\alpha,\beta$-Crown, can only handle {\em local} robustness and cannot be tested.

\iffalse

\begin{table}[h!]
	\centering
	\begin{tabular}{||c||c|c|c|c|c||}\hline\hline
	Benchmark & model & Bound $\downarrow$ &  Sol. &      Worst-Case $\uparrow$ & $\%$ of gap $\downarrow$ \\\hline \hline
	MNIST   & VHAGaR & 20.51 & 9.20 & 9.20 & 118 $\%$
    \\
	M-Conv & ITNE & 12.77 & 9.41 & 9.41 & 35 $\%$
	\\ 
    TO=1000s & {\em Diff} & {\bf 11.59}  & 9.38 & 9.38 & {\bf 23 $\%$}
    \\\hline \hline

	FMNIST & VHAGaR &  $41.52$ & $21.23$ & $21.23$ & 93 $\%$ 
    \\
	F-Conv  & ITNE &  $27.57$ &  $22.12$ & $22.12$ & 24 $\%$
	\\ 
    TO=1000s & {\em Diff} &  {\bf 23.67} &  $22.31$ &  $22.31$ & {\bf 6 $\%$}
    \\\hline 


	FMNIST & VHAGaR &  $37.26$ &  $22.00$ & $22.00$ & 72 $\%$
    \\
	F-Conv  & ITNE &  $24.56$ &  $22.20$ & $22.20$ & 10 $\%$
	\\ 
    TO=14400s & {\em Diff} &    {\bf 22.56} &  $22.32$ &  $22.32$ & {\bf 1 $\%$}
    \\\hline \hline
	
	CIFAR-10  & VHAGaR & does & not & run & -
    \\
	C-Conv  & ITNE & 28.61 & 11.58 & 11.58 & $33\%$
	\\ 
    TO=1000s & {\em Diff} &  25.93  & 14.43 & 14.43 & $24\%$
	\\
	& {\em 2b+1} &  25.31 & 19.77 & - & $23\%$
	\\
	& Best &  {\bf 23.44}  &  &  & {\bf 16} $\%$
    \\\hline
    
	CIFAR-10  & VHAGaR & does & not & run & -
    \\
	C-Conv  & ITNE &  & & &
	\\ 
    TO=14400s & {\em Diff} &  {\bf 20.67}  & 19.03 & 19.03 & {\bf 6} $\%$
	\\
	 & {\em 2b+1} &  {\bf }  &  &  & $\%$
    \\\hline \hline
    

     \end{tabular}
	\caption{Comparison with VHAGaR for $L_\infty=0.05$ perturbation.}
    \label{table.classical}
\end{table}


\fi




\begin{table}[t!]
	\centering
	\begin{tabular}{||c c||c c c c c ||c|}\hline
	$L_\infty=0.05$ & Timeout & MILP model: & VHAGaR &  ITNE & {\em Diff} & {\em 2b+1} & {\em Best} \\\hline \hline
	  & 1000s & up. Bound $\overline{\alpha}$ $\downarrow$ &  20.80 & 12.32 & {\em 11.14} & {\em \bf 10.95} & {\bf \em 10.85}
    \\
	MNIST-Conv & & ($\%$ of gap $\downarrow$) & (120$\%$) & (29 $\%$) & ({\em 17} $\%$) & ({\em \bf 15} $\%$) & ({\bf \em 14} $\%$)
	\\
	 LB $\underline{\alpha}$= 9.51 & 14400s & up. Bound $\overline{\alpha}$ $\downarrow$ & 17.54 & 11.15 & {\bf \em 10.14} & {\em 10.29} & {\bf \em 10.07}
    \\
	  & & ($\%$ of gap $\downarrow$) &  (84$\%$) & (17 $\%$) & ({\bf \em 7} $\%$) & ({\em 8} $\%$) & ({\bf \em 6} $\%$) 
    \\\hline

	
	& 1000s & up. Bound $\overline{\alpha}$ $\downarrow$ & 41.52 & 26.61 & {\bf \em 23.27}  & {\em 24.09} & {\bf \em 22.78}
    \\
	FMNIST-Conv & & ($\%$ of gap $\downarrow$) & (93 $\%$) & (20 $\%$) & ({\bf \em 5} $\%$) & ({\em 10} $\%$) & ({\bf \em 3} $\%$)
    \\


	LB $\underline{\alpha}$= 22.21 & 14400s & up. Bound $\overline{\alpha}$ $\downarrow$ & 37.26 & 24.06 & {\bf \em 22.42 } & {\em  23.80} & {\bf \em 22.34}
    \\
	 & & ($\%$ of gap $\downarrow$) & (72 $\%$) &  (9 $\%$) & ({\bf \em 1} $\%$) & ({\em 9} $\%$) & ({\bf \em .6} $\%$)
    \\\hline
	

	& 1000s  & up. Bound $\overline{\alpha}$ $\downarrow$ & 36.57* & 27.15 & 
	{\em 25.94} & {\bf \em 24.52}& {\bf \em 23.29}
    \\
	CIFAR-Conv & & ($\%$ of gap $\downarrow$) & (55* $\%$) & (28 $\%$) & ({\em 24} $\%$) & ({\bf \em 20} $\%$) & ({\bf \em 15} $\%$)
	\\
    
	LB $\underline{\alpha}$= 19.04 & 14400s  & up. Bound $\overline{\alpha}$ $\downarrow$ & 31.53* & 22.73 & {\bf \em 21.77} & {\em 21.81} & {\bf \em 20.38}
    \\
	 & & ($\%$ of gap $\downarrow$) & (39* $\%$) & (12 $\%$) & ({\bf \em 10} $\%$) & ({\bf \em 10} $\%$) & ({\bf \em 5}$\%$)
		\\\hline
    
     \end{tabular}
	\caption{Comparison with VHAGaR and ITNE of upper bound $\overline{\alpha}$ and (gap to lower bound $\underline{\alpha}$), lower is better, for $L_\infty=0.05$ perturbation, average over all bounds $\alpha_{i,j}$. * for CIFAR-Conv, VHAGaR failed to compute bounds for some of the cases. For such cases, we used numbers from ITNE instead (which is overperforming VHAGaR in every test), so these numbers are optimistic evaluation for VHAGaR. Our results are in {\em italic}. {\em Best} uses the best between our {\em Diff} and our {\em 2b+1} for each $(i,j)$. Lower gaps using {\em Best} means that there are cases $(i,j)$ where {\em Diff} is much better than {\em 2b+1}, and cases $(i',j')$ where it is the opposite.}
    \label{table.classical}
\end{table}

\noindent {\bf Discussion:} This test is relatively easy, where the best upper bound $\bar{\alpha}$ found is close to the lower bound $\underline{\alpha}$ found, with at most a $6\%$ gap. Using the 1$b$ model is not useful, as its abstraction will loose more than $6\%$. 
%That is why it is not reported in Table \ref{table.classical}. 
VHAGaR computes upper bounds $\bar{\alpha}$ quite far from the lower bound, with gap $40-120\%$. Although we are using its benchmarks, VHAGaR methodology is not well-suited for $L_\infty$ (it is more efficient on {\em Patch} \cite{vhagar}): ITNE is better, with gaps from $9-17\%$.
Our method provides the {\em Best} overall gaps: $.6-6\%$, significantly tighter. Breaking up numbers, 2$b$+1 is more accurate for shorter timeout, and {\em Diff} more accurate when there is sufficient time, as expected. 
Anyway, both are better for particular classes $(i,j)$, as {\em Best} is lower than both {\em Diff} and 2$b$+1 in every test.








%\begin{table}[h!]
%	\centering
%	\begin{tabular}{||c||c||c|c|c||c|}\hline\hline
%		Benchmark & MILP model: & {\em Diff} & {\em 2b+1} & {\em 1b} & {\em Best} \\\hline \hline
%		MNIST-FC  & Bound $\downarrow$ &&&&
%		\\
%		TO=1000s & $\%$ of gap $\downarrow$ &&&&
%		\\ \hline
%		TO=14400s  & Bound $\downarrow$ &&&&
%		\\
%		& $\%$ of gap $\downarrow$ &&&&
%		
%		\\\hline \hline
%		
%		FMNIST-CNN1 & Bound $\downarrow$ & 8.90 & 7.55 & 9.84 & 7.548
%		\\
%		TO=1000s & $\%$ of gap $\downarrow$ & 112 $\%$  & 80 $\%$ & 135 $\%$ & 80 $\%$
%		\\\hline 
%		
%		
%		TO=14400s & Bound $\downarrow$ & 7.43 & 6.37 & 9.62 & 6.22
%		\\
%		& $\%$ of gap $\downarrow$ & 77 $\%$  & 53 $\%$ & 129 $\%$ & 49 $\%$
%		\\\hline \hline
%		
%		
%		CIFAR10-CNN1  & Bound $\downarrow$ & 8.30 & 8.14 & 8.17 & 7.77 
%		\\
%		TO=1000s & $\%$ of gap $\downarrow$ & 150 $\%$  & 145 $\%$ & 145 $\%$ & 133 $\%$
%		\\\hline
%		
%		TO=14400s  & Bound $\downarrow$ & 7.19 & 7.45 & 7.31 & 6.67
%		\\
%		& $\%$ of gap $\downarrow$ & 114 $\%$  & 123 $\%$ & 118 $\%$ & 99 $\%$
%		\\\hline \hline
%		
%	\end{tabular}
%	\caption{Full input dimensions}
%	\label{table.classical_our_full}
%\end{table}



\subsection{Experimental results for $L_1=1$ and PCA-DNN.}
	
\label{s62}

We then turn to test with $L_1$ perturbations and comparing between DNNs and their PCA-DNN versions (which is not supported in VHAGaR). Here, we consider MNIST-FC (not supported by VHAGaR), because it has a higher accuracy than MNIST-Conv, and to test a different architecture than Convolution Networks.

We recall that PCA-DNNs are generated following the process described in Section 5: for each PCA dimension, we report in Table \ref{table.pca} 
the accuracy loss from using PCA-DNN rather than the original DNN.
We set the dimension at the minimal dimension reaching at most 1$\%$ 
loss of accuracy, that is $20, 25$ and $60$ for  MNIST-FC, FMNIST-Conv and CIFAR10-Conv respectively.


\begin{table}[t!]
	\centering
	\begin{tabular}{|c c c c c c c c c |}\hline
		PCA dimension: & 15 & 20 & 25 & 30 & 35 & 40 & ... & 60 
		
		\\\hline
		PCA-MNIST-FC  & 2$\%$ & {\color{blue} 0$\%$} & 0$\%$& 0$\%$ & 0$\%$& 0$\%$ & &0$\%$
		\\	
 		PCA-FMNIST-Conv & 5$\%$ &	3$\%$ &	{\color{blue} 1$\%$} & 0$\%$& 0$\%$	&0$\%$ & &	0$\%$ 
		\\
		PCA-CIFAR10-Conv  & 16$\%$ &	14$\%$ &	12$\%$ & 10$\%$ &	14$\%$ &	5$\%$ &	& {\color{blue} $1\%$}
		\\\hline
	\end{tabular}

	\caption{Loss of accuracy of PCA-DNN vs DNN (no PCA) 
	according to the PCA dimension.
	{\color{blue} Blue} corresponds to the minimal dimension ensuring loss $\leq 1\%$.}
    \label{table.pca}
\end{table}

We consider the $\bar{\beta}$ bounds reached for $L_1$-perturbation  at most 1, on the 3 original DNNs MNIST-FC, FMNIST-Conv and CIFAR10-Conv, plus their PCA variants for the dimension computed in Table~\ref{table.pca}.
We report the results in Table~\ref{table.L1}, in the same format as Table~\ref{table.classical}.





\begin{table}[b!]
	\centering
	\begin{tabular}{|cc| c | c c c|c|}\hline
		$L_1=1$ & Timeout & ITNE & {\em Diff} & {\em 2b+1} & {\em 1b} & {\em Best} \\\hline \hline 
		MNIST-FC  & 1000s & 45.6 & 38.88 & 38.68 & {\bf 37.34} & {\bf 37.33}
		\\
		LB $\underline{\beta}$= 0.50 & 14400s & 45.3 & 37.24 & 36.96 & {\bf 33.68} & {\bf 33.67}
		
		\\\hline
		PCA-MNIST-FC  & 1000s & 2.52 & {\bf 2.43} & {\bf 2.43} & 2.68 & {\bf 2.41}
		\\
		LB $\underline{\beta}$= 0.12 & 14400s & 2.33 & 2.18 & {\bf 2.00} & 2.46 & {\bf 2.00}
		\\ \hline \hline
		
 		FMNIST-Conv & 1000s & 12.27 & 8.90 & {\bf 7.55} & 9.84 & {\bf 7.55}
		\\
		LB $\underline{\beta} = 4.16$ & 14400s & 11.03 & 7.43 & {\bf 6.37} & 9.62 & {\bf 6.22}
		\\\hline

		PCA-FMNIST-Conv & 1000s & 3.64 & 3.02 & 3.02 & {\bf 2.74} & {\bf 2.74}
		\\
		LB $\underline{\beta}= 0.96$ & 14400s & 2.67 & 2.34  & {\bf 1.95} & 2.52 & {\bf 1.95}
		\\\hline  \hline
		
		
		
		CIFAR10-Conv  & 1000s & 10.49 &8.30 & {\bf 8.14} & 8.17 & {\bf 7.77}
		\\
		LB $\underline{\beta}= 3.38$ & 14400s & 10.49 & {\bf 7.19} & 7.45 & 7.31 & {\bf 6.67}
		\\\hline
		PCA-CIFAR10-Conv  & 1000s & 1.70 & 1.63 & 1.54 & {\bf 1.28} & {\bf 1.28}
		\\
		LB $\underline{\beta}$= 0.0015 & 14400s & 1.67 & 1.56 & 1.52 & {\bf 1.15} & {\bf 1.15}
		\\
		 extra long timeout& 216000s & 1.66 & 1.53 & 1.36 & {\bf 0.99} & {\bf 0.99}
		
		\\\hline

	\end{tabular}
	\caption{Comparison of upper bound $\overline{\beta}$ (lower is better $\downarrow$) averaged over all bounds $\overline{\beta}_{i,j}$,
	between ITNE and {\em Best} of {\em Diff, 2b+1 and 1b}, for original DNNs as well as PCA-DNNs over MNIST, FMNIST and CIFAR datasets. %Best report is the average of the best for every case.
	 %We use PCA dimension = 20 for MNIST-FC, PCA dimension = 25 for FMNIST CNN1, and PCA dimension = 45 for CIFAR10 CNN1.
	 }
	\label{table.L1}
\end{table}



\noindent {\bf Discussion:} Compared with results on $L_\infty$ (Table~\ref{table.classical}), the upper bound $\bar{\beta}$ computed by Best is now much larger than the lower bound $\underline{\beta}$: from 1.5 to 600 times. Using 
the faster 1$b$ model now becomes pertinent. In half of the cases, 1$b$ actually produces the best results. When the timeout is large enough compared to the complexity of the model, 2$b+$1 and sometimes {\em Diff} provide better bounds. ITNE is significantly worse, producing $25\%$ worse bounds $\bar{\beta}$ on average than Best.

Comparing with and without PCA, as expected, PCA-DNNs allow to reach much smaller upper bounds $\bar{beta}$ than without PCA, $15$x for MNIST-FC, $3$x for FMNIST-Conv and $7$x for CIFAR10-Conv. MNIST-FC is harder to solve than FMNIST-Conv, although they are comparable benchmark, because fully connected architecture are richer than Convolutional architecture of the same size. PCA-CIFAR10-Conv is the hardest test, because the dimension (60) is not as reduced as for MNIST and FMNIST (20 and 25) .



	
%	\begin{table}[h!]
%		\centering
%	\begin{tabular}{||l||c|c|c||}\hline\hline
%		model, nbr binary var &        Bound $\downarrow$ &  Sol. &      Worst-Case $\uparrow$ \\\hline \hline
%		1v, $500$ & {\bf 14.97} & $.845$ & $.009$ \\\hline 
%		3v, $1000$ & $17.66$ & $.813$ & {\bf .518} \\\hline 
%		2v ITNE, $1000$ & $19.08$ & n/a & n/a \\\hline 
%		2v ITNE, $1000$ & $19.38$ & $.185$ & $.185$ \\\hline 
%	    2v, $1000$ & $16.49$ & n/a & n/a \\\hline\hline	 
%	\end{tabular}
%	\caption{Bounds on $\beta^{.5}_{6,8}$ 
%	obtained by the "1v", "3v" and "2v" models 
%	on the {\bf full dimension} MNIST DNN, 
%	for timeouts of $14400$s, when all 500 / 1000 variables are binary.}
%	\label{table.mnist}
%\end{table}


\iffalse


\begin{table}[b!]
	\centering
	\begin{tabular}{||l||c|c|c||}\hline\hline
		model, nbr binary var &        Bound$\downarrow$ &  Sol. &      Worst-Case$\uparrow$ \\\hline \hline
%1v, $400 \times 1$ & $1.414$ &  $.691$ & $.010$ \\\hline 
%3v, $400 \times 3$ & $1.186$ & $.600$ & $.003$ \\\hline 
%2v, $400 \times 2$ & $1.274$ & $.566$ & $.002$ \\\hline\hline
	 
%1v, $475 \times 1$ &  $1.408$ & $.301$ & $.008$  \\\hline 
%3v, $475 \times 3$ &  $1.153$ & $.250$ & $.006$ \\ \hline 
%2v, $475 \times 2$ &  $1.247$ & $.1957$ & $.019$ \\\hline\hline

1v, $500$ & $1.412$ & $.161$ & .057 \\\hline 
3v, $1000$ & {\bf 1.137} & $.103$ & $.065$\\\hline 
2v, $1000$ &  $1.182$ & $.084$& {\bf .084}  \\\hline
ITNE, $1000$ &  $1.371$ & $.085$& {\bf .085}  \\\hline\hline
	 
	\end{tabular}
	\caption{Comparison of "1v", "3v" and "2v" models 
	to obtain bounds on $\beta^{.5}_{6,8}$ on the {\bf 20 dimension} reduced order MNIST DNN, for timeout of 14400s, where 
	all %400, 475,  or 
	neurons use 500 / 1000 binary variables.}
	\label{table.reduced}
\end{table}
\fi


%\paragraph{Reduced Space}

\subsection{Real-Time vertification of Robustness}

We now turn to how many images can be certified robust in real-time using the {\em Best} bounds $\beta^1_{i,j}$ computed in the previous subsection. We report the $\%$ of images certified robust in Table \ref{table.cert}, as well as the latency overhead to certify them: it is half a millisecond in every case (using a single CPU core), as it depends only on the number of outputs of the DNN, which is 10 for all 6 DNNs considered. It means 2000 images can be treated per second by each CPU core, which can be deemed {\em real-time}. We set a threshold of $70\%$ to evaluate how large the $L_1$  perturbation can be while still certifying at least this threshold: 
a threshold much lower than $70\%$ would mean labelling too many incoming images as uncertified, which would degrade the system too much.



\begin{table}[t!]
	\centering
	\begin{tabular}{|c|cccc|c|}\hline
		Benchmark & $L_1=1$ & $L_1=2$ & $L_1=3$ & $L_1=4$ & Latency overhead 
		
		\\\hline 
		MNIST-FC  & 0$\%$ & 0$\%$ & 0$\%$ & 0$\%$ & 0.5ms
		\\
		
		PCA-MNIST-FC  & {\color{blue} 80$\%$}  & 21 $\%$ & 1 $\%$ & 0 $\%$ & 0.5ms
		\\ \hline 
		
 		FMNIST-Conv & 
		{\color{blue} 75$\%$}  & 50 $\%$ & 28$\%$ &	12$\%$  & 0.5ms
		\\

		PCA-FMNIST-Conv & 
		92 $\%$ &	86 $\%$ &	79 $\%$ & {\color{blue} 70$\%$}  & 0.5ms
		\\\hline
		
		CIFAR10-Conv  & 39 $\%$ & 0$\%$ & 0$\%$ & 0$\%$ & 0.5ms
		\\ 
		PCA-CIFAR10-Conv  & 92 $\%$ & {\color{blue} 72$\%$} & 61 $\%$ & 51 $\%$ & 0.5ms
		\\\hline
	\end{tabular}

	\caption{Percentage of images certified robust in real-time 
	using the Best computed $(\beta^{1}_{i,j})_{i < j \leq 10}$, 
	for different values of $L_1$-perturbations.
	{\color{blue} Blue} depicts the largest perturbation ensuring $\geq 70\%$ of images being robust. Latency overhead are the same as the number of classes is the same (=10) for all datasets.} 
	%2000 images/s can be treated per cpu core.}
    \label{table.cert}
\end{table}

With that, we can guarantee at least $70\%$ of the image as certified
for all 3 PCA-DNN label  for a $L1$-perturbation of $1$, and even go up to perturbation $2$ for CIFAR10 and even $4$ for FMNIST.
We are unable to do it for the non-PCA version, but in the case of FMNIST-Conv, with a perturbation 4 times smaller than for its PCA-version:
PCA is extremely efficient at helping certifying.

\iffalse
To remove improbable images and limit the space of search, 
we consider a PCA model order reduction \cite{Paco}: We reduced to 20 dimensions, because the MNIST $100 \times 5$ DNN considered, once run on images obtained from projecting to the 20 dimension space and projected back to the full dimension space displays the same accuracy of $97$\% as the DNN on the original images. This means that considering the reduced 20 dimensional space
does not incur any loss in accuracy, which is the only thing which matters. On this reduced space, bounds obtained are much more precise, and one can certify in real-time robustness of images.





We report in Table \ref{table.reduced} the same $\beta^{.5}_{6,8}$
as in Table \ref{table.mnist}, obtained with the same time-out. The bounds are directly comparable: the best $\beta^{.5}_{6,8}$ obtained using the reduced dimension is $>10$ times smaller than when using the full dimension ($1.137$ vs $14.97$). 
With these bounds, most images ($86\%$) can be certified in real-time robust for a perturbation $L_1 \leq .5$, and even $53\%$ with a perturbation $L_1 \leq 1$ twice as large, see Table \ref{table.cert}. We illustrate that improbable images are removed by displaying in Fig.~\ref{fig4} the worst-case obtained for $\beta^{.5}_{6,8}$, which indeed looks like a realistic MNIST instance.




	
	


	


\subsection{Experimental results for regression (Pipe strain)}


	\begin{figure*}[t!]
\includegraphics[scale=0.5]{deform.png} \hspace{0.8cm}
\includegraphics[scale=0.5]{strain.png}
\caption{2 slightly different deformations and their associated quite different strain as obtained by the "2v" model with $200 $ binary variable.}
\label{fig5}
\end{figure*}	


	For the pipe system, we compared in Table \ref{table.pipe} the different models "1v","2v","3v" to produce bounds on the sum of the difference of strain over 10 specific points of the mesh, for a physically relevant perturbation of the deformation. The bounds we found are quite accurate, with a best bound of $.0329$ obtained by the "3v" model, slightly better than the bound $.0337$ found within the same time by the "2v" model, and better than the bound found $0.356$ by the "1v" model, although this bound has been found 70 times faster due to the simpler model. The certified lower bound is not too far, at $.245$, found by the fully accurate "2v" model when all the variables are binary. We did check that this worst-case found, displayed in Fig. \ref{fig5} and which is not too far from the actual worst case that is known to be $<.0329$, is coherent with the physical dinite element model the DNN surrogate has been learnt from, hence this is not an hallucination due to the brittleness of the learnt DNN.


	
	\vspace*{1ex}
	
	\iffalse
	\begin{table}[h!]
	\begin{tabular}{|l|l|l|l|l|}\hline
		$L_1\leq 0.83$ &        Bound $\downarrow$ &  Solution $\uparrow$ &      Real $\uparrow$ &  Time \\\hline
		1v,open 100 &     {\bf 0.035613} &  0.035613 &                       0.01288 & 10608 \\\hline
		3v,open 100 &     0.040074 &  0.028934 &                      0.021441 & 10922 \\\hline
		%3v,open 100 &     0.039824 &  0.028832 &                      0.022255 & 22153 \\\hline
		2v,open 100 &     0.046719 &  0.024364 &  {\bf 0.024436} & 10922 \\\hline
	\end{tabular}
	\caption{Comparison of 1v,2v and 3v models on the pipe system with a fixed timeout of 10.000s.}
\end{table}
\fi
	
		
	\begin{table}[h!]
	\begin{tabular}{||l||c|c|c|c||}\hline\hline
		model, nbr&        Bound$\downarrow$ &  Sol. &      Worst-Case$\uparrow$ &  Time(s) \\\hline \hline
		1v, $100$ &     {\bf .0356} &  $.0356$ & $.0191$ &  1000 \\\hline
		3v, $200$&     .0414 &  .0254 &  .0166 &  1000 \\\hline
		2v, $200$&     .0418 &  .0229 &   {\bf .0229} &  1000 \\\hline 
		2v ITNE, $200$&  .0446  & .0227 &  .0221  &  1000 \\\hline\hline
		%3v, $97 \times 3$&      ?? &  ?? &  ?? & 14440 \\\hline
		3v, $200$&      {\bf .0350} &  .0272 &  .0216 & 14440 \\\hline
		%2v, $97 \times 2$&     ?? &  ?? &   ?? & 14440 \\\hline
		2v, $200$&     .0360 &  .0236 &    {\bf .0236} & 14440 \\\hline 
		2v ITNE, $200$& .0424  &  .0237  & .0228   &  14400 \\\hline\hline
		3v, $200$&     {\bf .0329} &  .0277 &  .0165 & 72000 \\\hline
		2v, $200$&     .0337 &  .0245 &  {\bf .0245} & 72000 \\\hline
		2v ITNE, $200$&  .04159 & .0241 &  .0228  &  72000 \\\hline\hline
	\end{tabular}
	\caption{Comparison of "1v", "3v" and "2v" models on the pipe system with timeouts of 1000s, 14440s and 72000s, where all neurons use 100 / 200 binary variables.}
	%L1 corresponds to $3.9$ or $4$, and results should be the sum of 10 pixels, so around 10 times higher values.}
	\label{table.pipe}
\end{table}

\newpage

\noindent {\bf Supplementary material content:} We provide in supplementary materials additional content, in particular results with reduced number of binary variables. We also provide the proof of Prop.~\ref{Prop2}, and explanations on the reduced-order dimension pipeline.

\fi

	
	
	

	
	
	
	
	
	 
	
	
	\newpage

	\hfill

	\newpage
	
	
	\bibliography{references}
	
	
\end{document}
