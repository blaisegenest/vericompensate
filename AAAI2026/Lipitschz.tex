\documentclass[letterpaper]{article} % DO NOT CHANGE THIS
\usepackage[submission]{aaai2026}  % DO NOT CHANGE THIS
\usepackage{times}  % DO NOT CHANGE THIS
\usepackage{helvet}  % DO NOT CHANGE THIS
\usepackage{courier}  % DO NOT CHANGE THIS
\usepackage[hyphens]{url}  % DO NOT CHANGE THIS
\usepackage{graphicx} % DO NOT CHANGE THIS
\urlstyle{rm} % DO NOT CHANGE THIS
\def\UrlFont{\rm}  % DO NOT CHANGE THIS
\usepackage{natbib}  % DO NOT CHANGE THIS AND DO NOT ADD ANY OPTIONS TO IT
\usepackage{caption} % DO NOT CHANGE THIS AND DO NOT ADD ANY OPTIONS TO IT
\frenchspacing  % DO NOT CHANGE THIS
\setlength{\pdfpagewidth}{8.5in} % DO NOT CHANGE THIS
\setlength{\pdfpageheight}{11in} % DO NOT CHANGE THIS
%
% These are recommended to typeset algorithms but not required. See the subsubsection on algorithms. Remove them if you don't have algorithms in your paper.
\usepackage{algorithm}
\usepackage{algorithmic}
\pagestyle{plain}
\usepackage{threeparttable}
\input{math_commands.tex}
\usepackage{lineno}
\usepackage{subcaption}
\usepackage{tabularx}
\usepackage{cases}
\captionsetup{compatibility=false}
\usepackage{epstopdf}
\usepackage{placeins}
\usepackage{pgfplots}
\usepackage{tikz}
\usepackage{calc}
\usepackage{array}
%\usepackage[linesnumbered,ruled,vlined]{algorithm2e}
\usetikzlibrary{positioning, arrows.meta,calc}
\usepackage{newfloat}
\usepackage{listings}
\DeclareCaptionStyle{ruled}{labelfont=normalfont,labelsep=colon,strut=off} % DO NOT CHANGE THIS
\lstset{%
	basicstyle={\footnotesize\ttfamily},% footnotesize acceptable for monospace
	numbers=left,numberstyle=\footnotesize,xleftmargin=2em,% show line numbers, remove this entire line if you don't want the numbers.
	aboveskip=0pt,belowskip=0pt,%
	showstringspaces=false,tabsize=2,breaklines=true}
\floatstyle{ruled}
\newfloat{listing}{tb}{lst}{}
\floatname{listing}{Listing}
%
% Keep the \pdfinfo as shown here. There's no need
% for you to add the /Title and /Author tags.
\pdfinfo{
	/TemplateVersion (2026.1)
}

\title{Order reduction and partial MILP models \\
	for certifying Robustness of DNNs globally.}
\date{}
\author{
	%Authors
	% All authors must be in the same font size and format.
	Written by AAAI Press Staff\textsuperscript{\rm 1}\thanks{With help from the AAAI Publications Committee.}\\
	AAAI Style Contributions by Pater Patel Schneider,
	Sunil Issar,\\
	J. Scott Penberthy,
	George Ferguson,
	Hans Guesgen,
	Francisco Cruz\equalcontrib,
	Marc Pujol-Gonzalez\equalcontrib
}
\affiliations{
	%Afiliations
	\textsuperscript{\rm 1}Association for the Advancement of Artificial Intelligence\\
	% If you have multiple authors and multiple affiliations
	% use superscripts in text and roman font to identify them.
	% For example,
	
	% Sunil Issar\textsuperscript{\rm 2},
	% J. Scott Penberthy\textsuperscript{\rm 3},
	% George Ferguson\textsuperscript{\rm 4},
	% Hans Guesgen\textsuperscript{\rm 5}
	% Note that the comma should be placed after the superscript
	
	1101 Pennsylvania Ave, NW Suite 300\\
	Washington, DC 20004 USA\\
	% email address must be in roman text type, not monospace or sans serif
	proceedings-questions@aaai.org
	%
	% See more examples next
}

%Example, Single Author, ->> remove \iffalse,\fi and place them surrounding AAAI title to use it
\iffalse
\title{My Publication Title --- Single Author}
\author {
	Author Name
}
\affiliations{
	Affiliation\\
	Affiliation Line 2\\
	name@example.com
}
\fi

\iffalse
%Example, Multiple Authors, ->> remove \iffalse,\fi and place them surrounding AAAI title to use it
\title{My Publication Title --- Multiple Authors}
\author {
	% Authors
	First Author Name\textsuperscript{\rm 1},
	Second Author Name\textsuperscript{\rm 2},
	Third Author Name\textsuperscript{\rm 1}
}
\affiliations {
	% Affiliations
	\textsuperscript{\rm 1}Affiliation 1\\
	\textsuperscript{\rm 2}Affiliation 2\\
	firstAuthor@affiliation1.com, secondAuthor@affilation2.com, thirdAuthor@affiliation1.com
}
\fi


\newtheorem{proposition}{Proposition}
\newtheorem{definition}{Definition}
\newcommand{\vW}{\boldsymbol{W}}
\newcommand{\val}{{\textrm{value}}}
\newcommand{\Val}{{\textrm{value}}}
\newcommand{\MILP}{{\textrm{MILP}}}
\newcommand{\LP}{{\textrm{LP}}}
\newcommand{\Improve}{\mathrm{Improve}}
\newcommand{\Utility}{\mathrm{SAS}}
\newcommand{\Sol}{\mathrm{Sol}}
\newcommand{\sol}{\mathrm{sol}}
\newcommand{\UB}{\mathrm{UB}}
\newcommand{\LB}{\mathrm{LB}}
\newcommand{\ub}{\mathrm{ub}}
\newcommand{\lb}{\mathrm{lb}}
\newcommand{\B}{\mathrm{B}}
\usepackage{amsmath, amssymb, amsfonts}
\newcommand{\ReLU}{\mathrm{ReLU}}
\newcommand{\CMP}{{\textrm{CMP}}\ }
\newcommand{\fix}{\marginpar{FIX}}
\newcommand{\new}{\marginpar{NEW}}
\newcommand{\toolname}{Hybrid MILP}

% REMOVE THIS: bibentry
% This is only needed to show inline citations in the guidelines document. You should not need it and can safely delete it.
\usepackage{bibentry}
% END REMOVE bibentry

\begin{document}
	
	\maketitle
	
	\begin{abstract}
		Deep neural networks (DNNs) are often brittle to small perturbations, which has led to extensive research to verify their robustness. 
Most existing methods focus on {\em local} robustness, i.e., verification in the neighborhood of fixed specific inputs. Local robustness 
%does {\em not} provide guarantees on whether new specific incoming inputs are robust, e.g., real-time images in a video stream. Moreover, most verification 
techniques are impractical to guarantee robustness of {\em real-time} inputs
on embedded systems, due to excessive latency and computational intensivity.

In this paper, we consider {\em global} robustness, which is significantly more complex than local robustness, as the number of variables doubles (from the deviation image to the image and its deviation). 
%Further, the values each neuron can take are no longer limited to a small neighborhood. 
%, that is, guarantees not restricted to a set of local images. 
We focus on deriving {\em bounds} on the variation of output values across different decision classes of a DNN, given $L_\infty$- {\em and} $L_1$-perturbations. We develop novel {\em Diff} MILP models encoding the evolution of {\em diff}erential variables between the image and its perturbation, more efficient than the classical MILP encoding of variables independently. To obtain better bounds, we reduce the space dimension by using principal component analysis (PCA), focusing on {\em realistic} inputs. These bounds enable the {\em real-time} certification of robustness for $\geq 67\%$ 
of incoming images for an $L_1$-perturbation of $1,4,2$
over the MNIST, Fashion MNIST and CIFAR-10 datasets respectively,
adding only $0.5ms$ of latency using only a single CPU core.
		\iffalse
		Most DNNs are brittle to small perturbations. Extensive works have thus been performed to verify robustness for DNNs.
		However, these works mostly consider local robustness, i.e. in the neighborhood of an image.
		While local robustness is useful to have an idea how often non robust images happen, by repeating the verification on 1000 or 10000 pre-obtained images, the main shortcoming is that we have no guarantee that a specific new incoming image, e.g. in a video feed, is robust: The verification process takes too long and requires too much resources to be performed online on embedded systems.
		
		In this paper, we consider {\em global} robustness, that is, guarantees not restricted to a set of local images. For that, we consider {\em bounds} on the switch of values between the different decision classes of a DNN due to a given perturbation. 
		The verification question is much harder than local robustness, as the number of complex variables doubles (from the deviation image to the image and its deviation).
		Further, the values each neuron can take is no more in a small neighborhood.
		Therefore, the global verification process is very complex.
		To obtain useable bounds, we develop several novel partial MILP models for global robustness, with different trade-offs. Last, we use order reduction techniques to reduce the space of images considered, avoiding unrealistic inputs, by using linear PCA. 
		This results into usable bounds, allowing in real time to certify robustness for $87\%$ of incoming images in the MNIST benchmark for a L1-perturbation of $0.5$, as well as for a surrogate computing the hidden plastic strain associated to a deformation map of a pipe.
		\fi
	\end{abstract}
	
	
	\section{Introduction}
	
	Deep neural networks (DNNs for short) have demonstrated remarkable capabilities, achieving human-like or even superior performance across a wide range of tasks. However, their robustness is often compromised by their susceptibility to input perturbations \cite{szegedy}. This vulnerability has catalyzed the verification community to develop various methodologies, each presenting a unique balance between completeness and computational efficiency \cite{Marabou,Reluplex,deeppoly}. This surge in innovation has also led to the inception of competitions such as VNNComp \cite{VNNcomp}, which aim to systematically evaluate the performance of neural network verification tools. Among them, NNenum \cite{nnenum}, Marabou \cite{Marabou,Marabou2}, and PyRAT \cite{pyrat} MnBAB \cite{ferrari2022complete}, built upon ERAN \cite{deeppoly} and PRIMA \cite{prima}; and $\alpha,\beta$-CROWN \cite{crown,xu2020fast}, 
%the winner of the last 4 VNNcomp, benefiting from 
based on branch-and-bound based methodology \cite{cutting,BaB}.

These tools %benchmarks usually 
focus on {\em local} robustness, i.e. given a DNN, an image and a small neighborhood around this image, is it the case that all the images in the neighborhood are classified in the same way by the DNN? The neighborhood is provided by a maximal perturbation of the input image, often an 
$L_\infty$-perturbation, i.e. every subpixel of the input image can vary in a very small range, typically $\frac{2}{255}$ (that is 2 levels of grey/blue/red/green). 
Although it is not necessarily the most meaningful perturbation,
$L_\infty$ is the usual choice because it is perfectly linear and specifies 
subpixel perturbations independantly, which is easier to verify. 
Importantly, these verification tools for local robustness are too computationally-intensive to be used in a real-time decision making pipeline: considering an autonomous car with a video feed from the dashboard, 
images of the feed cannot be certified robust in few ms on embedded hardware to e.g. skip non-robust images and only consider certified robust images.
% for the decision-making process.

\smallskip

In this paper, we consider {\em global} robustness, that is we do not restrict the certification process to the neighborhood of a fix input. We follow a two steps procedure. 
The first step, performed offline once, computes global bounds on the shift between 
output values of different decision classes due to the perturbation, close to \cite{vhagar}. That is, considering decision classes $C$ and $D$ and perturbation $\varepsilon$, compute an 
upper bound $\bar{\beta}^\varepsilon_{C,D}$ of $\max_{I,I', |I-I'| \leq \epsilon}(value_{I}(C) - value_{I}(D) + value_{I'}(D) - value_{I'}(C))$, where $value_{J}(X)$ is the output value of class $X \in \{C,D\}$ for input image $J \in \{I,I'\}$. %Notice that as $I,I',C,D$ have symetrical roles, we can choose $\beta^\varepsilon_{D,C} = \beta^\varepsilon_{C,D}$. Also, as $I,I'$ have symetrcal role, the minimum value is exactly $-$ the maximum value. 
%Hence, if we have $n$ decision classes, we only have to compute $n (n-1)/2$ bounds. 
Bound $\bar{\beta}^\varepsilon_{C,D}$ is computed offline once for a DNN, 
and it is valid over the whole input space; compared with $k$ calls to 
{\em local} robustness, once for each of the $k$ input images, with results only valid for these $k$ images.

The second step is real-time, being performed with the DNN inference of the image $I$ considered: it suffices to consider the class $C$ with the highest output value $value_{I}(C)$, and check whether for every other class $D \neq C$, 
$value_{I}(C) - value_{I}(D) > \bar{\beta}^\varepsilon_{C,D}$. 
If this is the case, then we are certified that image $I$ is robust for perturbation $\varepsilon$, because $\varepsilon$-perturbed image $I'$ could at most get  $value_{I'}(D) \leq \bar{\beta}^\varepsilon_{C,D}  - (value_{I}(C) - value_{I}(D))  + value_{I'}(C) < value_{I'}(C)$, hence $C$ is also the predicted class for image $I'$. This typically needs a couple of tens CPU instructions, which can be performed under 1ms (mili-second) on a single CPU core. If the image is not certified robust, one could either skip image $I$ (in a video feed), or use safer degraded mode in the decision making process till a trustable robust image is received.


Our main contributions address the challenges to compute the {\em global bounds} $\bar{\beta}^\varepsilon_{C,D}$, for $C,D$ output neurons:
% for standard ReLU DNNs (e.g. \cite{vhagar}). Our findings can be extended to other activation functions, following similar extention by \cite{DivideAndSlide}:
%, with updated MILP models e.g. for maxpool:
\begin{enumerate}
	%\item  Our first contribution studies the {\em LP relaxation} of the exact MILP encoding of ReLUs. {\color{blue} We establish in Proposition \ref{LP} its equivalence with the so-called "triangular abstraction"}.
	
	\item We develop a novel {\em Diff MILP encoding} for the global robustness problem, %called the {\em "2v" model}, 
	where the variables are the values of the perturbed neurons, as well as the difference between the original and the perturbed neuron values (called the {\em diff variables}, introduced in \cite{diff}). We study and encode how the {\em diff variables} evolve after passing through a ReLU (Prop.~\ref{Prop2}), see Section \ref{s.diff}. 
	Compared with the {\em classical MILP model} \cite{MILP} employed in 
	\cite{vhagar,lipshitz,ITNE}, which considers the input and the perturbation but {\em diff variable}, we keep the same number of 2 binary variables per ReLU.
	%the results are more accurate for the same runtime.
	%the linear relaxation is much more accurate, as each {\em diff variable} can be bounded after a ReLU as a function of the value of the {\em diff variables} before the ReLU, whereas the linear relaxations of the classical model is extremely inaccurate, as the variables are independent of each other. This was observed in \cite{lipshitz,ITNE}, and constraints encoding
	%a part of the linear relaxation of our "2v" model were added explicitly. 
	%Experimentally, 
	However, our {\em Diff MILP model} is more efficient, one reason being 
	the accuracy of its linear relaxation.
	%The number of variables a priori doubles compared with local robustness, from each neuron value in the perturbed image to each neuron value in the perturbed image {\em and} in the original image, as the original image is no more fixed. 
	%Recall that the worst case complexity of MILP is exponential in the number of binary variables \cite{DivideAndSlide}. 
	%A straightforward MILP model would be to use the  for each of these variables, as in \cite{vhagar,lipshitz}. The main issue with the classical model is that its linear relaxations is extremely inaccurate, as the variables are independent of each other. Instead, we develop another exact MILP model, 

	\item Further, from the {\em Diff MILP model}, which is exact, 
	we develop two abstract MILP models, which are more efficient but also asymptotically less accurate than {\em Diff}. 
	Namely, the "2b+1" model, accurate on the {\em diff variables} but abstract on the perturbed variables; while the "1b" model has a unique binary variable per ReLU, only considering the {\em diff variables}.
	%\item We adapt the Solution Aware Scoring from \cite{ATVA25} to our novel MILP models, in order to select the most important ReLUs to be treated using complex binary variables, while less important variables are treated using linear relaxation. The chosen number of binary variables depends upon the complexity of the DNN as well as the targeted runtime.

   \item  In terms of perturbations, we consider conjunctions of $L_\infty$- and $L_1$-norms, which allow to accurately describe perturbations. For instance, "each subpixel is perturbed by at most $\frac{50}{255}$ ($L_\infty$) and the sum of the absolute value of perturbations over all subpixels is at most $1$" ($L_1$-perturbation). While $L_1$-perturbations are not linear (because of the absolute values), reason for which it is seldom used, we show in Section~\ref{s.L1} how to use it as a perturbation in the MILP model without incurring any expansive binary variables (only cheap linear variables are necessary). 



\item Bounds obtained on the full input space are particularly pessimistic, as all inputs, including {\em Out of Distribution (OOD \cite{OOD})} inputs far away from the training dataset, need to be accounted for. 
%As a result, the runtime to obtain the bound is particularly long. 
	%Finally, when computing worst-case 
	%pairs (image, perturbation), improbable images are generated, hence these  are not meaningful. 
To address this, we consider model order reduction techniques from engineering science \cite{Paco}. Specifically, we use Principal Component Analysis (PCA) to represent faithfully common inputs from the dataset. OOD	inputs may be represented unfaithfully, which is reasonable as the DNN is unlikely to provide reasonable answer on such inputs anyway.
	%	focus on reduce the space to a linear input space. We choose the number of dimensions of the space to equal the accuracy of the DNN on the reduced space. 
	%(using a projection to the reduced space then the inverse projection to obtain a very similar image understandable by the DNN). 
For instance, using just $20$ out of 784 dimensions suffices to represent faithfully MNIST inputs, without losing accuracy for the DNN on the MNIST dataset.
	%On the MNIST benchmark, this means 20 linear dimensions to match the $97\%$ accuracy of the DNN we considered, instead of the 784 dimensions of the full image space. 
	%Using PCA, which is linear, makes it easy to specify perturbations on the actual image (where it is meaningful) rather than on the reduced space.

\item Experimentally, the {\em Diff MILP model} computes upper bounds
$\bar{\beta}^{\varepsilon}_{C,D}$ which reduces the gap to the lower bound compared with optimized version of the classical MILP encoding; namely reducing the number of binary variables (Vhagar \cite{vhagar}); or adding linear constraints from the {\em diff variables} (ITNE \cite{ITNE}), see Table \ref{table.classical}.
The abstractions  "2b+1" and "1b" variant offer different trade-offs, 
reaching better bounds than the {\em Diff MILP} model when the instance is very complex, or when runtime is limited (Table \ref{table.L1}).  
Further, using PCA reduces the upper bound $\beta_{C,D}$ by $3$ to $20$ times.
Overall, using these different techniques ({\em Diff} MILP model, abstraction and PCA) enables the real-time certification of $> 67\%$ of fresh images for an $L1$-perturbation of $1,4,2$ over the MNIST, Fashion MNIST and CIFAR-10 datasets respectively, see Table \ref{table.cert}. The online process adds only 0.5ms of latency per image, and 2000 images/second can be treated per CPU core.
\end{enumerate}


%\newpage

%   
% 
%
%In this context, application of DNNs in safety critical applications is cautiously envisioned. For that to happen at a large scale, hard guarantees should be provided \cite{certification}, through e.g. incremental verification \cite{incremental}, so that to avoid dramatic consequences. It is the reason for the development of (hard) verification tools since 2016, with now many tools with different trade-offs from exact computation but slow (e.g. Marabou \cite{katz2019marabou}/Reluplex\cite{Reluplex}), up to very efficient but also incomplete (e.g. ERAN-DeepPoly \cite{deeppoly}). To benchmark these tools, a competition has been run since 2019, namely VNNcomp \cite{VNNcomp}. The current overall better performing verifier is $\alpha$-$\beta$-CROWN \cite{crown}, a fairly sophisticatedly engineered tool based mainly on "branch and bound" (BaB) \cite{BaB}, and which can scale all the way from complete on smaller DNNs \cite{xu2020fast} up to very efficient on larger DNNs, constantly upgraded, e.g. \cite{cutting}. 
%
%While the verification engines are generic, the benchmarks usually focus on local robustness, i.e. given a DNN, an image and a small neighbourhood around this image, 
%is it the case that all the images in the neighbourhood are classified in the same way.
%While some quite large DNNs (e.g. ResNet with tens of thousands of neurons) can be verified very efficiently (tens of seconds per input) \cite{crown}, with all inputs either certified robust or an attack on robustness is found; some smaller DNNs (with hundreds of neurons, only using the simpler ReLU activation function) cannot be analysed fully, with $12-20\%$ of inputs where neither of the decisions can be reached (\cite{crown} and Table \ref{tab:example}). Actually, DNNs which are trained to be robust (using DiffAI \cite{DiffAI} or PGD \cite{PGD}) are easier to verify, while the DNNs trained in a "natural" way are harder to verify.
%
%
%In this paper, we focus on DNNs trained in a "natural" way,
%%uncovering what makes the DNNs trained in a natural way so hard to verify (
%because for "easier" DNNs, adequate methods already exist. 
%To do so, we analyse the abstraction mechanisms at the heart of several efficient algorithms, namely Eran-DeepPoly \cite{deeppoly}, the Linear Programming approximation \cite{MILP}, PRIMA \cite{prima}, and different versions of ($\alpha$)($\beta$)-CROWN \cite{crown}. All these algorithms compute lower or/and upper bounds for the values of neurons (abstraction on values) for inputs in the considered input region, and conclude based on such bounds. For instance, if for all image $I'$ in the neighbourhood of image $I$, we have $weight_{I'}(n'-n) < 0$ for $n$ the output neuron corresponding to the expected class, then we know that the DNN is robust in the neighbourhood of image $I$. We restrict the formal study to DNNs using only the standard ReLU activation function, although nothing specific prevents the results to be extended to more general architectures. We uncover that {\em compensations} 
%(see next paragraph) is the phenomenon creating inaccuracies. We verified experimentally that a DNN trained in a natural way has heavier compensating pairs than DNNs trained in a robust way.
%
%Formally, a compensating pair is a pair of paths $(\pi,\pi')$ between a pair of neurons $(a,b)$, such that we have $w < 0 < w'$, for $w,w'$ the products of weight seen along $\pi$ and $\pi'$. Ignoring the (ReLU) activation functions, the weight of $b$ is loaded with $w \cdot weight(a)$ by $\pi$, while it is loaded with $w' \cdot weight(a)$ by $\pi'$. That is, it is loaded by $(w+w') weight(a)$. As $w,w'$ have opposite sign, they will compensate (partly) each other. The compensation is only partial due to the ReLU activation seen along the way of $\pi$ which can "clip" a part of $w \cdot weight(a)$, and similarly for $\pi'$. However, it is very hard to evaluate by how much without explicitly considering both phases of the ReLUs, which all the efficient tools try to avoid because it is very expansive (could be exponential in the number of such ReLU nodes opened).

%Our first main contribution is to formally show, in Theorem \ref{th1}, that compensation is the sole reason for the inaccuracies as (most) efficient algorithms will compute exact bounds for all neurons if there is no compensating pair of paths at all.
%While this theorem is theoretically interesting, it is not usable in practice as (almost) all networks have some compensating pairs. However, this notion of compensating pairs opens a first interesting idea concerning an exact abstraction of the network using a Mixed Integer Linear Program \cite{MILP}, where the weight of each neuron is a linear variable, and ReLU node may be associated with binary variables (exact encoding) or linear variables (overapproximation). While LP tools can scale to thousands of linear variables, MILP encoding can only be solved for a limited number of binary variables. This suggests that a simpler encoding could be used for those ReLUs that are not on compensating pairs, as their precise outcome may not be necessary.

%Our second main contribution is to show formally in Theorem \ref{th2}, that 
%encoding all ReLU nodes on a pair of compensating paths with a binary variable,
%and using linear relaxation for the other ReLU nodes, will lead to exact bounds for (most) of the algorithms considered. This theorem allows to restrict the number of integer variables, and thus to obtain encodings that are faster to solve. Practically, however, (almost) all ReLU nodes are on some compensating path, and using this exact restricted MILP encoding will be too time consuming.

%Our third main contribution is more practical, proposing Algorithm \ref{algo1} based on this knowledge that compensating pair of paths are the reason for inaccuracy. The idea is thus to use this information to rank the ReLU nodes in terms of importance, and only keep the most important ones as binary variables, and use linear relaxation for the least important ones.
%%More precisely, the algorithm will, as DeepPoly, consider layers one by one and neurons $b$ %on this layer one by one, selecting the heaviest pairs of compensating paths ending in $b$
%%and associating these nodes with a binary variable. Then an MILP tool such as Gurobi is used %to compute the lower and upper bound for node $b$. 
%Overall, the worst case complexity of algorithm \ref{algo1} is lower than $O(N 2^K LP(N))$, where $N$ is the number of nodes of the DNN, $K$ the number of ReLU nodes selected as binary variable, and $LP(N)$ is the (polynomial time) complexity of solving a linear program representing a DNN with $N$ nodes. This complexity is an upper bound, as e.g. Gurobi is fairly efficient and never need to consider all of the $2^K$ ReLU configurations to compute the bounds. Keeping $K$ reasonably low thus provides an efficient algorithm. 
%By design, it will never run into a complexity wall (unlike the full MILP encoding), although it can take a while on large networks because of the linear factor $N$ in the number of nodes. An additional interesting point is that it is extremely easy to parallelize, as all the nodes in the same layer can be run in parallel. We verify experimentally that the algorithm offers interesting trade-offs, by testing on local robustness for DNNs trained "naturally" (and thus difficult to verify).


%KSM: I suggest we move this to experimental evaluation
%This paper does not focus on producing the most efficient tool, and we did not spend engineering efforts to optimize it. The focus is instead on the novel notion of compensation, the associated methodology and its evaluation. For instance, our implementation is fully in Python, with uncompetitive runtime for our DeepPoly implementation ($\approx 100$ slower than in CROWN). Still, evaluation of the methodology versus even the most efficient tools reveals a lot of potential for the notion of compensation, opening up several opportunities for applying it in different contexts of DNN verification (see Section \ref{Discussion}). 


	
	\section{Notations and Preliminaries}
	
	In this paper, we will use lower case latin $a$ for scalars, bold $\boldsymbol{z}$ for vectors, 
	capitalized bold $\boldsymbol{W}$ for matrices, similar to notations in \cite{crown}.
	To simplify the notations, we restrict the presentation to feed-forward, 
	fully connected ReLU Deep Neural Networks (DNN for short), where the activation function is $\ReLU : \mathbb{R} \rightarrow \mathbb{R}$ with
	$\ReLU(x)=x$ for $x \geq 0$ and $\ReLU(x)=0$ for $x \leq 0$, which we extend componentwise on vectors.
	
	%In this paper, we will not use tensors with a dimension higher than matrices: those will be flattened.
	
	%\subsection{Neural Network and Verification}
	
	
	% testtesttesttest
	An $\ell$-layer DNN is provided by $\ell$ weight matrices 
	$\boldsymbol{W}^i \in \mathbb{R}^{d_i\times d_{i-1}}$
	and $\ell$ bias vectors $\vb^i \in \mathbb{R}^{d_i}$, for $i=1, \ldots, \ell$.
	We call $d_i$ the number of neurons of hidden layer 
	$i \in \{1, \ldots, \ell-1\}$,
	$d_0$ the input dimension, and $d_\ell$ the output dimension.
	
	Given an input vector $\boldsymbol{z}^0 \in \mathbb{R}^{d_0}$, 
	denoting $\hat{\boldsymbol{z}}^{0}={\boldsymbol{z}}^0$, we define inductively the value vectors $\boldsymbol{z}^i,\hat{\vz}^i$ at layer $1 \leq i \leq \ell$ with
	\begin{align*}
		\boldsymbol{z}^{i} = \boldsymbol{W}^i\cdot \hat{\boldsymbol{z}}^{i-1}+ \vb^i \qquad \, \qquad
		\hat{\boldsymbol{z}}^{i} = \ReLU({\boldsymbol{z}}^i).
	\end{align*} 
	
	The vector $\hat{\boldsymbol{z}}$ is called post-activation values, 
	$\boldsymbol{z}$ is called pre-activation values, 
	and $\boldsymbol{z}^{i}_j$ is used to call the $j$-th neuron in the $i$-th layer. 
	For $\boldsymbol{x}=\vz^0$ the (vector of) input, we denote by $f(\boldsymbol{x})=\vz^\ell$ the output. Finally, pre- and post-activation neurons are called \emph{nodes}.
	% and when we refer to a specific node/neuron, we use $a,b,c,d,n$ to denote them, and $W_{a,b} \in \mathbb{R}$ to denote the weight from neuron $a$ to $b$. Similarly, for input $\boldsymbol{x}$, we denote by $\val_{\boldsymbol{x}}(a)$ the value of neuron $a$ when the input is $\boldsymbol{x}$.	For convenience, we write $n < z$ if neuron $n$ is on a layer before $\ell_z$, and $n \leq z$ if $n< z$ or $n=z$.
	
    In this paper, we consider the {\em global} verification problem, where we optimize over all image $I$ and all perturbation $I'$ of $I$ with $|I-I'| \leq \varepsilon$. 	We will consider three kinds of variables: 
    \begin{itemize}
    \item $x_j,\hat{x}_j$, for nodes $j$ with input image $I$, 
    \item $x'_j,\hat{x}'_j$, for nodes $j$ with input the perturbed $I'$, 
    \item  $y_j = x_j - x'_j$ the {\em diff variable}, with 
    $\hat{y}_j = \hat{x}_j - \hat{x}'_j$ (and {\em not} 
    $\hat{y}_i = \ReLU(x_j-x'_j)$), similarly than in the 
	Interleaving Twin-Network Encoding (ITNE) model \cite{lipshitz}. 
    \end{itemize}
     
    %Concerning the verification problem, we focus on the global-robustness question. Global robustness asks to determine how the output of a neural network will be affected under a certain kind of small perturbations to any possible input. In this view, Lipschitz continuity is a good characterization of global robustness.
	
	
	
	\iffalse
	
	\section{Global robustness and Lipschitz constant}
	
	
	Recall the definition of Lipschitz continuity:
	under distance $d$, a function $f(x)$ is Lipschitz continuous with respect to constant $K$ if:
	\begin{align*}
		\forall \boldsymbol{x} \forall\boldsymbol{y} (|f(\boldsymbol{x}) -f(\boldsymbol{y}) |\leq K|\boldsymbol{x}-\boldsymbol{y}|)
	\end{align*} 
	In our practice, when we need global robustness, we will compute an optimization question respect to a certain number $\varepsilon$:	\begin{align}\label{global_robustness}
		\max_{|\boldsymbol{x}-\boldsymbol{y}| \leq \varepsilon} |f(\boldsymbol{x}) -f(\boldsymbol{y}) |
	\end{align} And this will lead to the following definition
	
	\begin{definition}[$\varepsilon$-diff bound]
		Suppose we have a function $f$ from $\mathbb{R}^n$ to $\mathbb{R}^m$ and $||$ is $L_\infty$ norm. 
		
		For a number $\varepsilon\in\mathbb{R}$, an $\varepsilon$-diff bound $D_\varepsilon$ is a number such that for any inputs $x,y$: \begin{align*}
			|x-y|\leq \varepsilon \implies |f(x)-f(y)| \leq D_\varepsilon \cdot \varepsilon
		\end{align*}
		
	\end{definition}
	
	From $\varepsilon$-diff bound, we cannot directly obtain a Lipschitz bound for the function, but we can get the following weaker bound:
	
	\begin{definition}[Lipschitz above $\varepsilon$ constant]
		Suppose we have a function $f$ from $\mathbb{R}^n$ to $\mathbb{R}^m$ and $||$ is $L_\infty$ norm. 
		
		For a number $\varepsilon\in\mathbb{R}$, a Lipschitz above $\varepsilon$ constant  $K_\varepsilon$,  is a number such that for any inputs $x,y$: \begin{align*}
			|x-y|\geq \varepsilon &\implies |f(x)-f(y)| \leq K_\varepsilon \cdot |x-y|\\
			|x-y|<\varepsilon &\implies |f(x)-f(y)| \leq K_\varepsilon \cdot \varepsilon\\
		\end{align*}		
	\end{definition}
	
	
	\begin{proposition}
		
		Suppose $D$ is an $\varepsilon$-diff bound for $f(x)$. Then for any $N\in\mathbb{Z}^+$, $D\frac{N+1}{N}$ is a Lipschitz about $N\varepsilon$ constant.
		
		That is, for any two inputs $x,y$, if \begin{align*}
			|x-y|\leq \varepsilon \implies |f(x)-f(y)| \leq D \cdot \varepsilon,
		\end{align*} then 	 \begin{align*}
			|x-y|\geq N\varepsilon &\implies |f(x)-f(y)| \leq D\frac{N+1}{N} \cdot |x-y|\\
			|x-y|<N\varepsilon &\implies |f(x)-f(y)| \leq D\frac{N+1}{N} \cdot N\varepsilon\\
		\end{align*}
	\end{proposition}
	
	\textbf{Proof.} We fix the number $N\in\mathbb{Z}^+$ and assume that we have two inputs $x, y$.
	
	The first case, $|x-y|\geq N\varepsilon$. Then we assume $|x-y| \in [M\varepsilon ,  (M+1)\varepsilon]$ for another integer $M\geq N$. Then we can divide the line segment between $x, y$ into $M+1$ pieces: $x_0 = x, x_1, x_2, \cdots, x_{M+1} = y$ such that $|x_i-x_{i+1}| \leq \varepsilon$ and apply the definition of $\varepsilon$-diff bound $D$ for each pieces:\begin{align*}
		|f(x)-f(y)| &= |f(x_{M+1})-f(x_M)+\cdots+f(x_1)-f(x_0)|\\
		&\leq |f(x_{M+1})-f(x_M)|+\cdots+|f(x_1)-f(x_0)|\\
		&\leq D\varepsilon + \cdots +D\varepsilon = (M+1)D\varepsilon
	\end{align*}
	Hence,\begin{align*}
		|f(x)-f(y)| &\leq (M+1)D\varepsilon \leq D\cdot (M+1)\varepsilon \frac{|x-y|}{M\varepsilon}\\
		&= D\cdot\frac{M+1}{M} |x-y|	\leq   D\cdot\frac{N+1}{N} |x-y|		
	\end{align*}
	The second case, $|x-y|< N\varepsilon$. Similarly we can divide the line segment between $x, y$ into $N$ pieces and then $|f(x)-f(y)|\leq D N\varepsilon\leq |f(x)-f(y)|\leq D \frac{N+1}{N} N\varepsilon$.
	
	This ends the proof.
	\hfill $\square$
	
	In practice, Lipschitz above $\varepsilon$ constant is already sufficient, since in most cases we care about the absolute difference under input perturbations, not the ratio. By combining the above proposition, the computation of $\varepsilon$-diff bound can satisfy our aim.
	
	
	Moreover, we have one more proposition connecting $\varepsilon$-diff bound and Lipschitz above $\varepsilon$ constant.
	
	
	\begin{proposition}
		For any $N\in\mathbb{Z}^+$, suppose for any $a$ in $\{\frac{N}{N}\varepsilon,\frac{N+1}{N}\varepsilon,\cdots, \frac{2N-1}{N}\varepsilon\}$, $D$ is an $a$-diff bound for $f(x)$. 
		
		Then $D\frac{N+1}{N}$ is a Lipschitz about $\varepsilon$ constant:\begin{align*}
			|x-y|\geq \varepsilon &\implies |f(x)-f(y)| \leq D\frac{N+1}{N} \cdot |x-y|\\
			|x-y|<\varepsilon &\implies |f(x)-f(y)| \leq D\frac{N+1}{N} \cdot \varepsilon\\
		\end{align*}
	\end{proposition}
	
	\textbf{Proof.}
	We fix the number $N\in\mathbb{Z}^+$ and assume we have two inputs $x, y$.
	
	For the case that $|x-y|<\varepsilon$, this is trivial by definition.
	
	For the case that $|x-y|\geq \varepsilon$, there exists a sum $x_1+x_2+\cdots+x_n$ by numbers from (allowing repetitions) $\{\frac{N}{N}\varepsilon,\frac{N+1}{N}\varepsilon,\cdots, \frac{2N-1}{N}\varepsilon\}$ such that \begin{align*}
		\varepsilon \leq x_1+x_2+\cdots+x_n -\frac{1}{N}\varepsilon \leq |x-y| \leq x_1+x_2+\cdots+x_n
	\end{align*}
	By assumption, divide the line segment from $x$ to $y$ into pieces according to $x_1, x_2,\cdots,x_n$, then we will have $$|f(x)-f(y)|\leq Dx_1+Dx_2+\cdots+Dx_n.$$
	
	Hence,\begin{align*}
		\dfrac{|f(x)-f(y)|}{|x-y|} &\leq \dfrac{Dx_1+Dx_2+\cdots+Dx_n}{x_1+x_2+\cdots+x_n -\frac{1}{N}\varepsilon}\\
		& \leq D\cdot( 1+  \dfrac{\frac{1}{N}\varepsilon}{\varepsilon})= D \frac{N+1}{N}\\
	\end{align*}
	This ends the proof.
	\hfill $\square$
	
	\fi

	
	%\section{MILP for local robustness}
	
	
	
	\subsection{MILP encoding for local ReLU}
	
	Mixed Integer Linear Programming (MILP) can encode faithfully ReLU DNNs:
	For an unstable neuron $n$, that is with values 
    $x \in [\LB(n),\UB(n)]$ with $\LB(n)<0<\UB(n)$, 
    the value $\hat{x}$ of $\ReLU(x)$ can be encoded exactly in an MILP formula with one binary / integer variable $a$ valued in $\{0,1\}$, using constants $\UB(n),\LB(n)$ with 4 constraints \cite{MILP}:
	
	\vspace{-0.4cm}
	\begin{equation} 
        \hat{x} \geq x \, \wedge \, \hat{x} \geq 0 \, \wedge \, \hat{x} \leq \UB(n) a \, \wedge \, \hat{x} \leq x-\LB(n) (1-a)
		\label{eq11}
	\end{equation}
	
\begin{proposition}
\cite{MILP}
\label{Prop1}
A solution $x,\hat{x},a$  of the above MILP program satisfies $\hat{x} = \ReLU(x)$,
and $a=1$ if $x> 0$ and $a=0$ if $x< 0$ (both are possible if $x=0$).
\end{proposition}

	%For all $x \in [\LB(n),\UB(n)] \setminus 0$, there exists a unique solution $(a,\hat{x})$ that meets these constraints, with $\hat{x}=\ReLU(x)$ \cite{MILP}. The value of $a$ is 0 if $x < 0$, and 1 if $x>0$, and can be either if $x=0$. This encoding approach can be applied to every (unstable) ReLU node, and optimizing its value can help getting more accurate bounds. However, for networks with hundreds of {\em unstable} nodes, the resulting MILP formulation will contain numerous integer variables and generally bounds obtained will not be accurate, even using powerful commercial solvers such as Gurobi.
	
    \iffalse
	The global structure is as follows, using Gurobi as an example:
	\begin{enumerate}
		\item For each input node, each output node, and each pre-activation and post-activation node in the hidden layers,  set one variable. 
		\item Set constraints for input nodes.
		\item For each pre-activation node in a hidden layer (and each output node), set linear constraints relating them to the post-activation or input nodes in the previous layer they connect to.
		\item Between pre- and post- activation nodes, set the MILP constraint described above.
	\end{enumerate} 
    
    \fi
    
    In the whole MILP model, each unstable ReLU is encoded in the above way with one integer variable. The encoding from $(\hat{x}_j)_{j \text{ in layer } i}$ to 
	$(x_{j'})_{j' \text{ in layer } i+1}$ variables is simply the linear combination 
	$\boldsymbol{z}^{i} = \boldsymbol{W}^i\cdot \hat{\boldsymbol{z}}^{i-1}+ \vb^i$.
	%This exact MILP encoding is often too computationally intensive, as the worst-case complexity of MILP is exponential in the number of integer variables \cite{DivideAndSlide}.
    
    \subsection{LP relaxation}

	MILP instances can be linearly relaxed into LP over-abstraction, where variables $a$ originally restricted to integers in $\{0,1\}$ (binary) are relaxed to real numbers in the interval $[0,1]$, while maintaining the same encoding. As solving LP instances is polynomial time, this optimization is significantly more efficient. However, this efficiency comes at the cost of precision, often resulting in less stringent bounds. This approach is termed the {\em LP relaxation}.
    % We invoke a folklore result on the LP relaxation of (\ref{eq11}), for which we provide a direct and explicit proof.
	
	
	%\subsection{partial MILP}
	
	Intermediate between these 2 extreme cases, there is {\em partial MILP} 
    (pMILP for short) to get trade-offs between accuracy and runtime
	\cite{DivideAndSlide}:
	Let $X$ be a subset of the set of unstable neurons, and $n$ a neuron for which we want to compute upper and lower bounds on values: the pMILP based on $X$ to compute neuron $n$ uses the MILP encpoding (\ref{eq11}), where variable $a$ is:
	\begin{itemize}
		\item binary for neurons in $X$ (exact encoding of the ReLU),
		\item linear for neurons not in $X$ (linear relaxation).
	\end{itemize}
	
	%We will denote the above model by MILP$_X$. We say that a node is {\em opened} if it is in $X$. 
	
	%To reduce the runtime, we will limit the size of subset $X$. This a priori hurts accuracy. To recover some of this accuracy, we use an iterative approach: computing lower and upper bounds $\LB,\UB$ for neurons $n$ of a each layer iteratively, that are used when computing values of the next layer.
	
	\iffalse
	\subsection{SAS}
	
	
	In pMILP, to decide the set $X$, we introduce the method {\em Solution-Aware Scoring} (SAS)
	to evaluate accurately how opening a ReLU impacts the accuracy. Again, here we use the definition from paper CITE. For details and explanation, see CITE.
	
	
	Assume that we want to compute an upper bound for neuron $z$ on layer $\ell_z$. For each node $n<z$, we denote ($\Sol\_\max_X^z(n))_{n \leq z}$ a solution of $\mathcal{M}_X$ maximizing $z$: $\Sol\_\max_X^z(z)$ is the maximum of $z$ under $\mathcal{M}_X$; and we denote $(\sol(n))_{n \leq z} = (\Sol\_\max_\emptyset^z(n))_{n \leq z}$ a solution maximizing the value for $z$ when all ReLU use the LP relaxation. Moreover,  we define the function
	$\Improve\_\max^z(n)=$ $\sol(z) - \Sol\_\max_{\{n\}}^z(z)$, 
	accurately represents how much opening neuron $n < z$ reduces the maximum computed for $z$
	compared with using only LP. 
	
	First, SAS will call solvers to compute the LP model to get a solution, which is reasonably fast as there is no binary variables. 
	
	Next, for a neuron $b$ on the layer before layer $\ell_z$, we define:
	
	
	\vspace{-0.4cm}
	$$\Utility\_\max\nolimits^z(b) = W_{bz} \times (\sol(\hat{b})- \ReLU(\sol(b)))$$
	\vspace{-0.4cm}
	
	
	And for a neuron $a$ two layers before $\ell_z$, 
	$b$ denoting neurons in the layer $\ell$ just before $\ell_z$.
	Recall the rate $r(b)=\frac{\max(0,\UB(b))}{\max(0,\UB(b))-\min(0,\LB(b))} \in [0,1]$.
	We define:
	
	
	\begin{flalign*}
		\Delta(\hat{a}) &= \ReLU(\sol(a))-\sol(\hat{a})&&\\
		\forall b \in \ell, \Delta(b) &= W_{ab}\Delta(\hat{a})&&\\	
	\end{flalign*}
	
	\vspace{-1.2cm}
	

		\begin{subnumcases}{\forall b \in \ell, \Delta(\hat{b}) =}
			r(b)\Delta(b),&for $W_{bz} > 0$ \\
			\max(\Delta(b),-\sol(b)),&for $W_{bz} < 0$ and $\sol(b)\geq0$\\
			\max(0,\Delta(b)+\sol(b)),&for $W_{bz} < 0$ and $\sol(b)<0$ \quad \, \quad \, \quad		 
		\end{subnumcases}

	
	
	\begin{flalign*}
		\Utility\_\max\nolimits^z(a) &= \Delta(z) = -\sum_{b \in \ell} W_{bz} \Delta(\hat{b})&&
	\end{flalign*}
	
	From paper CITE, we know that $\Utility$ is a safe overapproximation in the sense of following proposition:
	
	\begin{proposition}
		$0 \leq \Improve\max^z(a) \leq \Utility\max^z(a)$. 
	\end{proposition}
	\fi
	
	
	
	
	
	
	
	
	
	\section{Modeling for global robustness}
	
	
	
	\subsection{Using Two Identical Models}
	The most straightforward way is to use two identical MILP models, i.e., to build a model $\mathcal{M}^{large}$ based on two identical MILP models $\mathcal{M},\mathcal{M}'$ with completely disjoint variables (and their own constraints) plus some extra constraints:
	\begin{enumerate}
		\item Add constraints for connecting input nodes $\mathcal{M},\mathcal{M}'$ to meet the requirement $|\boldsymbol{x}-\boldsymbol{y}| \leq \varepsilon$.
		\item Set the optimization objective as the difference between two variables of the same output node in two models $\mathcal{M},\mathcal{M}'$ (because we want to compute $\max|f(\boldsymbol{x}) -f(\boldsymbol{y}) |$).
	\end{enumerate}
	This large model contains twice as many binary variables as $\mathcal{M}$. The computational cost of solving an MILP model grows roughly exponentially with the number of binary variables, and hence it will cost much more time.
	
	We may relax some constraints (by changing binary variables to continuous variables) as in the case of local robustness. However, the problem is that even relaxing a few nodes can cause a significant loss of accuracy, which motivates us to explore other modeling approaches.
	
	\subsection{A Simplified model}
	
	To avoid above problem, we introduce a simplified model that is to use one variable $y_i$ to represent the difference of each two variables: that is, if $x_i$ and $x'_i$ are two variables in $\mathcal{M}$ and $\mathcal{M}'$ representing the same node, then we set $y_i=x_i-x'_i$ and and $\hat{y}_i=\hat{x}_i-\hat{x}_i'$. The relation between $y_i$, $x'_i$ and $\hat{y}_i$ is $\hat{y}_i = \ReLU(x'_i+y_i)-\ReLU(x'_i).$ 
	
	Given  $\gamma_i$ be the upper bound of $y_i$, the constraints for $\hat{y}_i$ are the follows:\begin{align*}
		\hat{y}_i &\leq a \gamma_i               &\quad \hat{y}_i &\geq y_i - a \gamma_i \\
		\hat{y}_i &\geq (a-1) \gamma_i           &\quad \hat{y}_i &\leq y_i + (1-a) \gamma_i
	\end{align*} where $a$ is a binary variable.
	
	
	The following plot illustrates the constraints described above.
	
	\begin{figure}[b!]
		\centering
	\hspace*{10ex}\begin{tikzpicture}
		\begin{axis}[
			xlabel={$y_i$},
			ylabel={$\hat{y}_i$},
			xmin=-2, xmax=2,
			ymin=-2, ymax=2,
			axis lines=center,
			samples=100, 
			unit vector ratio=1 1 1, scale=1, xtick   = {-2,2},
			xticklabels = {$-\gamma_i$,$\gamma_i$},
			yticklabels = {},
			]
			\addplot[blue, thick, fill=blue, fill opacity=0.4] {x} \closedcycle; 
			\addplot[blue, thick] {0}; 
			
			\addplot[only marks, mark=*, mark size=2pt, blue] coordinates {(-2,-2)};
			\node[label={above:$(-\gamma_i,-\gamma_i)$}] at (axis cs: -1.35, -2.1) {};
			
			\addplot[only marks, mark=*, mark size=2pt, blue] coordinates {(2,2)};
			\node[label={above:$(\gamma_i,\gamma_i)$}] at (axis cs: 1.4, 1.5) {};
		\end{axis}
	\end{tikzpicture}
\caption{The possible values of $\val(\hat{y}_i)=\ReLU(\val(x_i))-\ReLU(\val(x'_i))$ depending on $\val(y_i) = \val(x_i)-\val(x'_i)$.}
	\label{fig.1v}
\end{figure}


	
	Based on above constraints, we can sketch this simplified model:
	\begin{enumerate}
		\item For each input node, each output node, and each pre-activation and post-activation node in the hidden layers,  set one variable $y_i$. 
		\item Set constraints for input nodes.
		\item Set linear constraints . In this case, since the meaning of $y_i$ is $x_i-x'_i$, this constraints will not use the bias.
		\item Between pre- and post- activation nodes, set the MILP constraint described above.
	\end{enumerate}
	
	The key point is that, although this model sets 3 variables (and their binary variables) for each node in the network, only $y_i$  contributes to the final results, and we can ignore $x_i,x_i'$ (and their binary variables) during the optimization.
	
	As a result, we can relax the binary variables used to $\hat{x}_i = \ReLU(x_i)$ and $\hat{x}'_i = \ReLU(x'_i)$.
	
	Now the simplified model contains the same number of binary variables as the MILP model for local robustness. Hence this model runs much faster compared to the previous one. The trade-off is lower accuracy compared to the first model under unlimited time. In practice, with a reasonable timeout, this simplified model can usually obtain a better bound.
	
	One major disadvantage of this model is that the solution obtained through optimization may not be valid-i.e., the output computed by the network on the optimized inputs may not equal the output value in the solution. Nevertheless, it is meaningful to compute the upper bound. 
	
	%	However, in practice, with a reasonable timeout, this simplified model can usually obtain a better bound.
	%	
	\subsection{Improving the Accuracy of the Simplified Model}
	We can remove the disadvantage and improve the accuracy of the above simplified model by modifying the constraints, at the cost of increased computational time — a trade-off between accuracy and speed. 
	
	%	(This model has the same binary set, although the meaning of binary variable for $y_i$ is somehow different.)
	
	
	
	%	The exact constraints for $$ \begin{align*}
		%		&\hat{y}_i \geq -\hat{x}'_i \hspace*{1ex} \wedge \hspace*{1ex} \hat{y}_i \leq -\hat{x}'_i+a\beta_i  \hspace*{1ex}\wedge\hspace*{1ex} x_i'+y_i \leq a\beta_i \hspace*{1ex}\wedge\hspace*{1ex}  x_i'+y_i \geq (1-a)\alpha_i \\
		%		&\hat{y}_i \geq -\hat{x}'_i+(x_i'+y_i) \hspace*{1ex}\wedge\hspace*{1ex} \hat{y}_i \leq -\hat{x}'_i+(x_i'+y_i) +(a-1)\alpha_i \\
		%	\end{align*} 
	%	
	%	
	%	Moreover, we can add two more natural constraints: $x_i'+y_i \geq \alpha_i \hspace*{1ex}\wedge\hspace*{1ex}  x_i'+y_i \leq \beta_i.$
	
	
	
	
	
	\begin{figure}[t!]
		\centering
	\hspace*{-10ex}
	\begin{tikzpicture}[scale=0.65]
		\begin{axis}[	axis on top, xlabel = \(x'_i\),
			ylabel = {\(y_i\)}, zlabel = \(\hat{y}_i\),
			set layers=default,
			xmax = 4, xmin = -4,
			ymax = 1, ymin = -1,		
			zmax = 1, zmin = -1,
			unit vector ratio=1 1 1, scale=2.5,  ytick   = {-1,0,1},
			yticklabels = {$-\gamma_i$,$0$,$\gamma_i$}, xtick = {0},
			xticklabels = {$0$}, ztick   = {-1,0,1},
			zticklabels = {$-\gamma_i$,$0$,$\gamma_i$},
			view={35}{14},
			]
			\addplot3[ fill=blue,opacity=0.1, fill opacity=0.4] 
			coordinates {
				(0,0,0) (-1,1,0) (-4,1,0) (-4,-1,0) (0,-1,0) (0,0,0)
			};
			
			\addplot3[	fill=blue,opacity=0.1, fill opacity=0.4] 
			coordinates { (0,0,0) (0,1,1) (4, 1, 1) (4, -1, -1) (1,-1,-1) (0,0,0)
			};
			
			\addplot3[	fill=blue,opacity=0.1, fill opacity=0.4	] 
			coordinates { (0,0,0)  (-1,1,0) (0,1,1) (0,0,0)
			};
			
			\addplot3[	fill=blue,opacity=0.1, fill opacity=0.4	] 
			coordinates { (0,0,0)  (0,-1,0) (1,-1,-1) (0,0,0)
			};
			
			\addplot3[only marks, mark=*, mark size=2pt, blue] coordinates {(1,-1,-1)};
			\node[label={$(\gamma_i,-\gamma_i, -\gamma_i)$}] at (axis cs: 1.2, -0.5 ,-1) {};
			
			\addplot3[only marks, mark=*, mark size=2pt, blue] coordinates {(-1,1,0)};
			\node[label={$(-\gamma_i,\gamma_i, 0)$}] at (axis cs: -1, 0.8 ,0) {};			
			
		\end{axis}
	\end{tikzpicture}
	\caption{The outcome $\val(\hat{y}_i)=\ReLU(\val(x_i))-\ReLU(\val(x'_i))$ 
	depending on $\val(y_i) = \val(x_i)-\val(x'_i)$ and on $\val(x'_i)$.}
	\label{fig.2v}
\end{figure}
	
Given $\gamma_i$ as the upper bound of $y_i$, $\alpha_i,\beta_i$ be the upper and lower bound of $x_i,x_i'$, 
	the precise  constraints for $\hat{y}_i = \ReLU(x'_i+y_i)-\ReLU(x'_i)$ (along with $\hat{x}'_i=\ReLU(x'_i)$) are as follows:
	\begin{align*}
		& \begin{aligned}
			y_i + x'_i &\leq a\beta_i        &
			y_i &+ x'_i \geq (1-a)\alpha_i \\
			x'_i       &\leq a'\beta_i       & 
			x'_i       &\geq (1-a')\alpha_i \\
			\hat{y}_i  &\leq a\gamma_i       &
			\hat{y}_i  &\geq -a'\gamma_i \\
			\hat{y}_i  &\leq y_i + (1-a)\gamma_i  &
			\hat{y}_i  &\geq y_i - (1-a')\gamma_i \\
			\hat{y}_i  &\leq -x'_i + a\beta_i &
			\hat{y}_i  &\geq -x'_i + (1-a')\alpha_i \\
			\hat{y}_i  &\leq y_i + x'_i + (1-a)(-\alpha_i) &
			\hat{y}_i  &\geq y_i + x'_i + a'(-\beta_i)
		\end{aligned}
	\end{align*} Here, $a,a'$ are binary variables, and $a'$ is also the binary variable in the constraints for $\hat{x}'_i=\ReLU(x'_i)$. 
	

Similar to the second model, the constraints on $x_i$ for $\hat{x}_i=\ReLU(x_i)$ are not necessary, or at least can be relaxed without any loss in accuracy.
	
	%	\begin{align*}
		%		& y_i+x'_i \leq a\beta_i \quad\wedge \quad y_i+x'_i\geq (1-a)\alpha_i\\	
		%		& x_i' \leq a'\beta_i \quad\wedge \quad x_i'\geq (1-a')\alpha_i\\
		%		&\hat{y}_i \leq a\gamma_i \quad\wedge \quad	\hat{y}_i \geq -a'\gamma_i \\
		%		&	\hat{y}_i \leq y_i+(1-a)\gamma_i \quad\wedge \quad	\hat{y}_i \geq y_i - (1-a')\gamma_i \\
		%		&	\hat{y}_i \leq -x'_i+a\beta_i \quad\wedge \quad	\hat{y}_i \geq -x'_i+(1-a')\alpha_i \\
		%		&	\hat{y}_i \leq y_i+x'_i+(1-a)(-\alpha_i)\quad\wedge \quad	\hat{y}_i \geq y_i+x'_i+a'(-\beta_i) \\
		%	\end{align*} 
	
	The following plot illustrates the constraints described above.
	

	In practice, when we relax the constraints for a node, we will convert $a_i$ and $a_i'$ to continuous variables together. In principle, we can also choose to change  only one of those two binary variables.
	
	%	The relaxation of this model is similar: let $a$s and $a'$s be continuous variables instead of binary/integer variables. Unlike the first model in this section, relaxing a few nodes does not lose too much accuracy.
	
	A special relaxation approach of this model is to treat the binary variables $a'$ appearing in the above constraints and the binary variables for $\hat{x}'_i=\ReLU(x'_i)$ as distinct variables, and relax only the latter. 
	
	This approach has a similar disadvantage as the model in the previous section: the output computed by the network on the optimized inputs may not equal the output value in the solution. Similarly, it is meaningful to compute the upper bound using this approach. 
	
	
	


	\section{Experimental Evaluation}
	
	We implemented our code in Python 3.8.
	Gurobi 9.52 was used for solving LP and MILP problems. We conducted our evaluation on an AMD Threadripper 7970X  ($32$ cores$@4.0$GHz) with 256 GB of main memory and 2 NVIDIA RTX 4090. 
	
	We will do experiments on two networks for MNIST dataset and another physical model. Two MNIST have the same structure. They are full-connected DNN with 5, 10 outputs and 784 inputs; each hidden layer has 100 neurons. One of them is a normally trained neural network and another is for trained for adversarial robustness. We call them MNIST $5\times100$-Normal and MNIST $5\times 100$-DiffAI. As the name shows MNIST $5\times 100$-DiffAI has a better robustness. The physical network is a simplified model with 10 input neurons, 26 output neurons, and 2 hidden layers with 50 neurons each.
	
	\subsection{Experiment results for robustness (MNIST)}
	

	
	\begin{table}[h!]
	\begin{tabular}{|l|c|c|c|c|}\hline\hline
		$L_1\leq 0.5$ &        Bound $\downarrow$ &  Sol. &      Worst-Case $\uparrow$ &  Time \\\hline \hline
		1v, $375 \times 1$ & 17.4724 & 0.9089  ?? & 0.0917 & 43200s \\\hline 
		2v, $375 \times 2$ & 21.0135 & 8.4672& 0.2229 & 43200s \\\hline		
		2v, $450 \times 2$ & 12.0097 & 3.4408 & {\bf 0.2584} & 43200s \\\hline
		1v, $500 \times 1$ & ?? & ?? & ?? & 43200s \\\hline 
		3v, $500 \times 3$ & ?? & ?? & ?? & 43200s \\\hline 
	 2v, $500 \times 2$ & {\bf 8.7608} & nan & nan & 43200s \\\hline\hline

	 2v, $485 \times 2$ & 20.6460 & 1.9152 & {\bf 0.3722} & 130000s \\\hline\hline

		 2v, $500 \times 2$ & {\bf 8.2573} & nan & nan & 260000s \\\hline\hline
		 
	\end{tabular}
	\caption{Comparison of "1v", "3v" and "2v" models 
	to obtain bounds on $\beta^{0.5}_{0,1}$ on the {\bf full dimension} MNIST DNN, for timeouts of 43200s, 130000s and 260000s, where 375, 450, 485 or 500 ($\times 1$, $\times 2$, $\times 3$) neurons use binary variables.}
\end{table}

On the full space, we reach bounds of $8.2$ (on $\beta^{.5}_{0,1}$), which is too pessimistic as it allows to certify 0 image robust. The worst-case found by the "2v" $485 \times 2$ binary variables after 130000s only displays a difference of 
$0.37$, $20 \times$ away from the bound found.

\begin{figure*}[t!]
	\centering
\includegraphics[scale=0.5]{image.png} \hspace{0.8cm}
\includegraphics[scale=0.5]{perturb.png}
\caption{An improbable image and its perturbation (difference around x=11, y=18) 
with maximal $\beta^{.5}_{0,1}=0.37$ for MNIST as obtained by the "2v" $485 \times 2$ model in full 758 dimension image space.}
\end{figure*}	




\paragraph{Reduced space}

We now consider a PCA model order reduction to avoid considering improbable images, 
speed up obtaining the bounds, and obtain less pessimistic bounds.
We settle on 20 orders, as the MNIST DNN considered run on a images obtained from projecting to the reduced order and projected back to the full dimension 
display the same accuracy of $97$\% as the DNN on the original images, meaning no accuracy is lost from reducing to 20 dimensions, which is the only thing which matters. On such a reduced space, bounds obtained are much more precise, and one can certify robustness of images online.



	\begin{table}[h!]
	\begin{tabular}{|l|c|c|c|c|}\hline\hline
		$L_1\leq 0.5$ &        Bound $\downarrow$ &  Sol. &      Worst-Case $\uparrow$ &  Time \\\hline \hline
		
1v, $?? \times 1$ & ?? & ?? & ?? & ?? \\\hline 
		3v, $?? \times 3$ & ?? & ?? & ?? & ?? \\\hline 
	 2v, $?? \times 2$ & ?? & ?? & ?? & ?? \\\hline\hline
	 
		1v, $?? \times 1$ & ?? & ?? & ?? & ?? \\\hline 
		3v, $?? \times 3$ & ?? & ?? & ?? & ?? \\\hline 
	 2v, $?? \times 2$ & ?? & ?? & ?? & ?? \\\hline\hline
	 
	\end{tabular}
	\caption{Comparison of "1v", "3v" and "2v" models 
	to obtain bounds on $\beta^{0.5}_{0,1}$ on the {\bf 20 dimension}  reduced order MNIST DNN, for timeouts of ??s, and ??s , where ??,  or 500 ($\times 1$, $\times 2$, $\times 3$) neurons use binary variables.}
\end{table}






\begin{figure*}[t!]
	\centering
\includegraphics[scale=0.5]{redimage.png} \hspace{0.8cm}
\includegraphics[scale=0.5]{redperturb.png}
\caption{An image and its perturbation from the 20-dimension reduced space with maximal $\beta^{.5}_{0,1}=0.17$ for MNIST as obtained by the "2v" $500 \times 2$ model.}
\end{figure*}	

	
	
\begin{table}[h!]
	\begin{tabular}{|l|c|c|c|c|}\hline\hline
		model &    $L_1\leq 0.5$ & $L_1\leq 1$ & $L_1\leq 1.5$ &  $L_1\leq 2$ \\\hline \hline
		1v, $? \times 1$ & $80 \%$ & $32\%$ & $7\%$ & $0\%$ \\\hline
		3v, $? \times 3$ & {\bf 86 \%} & {\bf 53\%} & {\bf 20\%} & {\bf 4\%} \\\hline
		2v, $? \times 2$ & 84\% & 51\% & 16\% & 4\% \\\hline \hline
	\end{tabular}
	\caption{Comparison of "1v", "3v" and "2v" models 
	in the percentage of images they can certify in real-time for different values of $L_1$ perturbations.}
\end{table}


	


\subsection{Experiment results for regression (Pipe strain)}
	
	For the pipe system, to find a example of large difference on outputs is one of the main aims.

	\begin{figure*}[t!]
\includegraphics[scale=0.5]{deform.png} \hspace{0.8cm}
\includegraphics[scale=0.5]{strain.png}
\caption{2 slightly different deformations and their associated quite different strain as obtained by the "2v" $100 \times 2$ model.}
\end{figure*}	


	The network of Pipe system is relatively simple and in the tests, we will open all $\ReLU$ nodes.

	

	We will compare the results of different modeling methods.
	
	\vspace*{1ex}
	
	\iffalse
	\begin{table}[h!]
	\begin{tabular}{|l|l|l|l|l|}\hline
		$L_1\leq 0.83$ &        Bound $\downarrow$ &  Solution $\uparrow$ &      Real $\uparrow$ &  Time \\\hline
		1v,open 100 &     {\bf 0.035613} &  0.035613 &                       0.01288 & 10608 \\\hline
		3v,open 100 &     0.040074 &  0.028934 &                      0.021441 & 10922 \\\hline
		%3v,open 100 &     0.039824 &  0.028832 &                      0.022255 & 22153 \\\hline
		2v,open 100 &     0.046719 &  0.024364 &  {\bf 0.024436} & 10922 \\\hline
	\end{tabular}
	\caption{Comparison of 1v,2v and 3v models on the pipe system with a fixed timeout of 10.000s.}
\end{table}
\fi
	
		
	\begin{table}[h!]
	\begin{tabular}{|l|c|c|c|c|}\hline\hline
		$L_1\leq 0.83$ &        Bound $\downarrow$ &  Sol. &      Worst-Case $\uparrow$ &  Time \\\hline \hline
		1v, $100 \times 1$ &     {\bf 0.0356} &  $0.0356$ & $0.0191$ &   1000s \\\hline
		3v, $100 \times 3$&     0.0414 &  0.0254 &  0.0166 &  1000s \\\hline
		2v, $100 \times 2$&     0.0418 &  0.0229 &   {\bf 0.0229} &  1000s \\\hline \hline
		3v, $90 \times 3$&      ?? &  ?? &  ?? & 14523s \\\hline
		3v, $100 \times 3$&      {\bf 0.0350} &  0.0272 &  0.0216 & 14523s \\\hline
		2v, $90 \times 2$&     ?? &  ?? &   ?? & 14523s \\\hline
		2v, $100 \times 2$&     0.0360 &  0.0236 &    {\bf 0.0236} & 14523s \\\hline \hline
		3v, $100 \times 3$&     {\bf 0.0329} &  0.0277 &  0.0165 & 72126s \\\hline
		2v, $100 \times 2$&     0.03375 &  0.0244 &  {\bf 0.0245} & 72126s \\\hline\hline
	\end{tabular}
	\caption{Comparison of "1v", "3v" and "2v" models on the pipe system with a fixed timeout of 1000s, 14523s and 72126s, where 90 ($\times 2$, $\times 3$) or all 100 ($\times 1, \times 2,\times 3$) neurons use binary variables.
	L1 corresponds to $3.9$ or $4$, and results should be the sum of 10 pixels, so around 10 times higher values.}
\end{table}

	
	
	
	
	
	 
	
	
	\newpage

	\hfill

	\newpage
	
	
	\bibliography{references}
	
	
\end{document}
