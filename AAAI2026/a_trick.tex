\subsection*{A trick of $L_1$ norm constraints on input}

A trick to set $L_1$ norm constraints on the input space into linear constraints is described below.

Suppose the dimension of input space is $n$, and $x_i,i\leq n$ are the input variables, $\vec{x}$ is the input vector, and we hope to set a constraints of $\vec{x}$ that $\|\vec{x}\| \leq c$,  then we can set the constraints by add variables $A_i,i<n$ with the only constraints \begin{align}\label{L1constraint}\begin{cases}
	A_i \geq x_i &\text{ for }i \leq n\\ 
	A_i\geq -x_i &\text{ for }i \leq n\\
	\Sigma_{i\leq n} A_i \leq c	&\end{cases}
\end{align}

We show that this constraint is equivalent to $\|\vec{x}\| \leq c$.

The first two lines of constraints \ref{L1constraint} is equivalent to that $A_i\geq |x_i|$. So the whole constraints is non-weaker than $\|\vec{x}\| \leq c$.

Next, for every instance that satisfy $\|\vec{x}\| \leq c$, then there is also an instance which satisfy constraints \ref{L1constraint}: simply let all $A_i = |x_i|$. And this shows that $\|\vec{x}\| \leq c$ is non-weaker than constraints \ref{L1constraint}. Therefore they are equivalent.
