\section{Notations and Preliminaries}
	
	In this paper, we will use lower case latin $a$ for scalars, bold $\boldsymbol{z}$ for vectors, 
	capitalized bold $\boldsymbol{W}$ for matrices, similar to notations in \cite{crown}.
	To simplify the notations, we restrict the presentation to feed-forward, 
	fully connected ReLU Deep Neural Networks (DNN for short), where the activation function is $\ReLU : \mathbb{R} \rightarrow \mathbb{R}$ with
	$\ReLU(x)=x$ for $x \geq 0$ and $\ReLU(x)=0$ for $x \leq 0$, which we extend componentwise on vectors.
	
	%In this paper, we will not use tensors with a dimension higher than matrices: those will be flattened.
	
	%\subsection{Neural Network and Verification}
	
	
	% testtesttesttest
	An $\ell$-layer DNN is provided by $\ell$ weight matrices 
	$\boldsymbol{W}^i \in \mathbb{R}^{d_i\times d_{i-1}}$
	and $\ell$ bias vectors $\vb^i \in \mathbb{R}^{d_i}$, for $i=1, \ldots, \ell$.
	We call $d_i$ the number of neurons of hidden layer 
	$i \in \{1, \ldots, \ell-1\}$,
	$d_0$ the input dimension, and $d_\ell$ the output dimension.
	
	Given an input vector $\boldsymbol{z}^0 \in \mathbb{R}^{d_0}$, 
	denoting $\hat{\boldsymbol{z}}^{0}={\boldsymbol{z}}^0$, we define inductively the value vectors $\boldsymbol{z}^i,\hat{\vz}^i$ at layer $1 \leq i \leq \ell$ with
	\begin{align*}
		\boldsymbol{z}^{i} = \boldsymbol{W}^i\cdot \hat{\boldsymbol{z}}^{i-1}+ \vb^i \qquad \, \qquad
		\hat{\boldsymbol{z}}^{i} = \ReLU({\boldsymbol{z}}^i).
	\end{align*} 
	
	The vector $\hat{\boldsymbol{z}}$ is called post-activation values, 
	$\boldsymbol{z}$ is called pre-activation values, 
	and $\boldsymbol{z}^{i}_j$ is used to call the $j$-th neuron in the $i$-th layer. 
	For $\boldsymbol{x}=\vz^0$ the (vector of) input, we denote by $f(\boldsymbol{x})=\vz^\ell$ the output. Finally, pre- and post-activation neurons are called \emph{nodes}, and when we refer to a specific node/neuron, we use $a,b,c,d,n$ to denote them, and $W_{a,b} \in \mathbb{R}$ to denote the weight from neuron $a$ to $b$. Similarly, for input $\boldsymbol{x}$, we denote by $\val_{\boldsymbol{x}}(a)$ the value of neuron $a$ when the input is $\boldsymbol{x}$.	For convenience, we write $n < z$ if neuron $n$ is on a layer before $\ell_z$, and $n \leq z$ if $n< z$ or $n=z$.
	
    In this paper, we consider the global verification problem, where we optimize over all image $I$ and all perturbation $I'$ of $I$ with $|I-I'| \leq \varepsilon$. 	We will consider three kinds of variables: 
    \begin{itemize}
    \item $x_j,\hat{x}_j$, for nodes $j$ with input image $I$, 
    \item $x'_j,\hat{x}'_j$, for nodes $j$ with input the perturbed $I'$, 
    \item  $y_j = x_j - x'_j$ the {\em diff variable}, with 
    $\hat{y}_j = \hat{x}_j - \hat{x}'_j$ (and {\em not} 
    $\hat{y}_i = \ReLU(x_j-x'_j)$). 
    \end{itemize}
     
    %Concerning the verification problem, we focus on the global-robustness question. Global robustness asks to determine how the output of a neural network will be affected under a certain kind of small perturbations to any possible input. In this view, Lipschitz continuity is a good characterization of global robustness.
	
	
	
	\iffalse
	
	\section{Global robustness and Lipschitz constant}
	
	
	Recall the definition of Lipschitz continuity:
	under distance $d$, a function $f(x)$ is Lipschitz continuous with respect to constant $K$ if:
	\begin{align*}
		\forall \boldsymbol{x} \forall\boldsymbol{y} (|f(\boldsymbol{x}) -f(\boldsymbol{y}) |\leq K|\boldsymbol{x}-\boldsymbol{y}|)
	\end{align*} 
	In our practice, when we need global robustness, we will compute an optimization question respect to a certain number $\varepsilon$:	\begin{align}\label{global_robustness}
		\max_{|\boldsymbol{x}-\boldsymbol{y}| \leq \varepsilon} |f(\boldsymbol{x}) -f(\boldsymbol{y}) |
	\end{align} And this will lead to the following definition
	
	\begin{definition}[$\varepsilon$-diff bound]
		Suppose we have a function $f$ from $\mathbb{R}^n$ to $\mathbb{R}^m$ and $||$ is $L_\infty$ norm. 
		
		For a number $\varepsilon\in\mathbb{R}$, an $\varepsilon$-diff bound $D_\varepsilon$ is a number such that for any inputs $x,y$: \begin{align*}
			|x-y|\leq \varepsilon \implies |f(x)-f(y)| \leq D_\varepsilon \cdot \varepsilon
		\end{align*}
		
	\end{definition}
	
	From $\varepsilon$-diff bound, we cannot directly obtain a Lipschitz bound for the function, but we can get the following weaker bound:
	
	\begin{definition}[Lipschitz above $\varepsilon$ constant]
		Suppose we have a function $f$ from $\mathbb{R}^n$ to $\mathbb{R}^m$ and $||$ is $L_\infty$ norm. 
		
		For a number $\varepsilon\in\mathbb{R}$, a Lipschitz above $\varepsilon$ constant  $K_\varepsilon$,  is a number such that for any inputs $x,y$: \begin{align*}
			|x-y|\geq \varepsilon &\implies |f(x)-f(y)| \leq K_\varepsilon \cdot |x-y|\\
			|x-y|<\varepsilon &\implies |f(x)-f(y)| \leq K_\varepsilon \cdot \varepsilon\\
		\end{align*}		
	\end{definition}
	
	
	\begin{proposition}
		
		Suppose $D$ is an $\varepsilon$-diff bound for $f(x)$. Then for any $N\in\mathbb{Z}^+$, $D\frac{N+1}{N}$ is a Lipschitz about $N\varepsilon$ constant.
		
		That is, for any two inputs $x,y$, if \begin{align*}
			|x-y|\leq \varepsilon \implies |f(x)-f(y)| \leq D \cdot \varepsilon,
		\end{align*} then 	 \begin{align*}
			|x-y|\geq N\varepsilon &\implies |f(x)-f(y)| \leq D\frac{N+1}{N} \cdot |x-y|\\
			|x-y|<N\varepsilon &\implies |f(x)-f(y)| \leq D\frac{N+1}{N} \cdot N\varepsilon\\
		\end{align*}
	\end{proposition}
	
	\textbf{Proof.} We fix the number $N\in\mathbb{Z}^+$ and assume that we have two inputs $x, y$.
	
	The first case, $|x-y|\geq N\varepsilon$. Then we assume $|x-y| \in [M\varepsilon ,  (M+1)\varepsilon]$ for another integer $M\geq N$. Then we can divide the line segment between $x, y$ into $M+1$ pieces: $x_0 = x, x_1, x_2, \cdots, x_{M+1} = y$ such that $|x_i-x_{i+1}| \leq \varepsilon$ and apply the definition of $\varepsilon$-diff bound $D$ for each pieces:\begin{align*}
		|f(x)-f(y)| &= |f(x_{M+1})-f(x_M)+\cdots+f(x_1)-f(x_0)|\\
		&\leq |f(x_{M+1})-f(x_M)|+\cdots+|f(x_1)-f(x_0)|\\
		&\leq D\varepsilon + \cdots +D\varepsilon = (M+1)D\varepsilon
	\end{align*}
	Hence,\begin{align*}
		|f(x)-f(y)| &\leq (M+1)D\varepsilon \leq D\cdot (M+1)\varepsilon \frac{|x-y|}{M\varepsilon}\\
		&= D\cdot\frac{M+1}{M} |x-y|	\leq   D\cdot\frac{N+1}{N} |x-y|		
	\end{align*}
	The second case, $|x-y|< N\varepsilon$. Similarly we can divide the line segment between $x, y$ into $N$ pieces and then $|f(x)-f(y)|\leq D N\varepsilon\leq |f(x)-f(y)|\leq D \frac{N+1}{N} N\varepsilon$.
	
	This ends the proof.
	\hfill $\square$
	
	In practice, Lipschitz above $\varepsilon$ constant is already sufficient, since in most cases we care about the absolute difference under input perturbations, not the ratio. By combining the above proposition, the computation of $\varepsilon$-diff bound can satisfy our aim.
	
	
	Moreover, we have one more proposition connecting $\varepsilon$-diff bound and Lipschitz above $\varepsilon$ constant.
	
	
	\begin{proposition}
		For any $N\in\mathbb{Z}^+$, suppose for any $a$ in $\{\frac{N}{N}\varepsilon,\frac{N+1}{N}\varepsilon,\cdots, \frac{2N-1}{N}\varepsilon\}$, $D$ is an $a$-diff bound for $f(x)$. 
		
		Then $D\frac{N+1}{N}$ is a Lipschitz about $\varepsilon$ constant:\begin{align*}
			|x-y|\geq \varepsilon &\implies |f(x)-f(y)| \leq D\frac{N+1}{N} \cdot |x-y|\\
			|x-y|<\varepsilon &\implies |f(x)-f(y)| \leq D\frac{N+1}{N} \cdot \varepsilon\\
		\end{align*}
	\end{proposition}
	
	\textbf{Proof.}
	We fix the number $N\in\mathbb{Z}^+$ and assume we have two inputs $x, y$.
	
	For the case that $|x-y|<\varepsilon$, this is trivial by definition.
	
	For the case that $|x-y|\geq \varepsilon$, there exists a sum $x_1+x_2+\cdots+x_n$ by numbers from (allowing repetitions) $\{\frac{N}{N}\varepsilon,\frac{N+1}{N}\varepsilon,\cdots, \frac{2N-1}{N}\varepsilon\}$ such that \begin{align*}
		\varepsilon \leq x_1+x_2+\cdots+x_n -\frac{1}{N}\varepsilon \leq |x-y| \leq x_1+x_2+\cdots+x_n
	\end{align*}
	By assumption, divide the line segment from $x$ to $y$ into pieces according to $x_1, x_2,\cdots,x_n$, then we will have $$|f(x)-f(y)|\leq Dx_1+Dx_2+\cdots+Dx_n.$$
	
	Hence,\begin{align*}
		\dfrac{|f(x)-f(y)|}{|x-y|} &\leq \dfrac{Dx_1+Dx_2+\cdots+Dx_n}{x_1+x_2+\cdots+x_n -\frac{1}{N}\varepsilon}\\
		& \leq D\cdot( 1+  \dfrac{\frac{1}{N}\varepsilon}{\varepsilon})= D \frac{N+1}{N}\\
	\end{align*}
	This ends the proof.
	\hfill $\square$
	
	\fi

	
	\section{(partial) MILP}
	
	
	
	\subsection{MILP encoding for local ReLU and LP relaxation}
	
	Mixed Integer Linear Programming (MILP) value abstraction is an accurate encoding of ReLU DNNs. For an unstable neuron $n$, that is with values 
    $x \in [\LB(n),\UB(n)]$ with $\LB(n)<0<\UB(n)$, 
    the value $\hat{x}$ of $\ReLU(x)$ can be encoded exactly in an MILP formula with one binary / integer variable $a$ valued in $\{0,1\}$, using constants $\UB(n),\LB(n)$ with 4 constraints \cite{MILP}:
	
	\vspace{-0.4cm}
	\begin{equation} 
        \hat{x} \geq x \, \wedge \, \hat{x} \geq 0 \, \wedge \, \hat{x} \leq \UB(n) a \, \wedge \, \hat{x} \leq x-\LB(n) (1-a)
		\label{eq11}
	\end{equation}
	
\begin{proposition}
\cite{MILP}
\label{Prop1}
A solution $x,\hat{x},a$  of the above MILP program satisfies $\hat{x} = \ReLU(x)$,
and $a=1$ if $x> 0$ and $a=0$ if $x< 0$ (both are possible if $x=0$).
\end{proposition}

	%For all $x \in [\LB(n),\UB(n)] \setminus 0$, there exists a unique solution $(a,\hat{x})$ that meets these constraints, with $\hat{x}=\ReLU(x)$ \cite{MILP}. The value of $a$ is 0 if $x < 0$, and 1 if $x>0$, and can be either if $x=0$. This encoding approach can be applied to every (unstable) ReLU node, and optimizing its value can help getting more accurate bounds. However, for networks with hundreds of {\em unstable} nodes, the resulting MILP formulation will contain numerous integer variables and generally bounds obtained will not be accurate, even using powerful commercial solvers such as Gurobi.
	
    \iffalse
	The global structure is as follows, using Gurobi as an example:
	\begin{enumerate}
		\item For each input node, each output node, and each pre-activation and post-activation node in the hidden layers,  set one variable. 
		\item Set constraints for input nodes.
		\item For each pre-activation node in a hidden layer (and each output node), set linear constraints relating them to the post-activation or input nodes in the previous layer they connect to.
		\item Between pre- and post- activation nodes, set the MILP constraint described above.
	\end{enumerate} 
    
    \fi
    
    We use $\mathcal{M}$ to denote the whole MILP model, where each unstable ReLU are encoded in the above way with one integer variable. 
	The encoding from $(\hat{x}_j)_{j \text{ in layer } i}$ to 
	$(x_{j'})_{j' \text{ in layer } i+1}$ variables is simply the linear combination 
	$\boldsymbol{z}^{i} = \boldsymbol{W}^i\cdot \hat{\boldsymbol{z}}^{i-1}+ \vb^i$.
	This exact MILP encoding is often too computationally intensive, as the worst-case complexity of MILP is exponential in the number of integer variables \cite{DivideAndSlide}.
    
    
	MILP instances can be linearly relaxed into LP over-abstraction, where variables $a$ originally restricted to integers in $\{0,1\}$ (binary) are relaxed to real numbers in the interval $[0,1]$, while maintaining the same encoding. As solving LP instances is polynomial time, this optimization is significantly more efficient. However, this efficiency comes at the cost of precision, often resulting in less stringent bounds. This approach is termed the {\em LP abstraction}.
    % We invoke a folklore result on the LP relaxation of (\ref{eq11}), for which we provide a direct and explicit proof.
	
	
	%\subsection{partial MILP}
	
	Intermediate between these 2 extreme cases, there is {\em partial MILP} 
    (pMILP for short) to get trade-offs between accuracy and runtime
	\cite{DivideAndSlide}:
	Let $X$ be a subset of the set of unstable neurons, and $n$ a neuron for which we want to compute upper and lower bounds on values: the pMILP based on $X$ to compute neuron $n$ uses the MILP encpoding (\ref{eq11}), where variable $a$ is:
	\begin{itemize}
		\item binary for neurons in $X$ (exact encoding of the ReLU),
		\item linear for neurons not in $X$ (linear relaxation).
	\end{itemize}
	We will denote the above model by MILP$_X$. We say that a node is {\em opened} if it is in $X$. 
	
	%To reduce the runtime, we will limit the size of subset $X$. This a priori hurts accuracy. To recover some of this accuracy, we use an iterative approach: computing lower and upper bounds $\LB,\UB$ for neurons $n$ of a each layer iteratively, that are used when computing values of the next layer.
	
	\iffalse
	\subsection{SAS}
	
	
	In pMILP, to decide the set $X$, we introduce the method {\em Solution-Aware Scoring} (SAS)
	to evaluate accurately how opening a ReLU impacts the accuracy. Again, here we use the definition from paper CITE. For details and explanation, see CITE.
	
	
	Assume that we want to compute an upper bound for neuron $z$ on layer $\ell_z$. For each node $n<z$, we denote ($\Sol\_\max_X^z(n))_{n \leq z}$ a solution of $\mathcal{M}_X$ maximizing $z$: $\Sol\_\max_X^z(z)$ is the maximum of $z$ under $\mathcal{M}_X$; and we denote $(\sol(n))_{n \leq z} = (\Sol\_\max_\emptyset^z(n))_{n \leq z}$ a solution maximizing the value for $z$ when all ReLU use the LP relaxation. Moreover,  we define the function
	$\Improve\_\max^z(n)=$ $\sol(z) - \Sol\_\max_{\{n\}}^z(z)$, 
	accurately represents how much opening neuron $n < z$ reduces the maximum computed for $z$
	compared with using only LP. 
	
	First, SAS will call solvers to compute the LP model to get a solution, which is reasonably fast as there is no binary variables. 
	
	Next, for a neuron $b$ on the layer before layer $\ell_z$, we define:
	
	
	\vspace{-0.4cm}
	$$\Utility\_\max\nolimits^z(b) = W_{bz} \times (\sol(\hat{b})- \ReLU(\sol(b)))$$
	\vspace{-0.4cm}
	
	
	And for a neuron $a$ two layers before $\ell_z$, 
	$b$ denoting neurons in the layer $\ell$ just before $\ell_z$.
	Recall the rate $r(b)=\frac{\max(0,\UB(b))}{\max(0,\UB(b))-\min(0,\LB(b))} \in [0,1]$.
	We define:
	
	
	\begin{flalign*}
		\Delta(\hat{a}) &= \ReLU(\sol(a))-\sol(\hat{a})&&\\
		\forall b \in \ell, \Delta(b) &= W_{ab}\Delta(\hat{a})&&\\	
	\end{flalign*}
	
	\vspace{-1.2cm}
	

		\begin{subnumcases}{\forall b \in \ell, \Delta(\hat{b}) =}
			r(b)\Delta(b),&for $W_{bz} > 0$ \\
			\max(\Delta(b),-\sol(b)),&for $W_{bz} < 0$ and $\sol(b)\geq0$\\
			\max(0,\Delta(b)+\sol(b)),&for $W_{bz} < 0$ and $\sol(b)<0$ \quad \, \quad \, \quad		 
		\end{subnumcases}

	
	
	\begin{flalign*}
		\Utility\_\max\nolimits^z(a) &= \Delta(z) = -\sum_{b \in \ell} W_{bz} \Delta(\hat{b})&&
	\end{flalign*}
	
	From paper CITE, we know that $\Utility$ is a safe overapproximation in the sense of following proposition:
	
	\begin{proposition}
		$0 \leq \Improve\max^z(a) \leq \Utility\max^z(a)$. 
	\end{proposition}
	\fi