\documentclass[letterpaper]{article} % DO NOT CHANGE THIS
\usepackage[submission]{aaai2026}  % DO NOT CHANGE THIS
\usepackage{times}  % DO NOT CHANGE THIS
\usepackage{helvet}  % DO NOT CHANGE THIS
\usepackage{courier}  % DO NOT CHANGE THIS
\usepackage[hyphens]{url}  % DO NOT CHANGE THIS
\usepackage{graphicx} % DO NOT CHANGE THIS
\urlstyle{rm} % DO NOT CHANGE THIS
\def\UrlFont{\rm}  % DO NOT CHANGE THIS
\usepackage{natbib}  % DO NOT CHANGE THIS AND DO NOT ADD ANY OPTIONS TO IT
\usepackage{caption} % DO NOT CHANGE THIS AND DO NOT ADD ANY OPTIONS TO IT
\frenchspacing  % DO NOT CHANGE THIS
\setlength{\pdfpagewidth}{8.5in} % DO NOT CHANGE THIS
\setlength{\pdfpageheight}{11in} % DO NOT CHANGE THIS
%
% These are recommended to typeset algorithms but not required. See the subsubsection on algorithms. Remove them if you don't have algorithms in your paper.
\usepackage{algorithm}
\usepackage{algorithmic}
\pagestyle{plain}
\usepackage{threeparttable}
%%%%% NEW MATH DEFINITIONS %%%%%

\usepackage{amsmath,amsfonts,bm}

% Mark sections of captions for referring to divisions of figures
\newcommand{\figleft}{{\em (Left)}}
\newcommand{\figcenter}{{\em (Center)}}
\newcommand{\figright}{{\em (Right)}}
\newcommand{\figtop}{{\em (Top)}}
\newcommand{\figbottom}{{\em (Bottom)}}
\newcommand{\captiona}{{\em (a)}}
\newcommand{\captionb}{{\em (b)}}
\newcommand{\captionc}{{\em (c)}}
\newcommand{\captiond}{{\em (d)}}

% Highlight a newly defined term
\newcommand{\newterm}[1]{{\bf #1}}


% Figure reference, lower-case.
\def\figref#1{figure~\ref{#1}}
% Figure reference, capital. For start of sentence
\def\Figref#1{Figure~\ref{#1}}
\def\twofigref#1#2{figures \ref{#1} and \ref{#2}}
\def\quadfigref#1#2#3#4{figures \ref{#1}, \ref{#2}, \ref{#3} and \ref{#4}}
% Section reference, lower-case.
\def\secref#1{section~\ref{#1}}
% Section reference, capital.
\def\Secref#1{Section~\ref{#1}}
% Reference to two sections.
\def\twosecrefs#1#2{sections \ref{#1} and \ref{#2}}
% Reference to three sections.
\def\secrefs#1#2#3{sections \ref{#1}, \ref{#2} and \ref{#3}}
% Reference to an equation, lower-case.
\def\eqref#1{equation~\ref{#1}}
% Reference to an equation, upper case
\def\Eqref#1{Equation~\ref{#1}}
% A raw reference to an equation---avoid using if possible
\def\plaineqref#1{\ref{#1}}
% Reference to a chapter, lower-case.
\def\chapref#1{chapter~\ref{#1}}
% Reference to an equation, upper case.
\def\Chapref#1{Chapter~\ref{#1}}
% Reference to a range of chapters
\def\rangechapref#1#2{chapters\ref{#1}--\ref{#2}}
% Reference to an algorithm, lower-case.
\def\algref#1{algorithm~\ref{#1}}
% Reference to an algorithm, upper case.
\def\Algref#1{Algorithm~\ref{#1}}
\def\twoalgref#1#2{algorithms \ref{#1} and \ref{#2}}
\def\Twoalgref#1#2{Algorithms \ref{#1} and \ref{#2}}
% Reference to a part, lower case
\def\partref#1{part~\ref{#1}}
% Reference to a part, upper case
\def\Partref#1{Part~\ref{#1}}
\def\twopartref#1#2{parts \ref{#1} and \ref{#2}}

\def\ceil#1{\lceil #1 \rceil}
\def\floor#1{\lfloor #1 \rfloor}
\def\1{\bm{1}}
\newcommand{\train}{\mathcal{D}}
\newcommand{\valid}{\mathcal{D_{\mathrm{valid}}}}
\newcommand{\test}{\mathcal{D_{\mathrm{test}}}}

\def\eps{{\epsilon}}


% Random variables
\def\reta{{\textnormal{$\eta$}}}
\def\ra{{\textnormal{a}}}
\def\rb{{\textnormal{b}}}
\def\rc{{\textnormal{c}}}
\def\rd{{\textnormal{d}}}
\def\re{{\textnormal{e}}}
\def\rf{{\textnormal{f}}}
\def\rg{{\textnormal{g}}}
\def\rh{{\textnormal{h}}}
\def\ri{{\textnormal{i}}}
\def\rj{{\textnormal{j}}}
\def\rk{{\textnormal{k}}}
\def\rl{{\textnormal{l}}}
% rm is already a command, just don't name any random variables m
\def\rn{{\textnormal{n}}}
\def\ro{{\textnormal{o}}}
\def\rp{{\textnormal{p}}}
\def\rq{{\textnormal{q}}}
\def\rr{{\textnormal{r}}}
\def\rs{{\textnormal{s}}}
\def\rt{{\textnormal{t}}}
\def\ru{{\textnormal{u}}}
\def\rv{{\textnormal{v}}}
\def\rw{{\textnormal{w}}}
\def\rx{{\textnormal{x}}}
\def\ry{{\textnormal{y}}}
\def\rz{{\textnormal{z}}}

% Random vectors
\def\rvepsilon{{\mathbf{\epsilon}}}
\def\rvtheta{{\mathbf{\theta}}}
\def\rva{{\mathbf{a}}}
\def\rvb{{\mathbf{b}}}
\def\rvc{{\mathbf{c}}}
\def\rvd{{\mathbf{d}}}
\def\rve{{\mathbf{e}}}
\def\rvf{{\mathbf{f}}}
\def\rvg{{\mathbf{g}}}
\def\rvh{{\mathbf{h}}}
\def\rvu{{\mathbf{i}}}
\def\rvj{{\mathbf{j}}}
\def\rvk{{\mathbf{k}}}
\def\rvl{{\mathbf{l}}}
\def\rvm{{\mathbf{m}}}
\def\rvn{{\mathbf{n}}}
\def\rvo{{\mathbf{o}}}
\def\rvp{{\mathbf{p}}}
\def\rvq{{\mathbf{q}}}
\def\rvr{{\mathbf{r}}}
\def\rvs{{\mathbf{s}}}
\def\rvt{{\mathbf{t}}}
\def\rvu{{\mathbf{u}}}
\def\rvv{{\mathbf{v}}}
\def\rvw{{\mathbf{w}}}
\def\rvx{{\mathbf{x}}}
\def\rvy{{\mathbf{y}}}
\def\rvz{{\mathbf{z}}}

% Elements of random vectors
\def\erva{{\textnormal{a}}}
\def\ervb{{\textnormal{b}}}
\def\ervc{{\textnormal{c}}}
\def\ervd{{\textnormal{d}}}
\def\erve{{\textnormal{e}}}
\def\ervf{{\textnormal{f}}}
\def\ervg{{\textnormal{g}}}
\def\ervh{{\textnormal{h}}}
\def\ervi{{\textnormal{i}}}
\def\ervj{{\textnormal{j}}}
\def\ervk{{\textnormal{k}}}
\def\ervl{{\textnormal{l}}}
\def\ervm{{\textnormal{m}}}
\def\ervn{{\textnormal{n}}}
\def\ervo{{\textnormal{o}}}
\def\ervp{{\textnormal{p}}}
\def\ervq{{\textnormal{q}}}
\def\ervr{{\textnormal{r}}}
\def\ervs{{\textnormal{s}}}
\def\ervt{{\textnormal{t}}}
\def\ervu{{\textnormal{u}}}
\def\ervv{{\textnormal{v}}}
\def\ervw{{\textnormal{w}}}
\def\ervx{{\textnormal{x}}}
\def\ervy{{\textnormal{y}}}
\def\ervz{{\textnormal{z}}}

% Random matrices
\def\rmA{{\mathbf{A}}}
\def\rmB{{\mathbf{B}}}
\def\rmC{{\mathbf{C}}}
\def\rmD{{\mathbf{D}}}
\def\rmE{{\mathbf{E}}}
\def\rmF{{\mathbf{F}}}
\def\rmG{{\mathbf{G}}}
\def\rmH{{\mathbf{H}}}
\def\rmI{{\mathbf{I}}}
\def\rmJ{{\mathbf{J}}}
\def\rmK{{\mathbf{K}}}
\def\rmL{{\mathbf{L}}}
\def\rmM{{\mathbf{M}}}
\def\rmN{{\mathbf{N}}}
\def\rmO{{\mathbf{O}}}
\def\rmP{{\mathbf{P}}}
\def\rmQ{{\mathbf{Q}}}
\def\rmR{{\mathbf{R}}}
\def\rmS{{\mathbf{S}}}
\def\rmT{{\mathbf{T}}}
\def\rmU{{\mathbf{U}}}
\def\rmV{{\mathbf{V}}}
\def\rmW{{\mathbf{W}}}
\def\rmX{{\mathbf{X}}}
\def\rmY{{\mathbf{Y}}}
\def\rmZ{{\mathbf{Z}}}

% Elements of random matrices
\def\ermA{{\textnormal{A}}}
\def\ermB{{\textnormal{B}}}
\def\ermC{{\textnormal{C}}}
\def\ermD{{\textnormal{D}}}
\def\ermE{{\textnormal{E}}}
\def\ermF{{\textnormal{F}}}
\def\ermG{{\textnormal{G}}}
\def\ermH{{\textnormal{H}}}
\def\ermI{{\textnormal{I}}}
\def\ermJ{{\textnormal{J}}}
\def\ermK{{\textnormal{K}}}
\def\ermL{{\textnormal{L}}}
\def\ermM{{\textnormal{M}}}
\def\ermN{{\textnormal{N}}}
\def\ermO{{\textnormal{O}}}
\def\ermP{{\textnormal{P}}}
\def\ermQ{{\textnormal{Q}}}
\def\ermR{{\textnormal{R}}}
\def\ermS{{\textnormal{S}}}
\def\ermT{{\textnormal{T}}}
\def\ermU{{\textnormal{U}}}
\def\ermV{{\textnormal{V}}}
\def\ermW{{\textnormal{W}}}
\def\ermX{{\textnormal{X}}}
\def\ermY{{\textnormal{Y}}}
\def\ermZ{{\textnormal{Z}}}

% Vectors
\def\vzero{{\bm{0}}}
\def\vone{{\bm{1}}}
\def\vmu{{\bm{\mu}}}
\def\vtheta{{\bm{\theta}}}
\def\va{{\bm{a}}}
\def\vb{{\bm{b}}}
\def\vc{{\bm{c}}}
\def\vd{{\bm{d}}}
\def\ve{{\bm{e}}}
\def\vf{{\bm{f}}}
\def\vg{{\bm{g}}}
\def\vh{{\bm{h}}}
\def\vi{{\bm{i}}}
\def\vj{{\bm{j}}}
\def\vk{{\bm{k}}}
\def\vl{{\bm{l}}}
\def\vm{{\bm{m}}}
\def\vn{{\bm{n}}}
\def\vo{{\bm{o}}}
\def\vp{{\bm{p}}}
\def\vq{{\bm{q}}}
\def\vr{{\bm{r}}}
\def\vs{{\bm{s}}}
\def\vt{{\bm{t}}}
\def\vu{{\bm{u}}}
\def\vv{{\bm{v}}}
\def\vw{{\bm{w}}}
\def\vx{{\bm{x}}}
\def\vy{{\bm{y}}}
\def\vz{{\bm{z}}}

% Elements of vectors
\def\evalpha{{\alpha}}
\def\evbeta{{\beta}}
\def\evepsilon{{\epsilon}}
\def\evlambda{{\lambda}}
\def\evomega{{\omega}}
\def\evmu{{\mu}}
\def\evpsi{{\psi}}
\def\evsigma{{\sigma}}
\def\evtheta{{\theta}}
\def\eva{{a}}
\def\evb{{b}}
\def\evc{{c}}
\def\evd{{d}}
\def\eve{{e}}
\def\evf{{f}}
\def\evg{{g}}
\def\evh{{h}}
\def\evi{{i}}
\def\evj{{j}}
\def\evk{{k}}
\def\evl{{l}}
\def\evm{{m}}
\def\evn{{n}}
\def\evo{{o}}
\def\evp{{p}}
\def\evq{{q}}
\def\evr{{r}}
\def\evs{{s}}
\def\evt{{t}}
\def\evu{{u}}
\def\evv{{v}}
\def\evw{{w}}
\def\evx{{x}}
\def\evy{{y}}
\def\evz{{z}}

% Matrix
\def\mA{{\bm{A}}}
\def\mB{{\bm{B}}}
\def\mC{{\bm{C}}}
\def\mD{{\bm{D}}}
\def\mE{{\bm{E}}}
\def\mF{{\bm{F}}}
\def\mG{{\bm{G}}}
\def\mH{{\bm{H}}}
\def\mI{{\bm{I}}}
\def\mJ{{\bm{J}}}
\def\mK{{\bm{K}}}
\def\mL{{\bm{L}}}
\def\mM{{\bm{M}}}
\def\mN{{\bm{N}}}
\def\mO{{\bm{O}}}
\def\mP{{\bm{P}}}
\def\mQ{{\bm{Q}}}
\def\mR{{\bm{R}}}
\def\mS{{\bm{S}}}
\def\mT{{\bm{T}}}
\def\mU{{\bm{U}}}
\def\mV{{\bm{V}}}
\def\mW{{\bm{W}}}
\def\mX{{\bm{X}}}
\def\mY{{\bm{Y}}}
\def\mZ{{\bm{Z}}}
\def\mBeta{{\bm{\beta}}}
\def\mPhi{{\bm{\Phi}}}
\def\mLambda{{\bm{\Lambda}}}
\def\mSigma{{\bm{\Sigma}}}

% Tensor
\DeclareMathAlphabet{\mathsfit}{\encodingdefault}{\sfdefault}{m}{sl}
\SetMathAlphabet{\mathsfit}{bold}{\encodingdefault}{\sfdefault}{bx}{n}
\newcommand{\tens}[1]{\bm{\mathsfit{#1}}}
\def\tA{{\tens{A}}}
\def\tB{{\tens{B}}}
\def\tC{{\tens{C}}}
\def\tD{{\tens{D}}}
\def\tE{{\tens{E}}}
\def\tF{{\tens{F}}}
\def\tG{{\tens{G}}}
\def\tH{{\tens{H}}}
\def\tI{{\tens{I}}}
\def\tJ{{\tens{J}}}
\def\tK{{\tens{K}}}
\def\tL{{\tens{L}}}
\def\tM{{\tens{M}}}
\def\tN{{\tens{N}}}
\def\tO{{\tens{O}}}
\def\tP{{\tens{P}}}
\def\tQ{{\tens{Q}}}
\def\tR{{\tens{R}}}
\def\tS{{\tens{S}}}
\def\tT{{\tens{T}}}
\def\tU{{\tens{U}}}
\def\tV{{\tens{V}}}
\def\tW{{\tens{W}}}
\def\tX{{\tens{X}}}
\def\tY{{\tens{Y}}}
\def\tZ{{\tens{Z}}}


% Graph
\def\gA{{\mathcal{A}}}
\def\gB{{\mathcal{B}}}
\def\gC{{\mathcal{C}}}
\def\gD{{\mathcal{D}}}
\def\gE{{\mathcal{E}}}
\def\gF{{\mathcal{F}}}
\def\gG{{\mathcal{G}}}
\def\gH{{\mathcal{H}}}
\def\gI{{\mathcal{I}}}
\def\gJ{{\mathcal{J}}}
\def\gK{{\mathcal{K}}}
\def\gL{{\mathcal{L}}}
\def\gM{{\mathcal{M}}}
\def\gN{{\mathcal{N}}}
\def\gO{{\mathcal{O}}}
\def\gP{{\mathcal{P}}}
\def\gQ{{\mathcal{Q}}}
\def\gR{{\mathcal{R}}}
\def\gS{{\mathcal{S}}}
\def\gT{{\mathcal{T}}}
\def\gU{{\mathcal{U}}}
\def\gV{{\mathcal{V}}}
\def\gW{{\mathcal{W}}}
\def\gX{{\mathcal{X}}}
\def\gY{{\mathcal{Y}}}
\def\gZ{{\mathcal{Z}}}

% Sets
\def\sA{{\mathbb{A}}}
\def\sB{{\mathbb{B}}}
\def\sC{{\mathbb{C}}}
\def\sD{{\mathbb{D}}}
% Don't use a set called E, because this would be the same as our symbol
% for expectation.
\def\sF{{\mathbb{F}}}
\def\sG{{\mathbb{G}}}
\def\sH{{\mathbb{H}}}
\def\sI{{\mathbb{I}}}
\def\sJ{{\mathbb{J}}}
\def\sK{{\mathbb{K}}}
\def\sL{{\mathbb{L}}}
\def\sM{{\mathbb{M}}}
\def\sN{{\mathbb{N}}}
\def\sO{{\mathbb{O}}}
\def\sP{{\mathbb{P}}}
\def\sQ{{\mathbb{Q}}}
\def\sR{{\mathbb{R}}}
\def\sS{{\mathbb{S}}}
\def\sT{{\mathbb{T}}}
\def\sU{{\mathbb{U}}}
\def\sV{{\mathbb{V}}}
\def\sW{{\mathbb{W}}}
\def\sX{{\mathbb{X}}}
\def\sY{{\mathbb{Y}}}
\def\sZ{{\mathbb{Z}}}

% Entries of a matrix
\def\emLambda{{\Lambda}}
\def\emA{{A}}
\def\emB{{B}}
\def\emC{{C}}
\def\emD{{D}}
\def\emE{{E}}
\def\emF{{F}}
\def\emG{{G}}
\def\emH{{H}}
\def\emI{{I}}
\def\emJ{{J}}
\def\emK{{K}}
\def\emL{{L}}
\def\emM{{M}}
\def\emN{{N}}
\def\emO{{O}}
\def\emP{{P}}
\def\emQ{{Q}}
\def\emR{{R}}
\def\emS{{S}}
\def\emT{{T}}
\def\emU{{U}}
\def\emV{{V}}
\def\emW{{W}}
\def\emX{{X}}
\def\emY{{Y}}
\def\emZ{{Z}}
\def\emSigma{{\Sigma}}

% entries of a tensor
% Same font as tensor, without \bm wrapper
\newcommand{\etens}[1]{\mathsfit{#1}}
\def\etLambda{{\etens{\Lambda}}}
\def\etA{{\etens{A}}}
\def\etB{{\etens{B}}}
\def\etC{{\etens{C}}}
\def\etD{{\etens{D}}}
\def\etE{{\etens{E}}}
\def\etF{{\etens{F}}}
\def\etG{{\etens{G}}}
\def\etH{{\etens{H}}}
\def\etI{{\etens{I}}}
\def\etJ{{\etens{J}}}
\def\etK{{\etens{K}}}
\def\etL{{\etens{L}}}
\def\etM{{\etens{M}}}
\def\etN{{\etens{N}}}
\def\etO{{\etens{O}}}
\def\etP{{\etens{P}}}
\def\etQ{{\etens{Q}}}
\def\etR{{\etens{R}}}
\def\etS{{\etens{S}}}
\def\etT{{\etens{T}}}
\def\etU{{\etens{U}}}
\def\etV{{\etens{V}}}
\def\etW{{\etens{W}}}
\def\etX{{\etens{X}}}
\def\etY{{\etens{Y}}}
\def\etZ{{\etens{Z}}}

% The true underlying data generating distribution
\newcommand{\pdata}{p_{\rm{data}}}
% The empirical distribution defined by the training set
\newcommand{\ptrain}{\hat{p}_{\rm{data}}}
\newcommand{\Ptrain}{\hat{P}_{\rm{data}}}
% The model distribution
\newcommand{\pmodel}{p_{\rm{model}}}
\newcommand{\Pmodel}{P_{\rm{model}}}
\newcommand{\ptildemodel}{\tilde{p}_{\rm{model}}}
% Stochastic autoencoder distributions
\newcommand{\pencode}{p_{\rm{encoder}}}
\newcommand{\pdecode}{p_{\rm{decoder}}}
\newcommand{\precons}{p_{\rm{reconstruct}}}

\newcommand{\laplace}{\mathrm{Laplace}} % Laplace distribution

\newcommand{\E}{\mathbb{E}}
\newcommand{\Ls}{\mathcal{L}}
\newcommand{\R}{\mathbb{R}}
\newcommand{\emp}{\tilde{p}}
\newcommand{\lr}{\alpha}
\newcommand{\reg}{\lambda}
\newcommand{\rect}{\mathrm{rectifier}}
\newcommand{\softmax}{\mathrm{softmax}}
\newcommand{\sigmoid}{\sigma}
\newcommand{\softplus}{\zeta}
\newcommand{\KL}{D_{\mathrm{KL}}}
\newcommand{\Var}{\mathrm{Var}}
\newcommand{\standarderror}{\mathrm{SE}}
\newcommand{\Cov}{\mathrm{Cov}}
% Wolfram Mathworld says $L^2$ is for function spaces and $\ell^2$ is for vectors
% But then they seem to use $L^2$ for vectors throughout the site, and so does
% wikipedia.
\newcommand{\normlzero}{L^0}
\newcommand{\normlone}{L^1}
\newcommand{\normltwo}{L^2}
\newcommand{\normlp}{L^p}
\newcommand{\normmax}{L^\infty}

\newcommand{\parents}{Pa} % See usage in notation.tex. Chosen to match Daphne's book.

\DeclareMathOperator*{\argmax}{arg\,max}
\DeclareMathOperator*{\argmin}{arg\,min}

\DeclareMathOperator{\sign}{sign}
\DeclareMathOperator{\Tr}{Tr}
\let\ab\allowbreak

\usepackage{lineno}
\usepackage{subcaption}
\usepackage{tabularx}
\usepackage{cases}
\captionsetup{compatibility=false}
\usepackage{epstopdf}
\usepackage{placeins}
\usepackage{pgfplots}
\usepackage{tikz}
\usepackage{calc}
\usepackage{array}
%\usepackage[linesnumbered,ruled,vlined]{algorithm2e}
\usetikzlibrary{positioning, arrows.meta,calc}
\usepackage{newfloat}
\usepackage{listings}
\DeclareCaptionStyle{ruled}{labelfont=normalfont,labelsep=colon,strut=off} % DO NOT CHANGE THIS
\lstset{%
	basicstyle={\footnotesize\ttfamily},% footnotesize acceptable for monospace
	numbers=left,numberstyle=\footnotesize,xleftmargin=2em,% show line numbers, remove this entire line if you don't want the numbers.
	aboveskip=0pt,belowskip=0pt,%
	showstringspaces=false,tabsize=2,breaklines=true}
\floatstyle{ruled}
\newfloat{listing}{tb}{lst}{}
\floatname{listing}{Listing}
%
% Keep the \pdfinfo as shown here. There's no need
% for you to add the /Title and /Author tags.
\pdfinfo{
	/TemplateVersion (2026.1)
}

\title{Order Reduction and MILP Models for Real-Time DNN Robustness Certification \\ Supplementary Material}
\date{}
\author{
	%Authors
	% All authors must be in the same font size and format.
	Written by AAAI Press Staff\textsuperscript{\rm 1}\thanks{With help from the AAAI Publications Committee.}\\
	AAAI Style Contributions by Pater Patel Schneider,
	Sunil Issar,\\
	J. Scott Penberthy,
	George Ferguson,
	Hans Guesgen,
	Francisco Cruz\equalcontrib,
	Marc Pujol-Gonzalez\equalcontrib
}
\affiliations{
	%Afiliations
	\textsuperscript{\rm 1}Association for the Advancement of Artificial Intelligence\\
	% If you have multiple authors and multiple affiliations
	% use superscripts in text and roman font to identify them.
	% For example,
	
	% Sunil Issar\textsuperscript{\rm 2},
	% J. Scott Penberthy\textsuperscript{\rm 3},
	% George Ferguson\textsuperscript{\rm 4},
	% Hans Guesgen\textsuperscript{\rm 5}
	% Note that the comma should be placed after the superscript
	
	1101 Pennsylvania Ave, NW Suite 300\\
	Washington, DC 20004 USA\\
	% email address must be in roman text type, not monospace or sans serif
	proceedings-questions@aaai.org
	%
	% See more examples next
}

%Example, Single Author, ->> remove \iffalse,\fi and place them surrounding AAAI title to use it
\iffalse
\title{My Publication Title --- Single Author}
\author {
	Author Name
}
\affiliations{
	Affiliation\\
	Affiliation Line 2\\
	name@example.com
}
\fi

\iffalse
%Example, Multiple Authors, ->> remove \iffalse,\fi and place them surrounding AAAI title to use it
\title{My Publication Title --- Multiple Authors}
\author {
	% Authors
	First Author Name\textsuperscript{\rm 1},
	Second Author Name\textsuperscript{\rm 2},
	Third Author Name\textsuperscript{\rm 1}
}
\affiliations {
	% Affiliations
	\textsuperscript{\rm 1}Affiliation 1\\
	\textsuperscript{\rm 2}Affiliation 2\\
	firstAuthor@affiliation1.com, secondAuthor@affilation2.com, thirdAuthor@affiliation1.com
}
\fi


\newtheorem{proposition}{Proposition}
\newtheorem{definition}{Definition}
\newcommand{\vW}{\boldsymbol{W}}
\newcommand{\val}{{\textrm{value}}}
\newcommand{\Val}{{\textrm{value}}}
\newcommand{\MILP}{{\textrm{MILP}}}
\newcommand{\LP}{{\textrm{LP}}}
\newcommand{\Improve}{\mathrm{Improve}}
\newcommand{\Utility}{\mathrm{SAS}}
\newcommand{\Sol}{\mathrm{Sol}}
\newcommand{\sol}{\mathrm{sol}}
\newcommand{\UB}{\mathrm{UB}}
\newcommand{\LB}{\mathrm{LB}}
\newcommand{\ub}{\mathrm{ub}}
\newcommand{\lb}{\mathrm{lb}}
\newcommand{\B}{\mathrm{B}}
\usepackage{amsmath, amssymb, amsfonts}
\newcommand{\ReLU}{\mathrm{ReLU}}
\newcommand{\CMP}{{\textrm{CMP}}\ }
\newcommand{\fix}{\marginpar{FIX}}
\newcommand{\new}{\marginpar{NEW}}
\newcommand{\toolname}{Hybrid MILP}

% REMOVE THIS: bibentry
% This is only needed to show inline citations in the guidelines document. You should not need it and can safely delete it.
\usepackage{bibentry}
% END REMOVE bibentry

\begin{document}
	
\maketitle


\section{Reproducibility Checklist}

This paper:
\begin{itemize}
\item Includes a conceptual outline and/or pseudocode description of AI methods introduced {\bf yes}
\item Clearly delineates statements that are opinions, hypothesis, and speculation from objective facts and results {\bf yes}
\item Provides well marked pedagogical references for less-familiare readers to gain background necessary to replicate the paper {\bf yes}
\item Does this paper make theoretical contributions? {\bf yes}
\end{itemize}

if yes, then:

\begin{itemize}
\item All assumptions and restrictions are stated clearly and formally. {\bf yes}
\item All novel claims are stated formally (e.g., in theorem statements). {\bf yes}
\item Proofs of all novel claims are included. {\bf yes} (in the supplementary material)
\item Proof sketches or intuitions are given for complex and/or novel results. {\bf yes}
\item Appropriate citations to theoretical tools used are given. {\bf yes}
\item All theoretical claims are demonstrated empirically to hold. {\bf yes}
\item All experimental code used to eliminate or disprove claims is included. {\bf yes}
\end{itemize}




\section{Proof of Proposition 2}

Recall first that we use the classical MILP encoding 
for $\hat{x}'_i=\ReLU(x'_i)$, using one binary variable $a'$. We reuse 
$a'$ (with its value already settled, see Proposition 1 in the main paper) 
in the encoding of $\hat{y}_i = \ReLU(x_i)-\ReLU(x'_i)$:

\addtocounter{proposition}{1}
\addtocounter{table}{1}

\begin{proposition}
Assuming that $y_i \in [-\gamma_i, \gamma_i]$,
and that $x'_i \in [\alpha_i,\beta_i]$,
we have that $\hat{y}_i = \ReLU(x_i=x'_i+y_i)-\ReLU(x'_i)$ is the solution of:
	\begin{align*}
		& \begin{aligned}
			y_i + x'_i &\leq a\beta_i        &
			y_i &+ x'_i \geq (1-a)\alpha_i \\
			x'_i       &\leq a'\beta_i       & 
			x'_i       &\geq (1-a')\alpha_i \\
			\hat{y}_i  &\leq a\gamma_i       &
			\hat{y}_i  &\geq -a'\gamma_i \\
			\hat{y}_i  &\leq y_i + (1-a)\gamma_i  &
			\hat{y}_i  &\geq y_i - (1-a')\gamma_i \\
			\hat{y}_i  &\leq -x'_i + a\beta_i &
			\hat{y}_i  &\geq -x'_i + (1-a')\alpha_i \\
			\hat{y}_i  &\leq y_i + x'_i + (1-a)(-\alpha_i) &
			\hat{y}_i  &\geq y_i + x'_i + a'(-\beta_i)
		\end{aligned}
	\end{align*} 
    where $a,a' \in \{0,1\}$ are binary variables, 
    and $a'$ is shared with the classical MILP
    encoding for $\hat{x}'_i=\ReLU(x'_i)$.
\end{proposition}



	We do a case analysis depending on the value of both binary variables. 
First, we know the constraints for $\hat{x_i}'=\ReLU(x_i')$ are exact. 

We have 2 binary variables and 4 cases in total. We only need to check that, in all 4 cases, $$\hat{y}_i = \ReLU(x'_i+y_i)-\ReLU(x'_i).$$

\textbf{Case 1:} if $a = 1$ and $a' = 1$, then $x'_i \geq 0 $ and $y_i+x'_i\geq 0$, then we need to show $\hat{y}_i = y_i$ based on  $\hat{x'_i} = x'_i$. This is true by the two inequalities in line 4.

\textbf{Case 2:}  if $a = 1$ and $a' = 0$, then $x'_i \leq 0 $ and $y_i+x'_i\geq 0$, then we need to show $\hat{y}_i = y_i+x'_i$ based on  $\hat{x'_i} = 0$. This is true by the two inequalities in line 6.

\textbf{Case 3:} if $a = 0$ and $a' = 0$, then $x'_i \leq 0 $ and $y_i+x'_i\leq 0$, then we need to show $\hat{y}_i = 0$ based on  $\hat{x'_i} = 0$. This is true by the two inequalities in line 3.

\textbf{Case 4:} if $a = 0$ and $a' = 1$, then $x'_i \geq 0 $ and $y_i+x'_i\leq 0$, then we need to show $\hat{y}_i = -x'_i$ based on  $\hat{x'_i} = x'_i$. This is true by the two inequalities in line 5.

\section{Explanations on PCA model order reduction}

Some words on PCA, learn of 2 Matrix full-dim to reduced-dim and back.


Then how we use it (pictures and text).

	\section{Additional Experimental Evaluations}
	

\iffalse

\subsection{Classical vs our "2v" model vs ITNE}


\begin{table}[h!]
	\centering
	\begin{tabular}{||l|c|c|c||}\hline\hline
		model &        Bound $\downarrow$ &  Sol. &      Worst-Case $\uparrow$ \\\hline \hline
	Classical, $0 \times 2$ (LP)&  ? & ? & ?
    \\\hline
	ITNE, $0 \times 2$ (LP) &    ? & ? & ?
    \\\hline
	2v model, $0 \times 2$ (LP) &    ? & ? & ?
    \\\hline \hline
	
		Classical, $50 \times 2$ &    $.320$ &  $.320$ & $.017$ 
    \\\hline
	ITNE, $50 \times 2$ &    $.042$ &  $.037$ & $.022$
	\\ \hline
    2v model, $50 \times 2$ &    {\bf .040} &  $.037$ &  $.018$ 
    \\\hline \hline
    Classical, $100 \times 2$ &  .186  &  $.022$ & $.022$ 
    \\\hline
	ITNE, $100 \times 2$ &    $.045$ &  $.023$ & .023
    \\\hline
	2v model, $100 \times 2$&     {\bf .042} &  $.023$ &   .023 
    \\\hline \hline
	\end{tabular}
	\caption{Comparison of the classical encoding, ITNE, and our "2v" model on the pipe system 
	with a fixed timeout of 1000s, with either 0 (LP), $50 \times 2$, 
    or the full $100 \times 2$ binary variables.}
    \label{table.classical}
\end{table}

\fi




	\subsection{Experimental results for robustness (MNIST)}
	
	\begin{table}[h!]
		\centering
	\begin{tabular}{||l||c|c|c||}\hline\hline
		model &        Bound $\downarrow$ &  Sol. &      Worst-Case $\uparrow$ \\\hline \hline
		1v, $0 \times 1$ (LP) & 20.2612  & 20.2612  & .028 \\\hline 
		3v, $0 \times 3$ (LP) & {\bf 20.2614}  & 20.2614  & .125 \\\hline 
	    2v, $0 \times 2$ (LP) & {\bf 20.2614}  & 20.2614  & {\bf .145} \\\hline\hline	 

	   %1v, $100 \times 1$ & 17.70 & 17.67 & .004 \\\hline\hline	 

		1v, $300 \times 1$ & {\bf 15.19} & 11.68 & .093 \\\hline 
		3v, $300 \times 3$ & 16.57 & 11.43 & .200 \\\hline 
	    2v, $300 \times 2$ & 16.65 & 10.05 & {\bf .331} \\\hline\hline	 

		1v, $400 \times 1$ & {\bf 14.87} & 6.456 & .032 \\\hline 
		3v, $400 \times 3$ & 16.62 & 6.343 & .254 \\\hline 
	    2v, $400 \times 2$ & 16.33 & 5.777 & {\bf .371} \\\hline \hline

		1v, $500 \times 1$ & {\bf 14.97} & $.845$ & $.009$ \\\hline 
		3v, $500 \times 3$ & $17.66$ & $.813$ & {\bf .518} \\\hline 
	    2v, $500 \times 2$ & $16.49$ & n/a & n/a \\\hline\hline	 
	\end{tabular}
	\caption{Bounds on $\beta^{.5}_{6,8}$ 
	obtained by the "1v", "3v" and "2v" models 
	on the {\bf full dimension} MNIST DNN, 
	for timeouts of $14400$s, when 0, 300, 400 or 500 ($\times 1$, $\times 2$, $\times 3$) variables are binary.}
	\label{table.mnist}
\end{table}

In Table \ref{table.mnist}, we further tested different number of reduced variables. 
With 0 binary variables (full LP), the "1v" model converges in 20 seconds, and the "2v" and "3v" models in 200s.
Opening $400 (\times 1,2,3)$ variables shows a very slight improvement in the bound over 500 variables, 
far insufficient to prove that any image is robust in real-time. As expected, the lower the number of variables, the closer the MILP solver converges with the same time-out.




\subsection{Reduced Space}

\begin{table}[h!]
	\centering
	\begin{tabular}{||l||c|c|c||}\hline\hline
		model &        Bound$\downarrow$ &  Sol. &      Worst-Case$\uparrow$ \\\hline \hline
1v, $400 \times 1$ & $1.414$ &  $.691$ & {\bf .010} \\\hline 
3v, $400 \times 3$ & {\bf 1.186} & $.600$ & $.003$ \\\hline 
2v, $400 \times 2$ & $1.274$ & $.566$ & $.002$ \\\hline\hline
	 
1v, $475 \times 1$ &  $1.408$ & $.301$ & $.008$  \\\hline 
3v, $475 \times 3$ &  {\bf 1.153} & $.250$ & $.006$ \\ \hline 
2v, $475 \times 2$ &  $1.247$ & $.196$ & {\bf .019} \\\hline\hline

1v, $500 \times 1$ & $1.412$ & $.161$ & .057 \\\hline 
3v, $500 \times 3$ & {\bf 1.137} & $.103$ & $.065$\\\hline 
2v, $500 \times 2$ &  $1.182$ & $.084$& {\bf .084}  \\\hline\hline
	 
	\end{tabular}
	\caption{Comparison of "1v", "3v" and "2v" models 
	to obtain bounds on $\beta^{.5}_{6,8}$ on the {\bf 20 dimension} reduced order MNIST DNN, for timeout of 14400s, 
	where 400, 475,  or 500 ($\times 1$, $\times 2$, $\times 3$) neurons use binary variables.}
	\label{table.reduced}
\end{table}


We also experimented reducing the number of binary variables (by considering some with linear relaxation) from 500 ($\times 1,2,3$) to 475 ($\times 1,2,3$) and 400 ($\times 1,2,3$) for the PCA reduced space to 20 dimensions (Table~\ref{table.reduced}). 
As for Table~\ref{table.mnist}, it helps the MILP solver converge towards closer (bound, solution) pairs. However, the bounds found are not better than with all variables as binary variables, and this does not improve the percentage of images certified robust.


Notice that the MILP process is far from converging (large difference between the Bound and the Solution) after 4 hours (14400s), and we can use a much longer 
runtime to obtain better bounds. We did experiment the "3v" model with an extended time-out at 129600s, which does improve the bounds by around $10\%$, as well as the number of images certified robust in real-time (e.g. $65\%$ for $L_1 \leq 1$), Table~\ref{table.cert}.

Last, we report in Table~\ref{table.pair} the bounds $\beta^{.5}_{C,D}$ obtained for every pair of class $C < D \in \{0, \ldots, 9\}$ using all $500 (\times 1,2,3) $ binary variables, which we could not provide in the main file for space issue - $\beta^{.5}_{6,8}$ being a good representative, with the timeout of 14400s.


\begin{table}[t!]
	\begin{tabular}{||l||c|c|c|c||}\hline\hline
		model &    $L_1\leq 0.5$ & $L_1\leq 1$ & $L_1\leq 1.5$ &  $L_1\leq 2$ \\\hline \hline
		1v, $500\times1$ & $80 \%$ & $32\%$ & $7\%$ & $0\%$ \\\hline
		3v, $500 \times 3$ & {\bf 86\%} & {\bf 53\%} & {\bf 20\%} & {\bf 4\%} \\\hline
		2v, $500 \times 2$ & 84\% & 51\% & 16\% & {\bf 4\%} \\\hline \hline
		3vEx, $500 \times 3$ & {\bf 89\%} & {\bf 69\%} & {\bf 35\%} & {\bf 16\%} \\\hline\hline
	\end{tabular}
	\caption{Percentage of images certified robust in real-time 
	using the computed $(\beta^{.5}_{i,j})_{i < j \leq 10}$ 
	by the "1v", "3v" and "2v" models, for different values of $L_1$-perturbations, 
	as wel as for "3v"Ex with the extended time out of 129600s.}
    \label{table.cert}
\end{table}







%\vspace{1cm}
	
	

\subsection{Experimental Results for Regression (Pipe Strain)}

	\begin{table}[b!]
	\begin{tabular}{||l||c|c|c|c||}\hline\hline
		model &        Bound$\downarrow$ &  Sol. &      Worst-Case$\uparrow$ &  Time(s) \\\hline \hline
		1v, $100 \times 1$ &     {\bf .0356} &  $.0356$ & $.0191$ &  1000 \\\hline
		3v, $100 \times 3$&     .0414 &  .0254 &  .0166 &  1000 \\\hline
		2v, $100 \times 2$&     .0418 &  .0229 &   {\bf .0229} &  1000 \\\hline \hline
		
	    3v, $90 \times 3$&     .0468 &  .0468 &  .0181 & 14440 \\\hline	
		3v, $97 \times 3$&     .0352 &  .0284 &  .0196 & 14440 \\\hline
		3v, $100 \times 3$&      {\bf .0350} &  .0272 &  .0216 & 14440 \\\hline

		2v, $90 \times 2$&     .0472 &  .0472 &  .0195 & 14440 \\\hline	
		2v, $97 \times 2$&     .0393 &  .0240 &   {\bf .0237} & 14440 \\\hline
		2v, $100 \times 2$&     .0360 &  .0236 &   .0236 & 14440 \\\hline \hline

		3v, $100 \times 3$&     {\bf .0329} &  .0277 &  .0165 & 72000 \\\hline
		2v, $100 \times 2$&     .0337 &  .0245 &  {\bf .0245} & 72000 \\\hline\hline
	\end{tabular}
	\caption{Comparison of "1v", "3v" and "2v" models on the pipe system with timeouts of 1000s, 14440s and 72000s, where 90, 97 and 100 ($\times 1, \times 2,\times 3$) binary variables.}
	\label{table.pipe}
\end{table}



Last, for the pipe system in Table \ref{table.pipe}, we report for the "standard" 14400s time-out we use in other benchmarks, additional experiments with limited number of binary variables, namely $97 (\times 2,3)$ and $90 (\times 2,3)$ --- there is no point to consider the 1v model with reduced number of binary variable as it has already converged after 1000s with all the variables binary. Again, lowering the number of binary variables helps the MILP solver converge closer (it did converge within 2000s for 90 variables), but the bounds found are worse than with all the variables binary.



	
	\iffalse
	\begin{table}[h!]
	\begin{tabular}{|l|l|l|l|l|}\hline
		$L_1\leq 0.83$ &        Bound $\downarrow$ &  Solution $\uparrow$ &      Real $\uparrow$ &  Time \\\hline
		1v,open 100 &     {\bf 0.035613} &  0.035613 &                       0.01288 & 10608 \\\hline
		3v,open 100 &     0.040074 &  0.028934 &                      0.021441 & 10922 \\\hline
		%3v,open 100 &     0.039824 &  0.028832 &                      0.022255 & 22153 \\\hline
		2v,open 100 &     0.046719 &  0.024364 &  {\bf 0.024436} & 10922 \\\hline
	\end{tabular}
	\caption{Comparison of 1v,2v and 3v models on the pipe system with a fixed timeout of 10.000s.}
\end{table}
\fi
	
		
%\addtocounter{table}{1}





\begin{table*}
	\centering
	\begin{tabular}{|r||c|c|c|c|c|c|c|c|c|c|}\hline
		{\bf "1v"} & 0 & 1 & 2 & 3 & 4 & 5 & 6 & 7 & 8 & 9 \\\hline\hline
0 & &1.51 &1.21 &1.45 &1.45 &1.46 &1.31 &1.47 &1.44 &1.14  \\\hline
1 &1.51 & &1.44 &1.38 &1.16 &1.29 &1.37 &1.22 &1.51 &1.29  \\\hline
2 &1.21 &1.44 & &1.42 &1.60 &1.69 &1.53 &1.32 &1.52 &1.34  \\\hline
3 &1.45 &1.38 &1.42 & &1.58 &1.41 &1.64 &1.42 &1.52 &1.34  \\\hline
4 &1.45 &1.16 &1.60 &1.58 & &1.53 &1.38 &1.41 &1.64 &1.10  \\\hline
5 &1.46 &1.29 &1.69 &1.41 &1.53 & &1.16 &1.53 &1.61 &1.13  \\\hline
6 &1.31 &1.37 &1.53 &1.64 &1.38 &1.16 & &1.67 &1.41 &1.49  \\\hline
7 &1.47 &1.22 &1.32 &1.42 &1.41 &1.53 &1.67 & &1.67 &1.32  \\\hline
8 &1.44 &1.51 &1.52 &1.52 &1.64 &1.61 &1.41 &1.67 & &1.36  \\\hline
9 &1.14 &1.29 &1.34 &1.34 &1.10 &1.13 &1.49 &1.32 &1.36 &  \\\hline
	\end{tabular}

	\vspace{0.3cm}
	
	\begin{tabular}{|r||c|c|c|c|c|c|c|c|c|c|}\hline
		{\bf "3v"} & 0 & 1 & 2 & 3 & 4 & 5 & 6 & 7 & 8 & 9 \\\hline\hline
0 & &1.20 &0.97 &1.23 &1.16 &1.24 &1.02 &1.26 &1.17 &0.91  \\\hline
1 &1.20 & &1.18 &1.09 &0.93 &1.11 &1.16 &0.99 &1.25 &1.10  \\\hline
2 &0.97 &1.18 & &1.17 &1.30 &1.40 &1.35 &1.09 &1.26 &1.11  \\\hline
3 &1.23 &1.09 &1.17 & &1.35 &1.15 &1.37 &1.16 &1.24 &1.06  \\\hline
4 &1.16 &0.93 &1.30 &1.35 & &1.31 &1.12 &1.16 &1.36 &0.83  \\\hline
5 &1.24 &1.11 &1.40 &1.15 &1.31 & &0.94 &1.31 &1.23 &0.91  \\\hline
6 &1.02 &1.16 &1.35 &1.37 &1.12 &0.94 & &1.43 &1.14 &1.23  \\\hline
7 &1.26 &0.99 &1.09 &1.16 &1.16 &1.31 &1.43 & &1.34 &1.07  \\\hline
8 &1.17 &1.25 &1.26 &1.24 &1.36 &1.23 &1.14 &1.34 & &1.10  \\\hline
9 &0.91 &1.10 &1.11 &1.06 &0.83 &0.91 &1.23 &1.07 &1.10 &  \\\hline
	\end{tabular}
	
	
\iffalse	
		\begin{tabular}{|r||c|c|c|c|c|c|c|c|c|c|}\hline
		{\bf "3v"} & 0 & 1 & 2 & 3 & 4 & 5 & 6 & 7 & 8 & 9 \\\hline\hline
		0 & &2.02 &1.53 &2.09 &1.93 &2.10 &1.67 &1.98 &1.81 &1.48  \\\hline
	1 &2.02 & &1.82 &1.69 &1.53 &1.83 &1.90 &1.50 &2.04 &1.79  \\\hline
	2 &1.53 &1.82 & &1.90 &2.22 &2.42 &2.25 &1.61 &2.07 &1.73  \\\hline
	3 &2.09 &1.69 &1.90 & &2.18 &1.79 &2.22 &1.72 &1.82 &1.71  \\\hline
	4 &1.93 &1.53 &2.22 &2.18 & &2.20 &1.89 &1.90 &2.24 &1.46  \\\hline
	5 &2.10 &1.83 &2.42 &1.79 &2.20 & &1.64 &2.23 &2.07 &1.51  \\\hline
	6 &1.67 &1.90 &2.25 &2.22 &1.89 &1.64 & &2.43 &1.95 &2.05  \\\hline
	7 &1.98 &1.50 &1.61 &1.72 &1.90 &2.23 &2.43 & &2.34 &1.75  \\\hline
	8 &1.81 &2.04 &2.07 &1.82 &2.24 &2.07 &1.95 &2.34 & &1.72  \\\hline
	9 &1.48 &1.79 &1.73 &1.71 &1.46 &1.51 &2.05 &1.75 &1.72 &  \\\hline
	\end{tabular}
\fi

	
	
	
	
	
\vspace{0.6cm}

	\begin{tabular}{|r||c|c|c|c|c|c|c|c|c|c|}\hline
		{\bf "2v"} & 0 & 1 & 2 & 3 & 4 & 5 & 6 & 7 & 8 & 9 \\\hline\hline
	0 & &1.24 &0.99 &1.23 &1.24 &1.31 &1.02 &1.33 &1.17 &1.04  \\\hline
	1 &1.24 & &1.20 &1.09 &0.92 &1.13 &1.15 &0.97 &1.27 &1.16  \\\hline
	2 &0.99 &1.20 & &1.20 &1.29 &1.50 &1.46 &1.08 &1.30 &1.10  \\\hline
	3 &1.23 &1.09 &1.20 & &1.41 &1.21 &1.35 &1.19 &1.26 &1.05  \\\hline
	4 &1.24 &0.92 &1.29 &1.41 & &1.37 &1.23 &1.31 &1.50 &0.92  \\\hline
	5 &1.31 &1.13 &1.50 &1.21 &1.37 & &1.03 &1.40 &1.40 &0.98  \\\hline
	6 &1.02 &1.15 &1.46 &1.35 &1.23 &1.03 & &1.56 &1.18 &1.37  \\\hline
	7 &1.33 &0.97 &1.08 &1.19 &1.31 &1.40 &1.56 & &1.44 &1.15  \\\hline
	8 &1.17 &1.27 &1.30 &1.26 &1.50 &1.40 &1.18 &1.44 & &1.24  \\\hline
	9 &1.04 &1.16 &1.10 &1.05 &0.92 &0.98 &1.37 &1.15 &1.24 &  \\\hline
	\end{tabular}
	
	
	
	
	

	
\vspace{0.6cm}

\begin{tabular}{|r||c|c|c|c|c|c|c|c|c|c|}\hline
		{\bf "3v" extended TO} & 0 & 1 & 2 & 3 & 4 & 5 & 6 & 7 & 8 & 9 \\\hline\hline
		0 & &1.08 &0.85 &1.13 &1.06 &1.12 &0.90 &1.15 &1.03 &0.79  \\\hline
		1 &1.08 & &1.01 &0.99 &0.84 &1.02 &1.09 &0.87 &1.14 &0.99  \\\hline
		2 &0.85 &1.01 & &1.01 &1.16 &1.31 &1.26 &0.92 &1.14 &0.96  \\\hline
		3 &1.13 &0.99 &1.01 & &1.26 &1.03 &1.27 &0.99 &1.12 &0.95  \\\hline
		4 &1.06 &0.84 &1.16 &1.26 & &1.19 &1.01 &1.05 &1.25 &0.73  \\\hline
		5 &1.12 &1.02 &1.31 &1.03 &1.19 & &0.86 &1.17 &1.15 &0.84  \\\hline
		6 &0.90 &1.09 &1.26 &1.27 &1.01 &0.86 & &1.31 &1.04 &1.13  \\\hline
		7 &1.15 &0.87 &0.92 &0.99 &1.05 &1.17 &1.31 & &1.26 &0.97  \\\hline
		8 &1.03 &1.14 &1.14 &1.12 &1.25 &1.15 &1.04 &1.26 & &1.01  \\\hline
		9 &0.79 &0.99 &0.96 &0.95 &0.73 &0.84 &1.13 &0.97 &1.01 &  \\\hline
		
	\end{tabular}
	


	\caption{Bounds $(\beta^{.5}_{C,D})_{C < D \leq 9}$ 
	for MNIST on {\bf reduced 20 dimensions}, as reached by the "1v", "3v" and "2v" models
	with 500 $(\times 1,2,3)$ binary variables within 14400s timeout, as well as for "3v" with an extended time out at 
	129600s.}
    \label{table.pair}
\end{table*}


\begin{table}
	
	Open 500 original case:
		
	\vspace{0.6cm}
	
	
	\begin{tabular}{|r||c|c|c|c|c|c|c|c|c|c|}\hline
		{\bf "2v"} & 0 & 1 & 2 & 3 & 4 & 5 & 6 & 7 & 8 & 9 \\\hline\hline
0 & &1.24|\ \ \ \ \  &0.99|\ \ \ \ \  &1.23|\ \ \ \ \  &1.24|\ \ \ \ \  &1.31|\ \ \ \ \  &1.02|\ \ \ \ \  &1.33|\ \ \ \ \  &1.17|\ \ \ \ \  &1.04|\ \ \ \ \   \\\hline
1 &1.24|\ \ \ \ \  & &1.20|\ \ \ \ \  &1.09|\ \ \ \ \  &0.92|\ \ \ \ \  &1.13|\ \ \ \ \  &1.15|\ \ \ \ \  &0.97|\ \ \ \ \  &1.27|\ \ \ \ \  &1.16|\ \ \ \ \   \\\hline
2 &0.99|\ \ \ \ \  &1.20|\ \ \ \ \  & &1.20|\ \ \ \ \  &1.29|\ \ \ \ \  &1.50|\ \ \ \ \  &1.46|\ \ \ \ \  &1.08|\ \ \ \ \  &1.30|\ \ \ \ \  &1.10|\ \ \ \ \   \\\hline
3 &1.23|\ \ \ \ \  &1.09|\ \ \ \ \  &1.20|\ \ \ \ \  & &1.41|\ \ \ \ \  &1.21|\ \ \ \ \  &1.35|\ \ \ \ \  &1.19|\ \ \ \ \  &1.26|\ \ \ \ \  &1.05|\ \ \ \ \   \\\hline
4 &1.24|\ \ \ \ \  &0.92|\ \ \ \ \  &1.29|\ \ \ \ \  &1.41|\ \ \ \ \  & &1.37|\ \ \ \ \  &1.23|\ \ \ \ \  &1.31|\ \ \ \ \  &1.50|\ \ \ \ \  &0.92|\ \ \ \ \   \\\hline
5 &1.31|\ \ \ \ \  &1.13|\ \ \ \ \  &1.50|\ \ \ \ \  &1.21|\ \ \ \ \  &1.37|\ \ \ \ \  & &1.03|\ \ \ \ \  &1.40|\ \ \ \ \  &1.40|\ \ \ \ \  &0.98|\ \ \ \ \   \\\hline
6 &1.02|\ \ \ \ \  &1.15|\ \ \ \ \  &1.46|\ \ \ \ \  &1.35|\ \ \ \ \  &1.23|\ \ \ \ \  &1.03|\ \ \ \ \  & &1.56|\ \ \ \ \  &1.18|0.08 &1.37|\ \ \ \ \   \\\hline
7 &1.33|\ \ \ \ \  &0.97|\ \ \ \ \  &1.08|\ \ \ \ \  &1.19|\ \ \ \ \  &1.31|\ \ \ \ \  &1.40|\ \ \ \ \  &1.56|\ \ \ \ \  & &1.44|\ \ \ \ \  &1.15|\ \ \ \ \   \\\hline
8 &1.17|\ \ \ \ \  &1.27|\ \ \ \ \  &1.30|\ \ \ \ \  &1.26|\ \ \ \ \  &1.50|\ \ \ \ \  &1.40|\ \ \ \ \  &1.18|0.08 &1.44|\ \ \ \ \  & &1.24|\ \ \ \ \   \\\hline
9 &1.04|\ \ \ \ \  &1.16|\ \ \ \ \  &1.10|\ \ \ \ \  &1.05|\ \ \ \ \  &0.92|\ \ \ \ \  &0.98|\ \ \ \ \  &1.37|\ \ \ \ \  &1.15|\ \ \ \ \  &1.24|\ \ \ \ \  &  \\\hline
	\end{tabular}
	
		\vspace{0.6cm}
	
		Open 500 without 0 improvement case:
	
	\vspace{0.6cm}
	
	
	\begin{tabular}{|r||c|c|c|c|c|c|c|c|c|c|}\hline
		{\bf "2v"} & 0 & 1 & 2 & 3 & 4 & 5 & 6 & 7 & 8 & 9 \\\hline\hline
0 & &1.26|0.26 &0.96|0.29 &1.24|0.38 &1.22|0.30 &1.36|0.36 &1.10|0.21 &1.29|0.27 &1.18|0.20 &0.92|0.20  \\\hline
1 &1.26|0.26 & &1.13|0.25 & & & & & & &  \\\hline
2 &0.96|0.29 &1.13|0.25 & & & & & & & &  \\\hline
3 &1.24|0.38 & & & & & & & & &  \\\hline
4 &1.22|0.30 & & & & & & & & &  \\\hline
5 &1.36|0.36 & & & & & &0.99|0.26 &1.46|0.31 &1.40|0.16 &1.04|0.27  \\\hline
6 &1.10|0.21 & & & & &0.99|0.26 & &1.55|0.53 &1.38|0.28 &1.32|0.37  \\\hline
7 &1.29|0.27 & & & & &1.46|0.31 &1.55|0.53 & &1.50|0.39 &1.20|0.29  \\\hline
8 &1.18|0.20 & & & & &1.40|0.16 &1.38|0.28 &1.50|0.39 & &1.10|0.23  \\\hline
9 &0.92|0.20 & & & & &1.04|0.27 &1.32|0.37 &1.20|0.29 &1.10|0.23 &  \\\hline
	\end{tabular}
	
	\vspace{0.6cm}
	
	Open 300 case 1:
	
	\vspace{0.6cm}
	
	\begin{tabular}{|r||c|c|c|c|c|c|c|c|c|c|}\hline
		{\bf "2v"} & 0 & 1 & 2 & 3 & 4 & 5 & 6 & 7 & 8 & 9 \\\hline\hline
	0 & &1.22|0.65 &0.98|0.50 &1.23|0.63 &1.21|0.58 &1.25|0.64 &1.04|0.48 &1.26|0.63 &1.12|0.47 &0.91|0.41  \\\hline
	1 &1.22|0.65 & &1.10|0.55 & & & & & & &  \\\hline
	2 &0.98|0.50 &1.10|0.55 & & & & & & & &  \\\hline
	3 &1.23|0.63 & & & & & & & & &  \\\hline
	4 &1.21|0.58 & & & & & & & & &  \\\hline
	5 &1.25|0.64 & & & & & &1.01|0.47 &1.38|0.68 &1.30|0.53 &0.98|0.48  \\\hline
	6 &1.04|0.48 & & & & &1.01|0.47 & &1.54|0.83 &1.17|0.55 &1.29|0.68  \\\hline
	7 &1.26|0.63 & & & & &1.38|0.68 &1.54|0.83 & &1.42|0.70 &1.11|0.51  \\\hline
	8 &1.12|0.47 & & & & &1.30|0.53 &1.17|0.55 &1.42|0.70 & &1.11|0.51  \\\hline
	9 &0.91|0.41 & & & & &0.98|0.48 &1.29|0.68 &1.11|0.51 &1.11|0.51 &  \\\hline
	\end{tabular}
	
	
		\vspace{0.6cm}
	
	Open 300 case 2:
	
	\vspace{0.6cm}
	
	\begin{tabular}{|r||c|c|c|c|c|c|c|c|c|c|}\hline
		{\bf "2v"} & 0 & 1 & 2 & 3 & 4 & 5 & 6 & 7 & 8 & 9 \\\hline\hline
0 & &1.27|0.64 &1.02|0.50 &1.29|0.62 &1.23|0.57 &1.30|0.64 &1.03|0.48 &1.29|0.63 &1.17|0.46 &0.93|0.40  \\\hline
1 &1.27|0.64 & &1.19|0.54 & & & & & & &  \\\hline
2 &1.02|0.50 &1.19|0.54 & & & & & & & &  \\\hline
3 &1.29|0.62 & & & & & & & & &  \\\hline
4 &1.23|0.57 & & & & & & & & &  \\\hline
5 &1.30|0.64 & & & & & &1.00|0.47 &1.35|0.68 &1.33|0.53 &0.95|0.48  \\\hline
6 &1.03|0.48 & & & & &1.00|0.47 & &1.46|0.83 &1.16|0.56 &1.27|0.68  \\\hline
7 &1.29|0.63 & & & & &1.35|0.68 &1.46|0.83 & &1.43|0.70 &1.10|0.52  \\\hline
8 &1.17|0.46 & & & & &1.33|0.53 &1.16|0.56 &1.43|0.70 & &1.11|0.51  \\\hline
9 &0.93|0.40 & & & & &0.95|0.48 &1.27|0.68 &1.10|0.52 &1.11|0.51 &  \\\hline
	\end{tabular}
	
			\vspace{0.6cm}
\end{table}




\clearpage



\begin{table}

	\centering
	
		Open 300 new case: average = 1.2094235122780839
	

	
	\begin{tabular}{|r||c|c|c|c|c|c|c|c|c|c|}\hline
		{\bf "2v"} & 0 & 1 & 2 & 3 & 4 & 5 & 6 & 7 & 8 & 9 \\\hline\hline
		0 & &1.28|0.97 &1.01|0.76 &1.26|0.91 &1.20|0.88 &1.28|0.95 &1.06|0.75 &1.28|0.98 &1.18|0.80 &0.96|0.67  \\\hline
		1 &1.28|0.97 & &1.15|0.82 & & & & & & &  \\\hline
		2 &1.01|0.76 &1.15|0.82 & & & & & & & &  \\\hline
		3 &1.26|0.91 & & & & & & & & &  \\\hline
		4 &1.20|0.88 & & & & & & & & &  \\\hline
		5 &1.28|0.95 & & & & & &1.03|0.74 &1.37|1.10 &1.35|0.92 &0.99|0.72  \\\hline
		6 &1.06|0.75 & & & & &1.03|0.74 & &1.52|1.21 &1.19|0.86 &1.33|1.00  \\\hline
		7 &1.28|0.98 & & & & &1.37|1.10 &1.52|1.21 & &1.49|1.16 &1.13|0.79  \\\hline
		8 &1.18|0.80 & & & & &1.35|0.92 &1.19|0.86 &1.49|1.16 & &1.14|0.78  \\\hline
		9 &0.96|0.67 & & & & &0.99|0.72 &1.33|1.00 &1.13|0.79 &1.14|0.78 &  \\\hline
	\end{tabular}
	
	\vspace{0.6cm}
	
	Open min value case 1, $imp\leq 2\times10^{-4}$: average = 1.1781991237459444
	
	\vspace{0.6cm}
	
		\begin{tabular}{|r||c|c|c|c|c|c|c|c|c|c|}\hline
		{\bf "2v"} & 0 & 1 & 2 & 3 & 4 & 5 & 6 & 7 & 8 & 9 \\\hline\hline
	0 & &1.21|0.97 &0.96|0.76 &1.23|0.91 &1.21|0.88 &1.25|0.95 &1.04|0.75 &1.22|0.98 &1.14|0.80 &0.96|0.67  \\\hline
	1 &1.21|0.97 & &1.14|0.82 & & & & & & &  \\\hline
	2 &0.96|0.76 &1.14|0.82 & & & & & & & &  \\\hline
	3 &1.23|0.91 & & & & & & & & &  \\\hline
	4 &1.21|0.88 & & & & & & & & &  \\\hline
	5 &1.25|0.95 & & & & & &1.03|0.74 &1.32|1.10 &1.33|0.92 &0.96|0.72  \\\hline
	6 &1.04|0.75 & & & & &1.03|0.74 & &1.50|1.21 &1.16|0.86 &1.27|1.00  \\\hline
	7 &1.22|0.98 & & & & &1.32|1.10 &1.50|1.21 & &1.42|1.16 &1.10|0.79  \\\hline
	8 &1.14|0.80 & & & & &1.33|0.92 &1.16|0.86 &1.42|1.16 & &1.12|0.78  \\\hline
	9 &0.96|0.67 & & & & &0.96|0.72 &1.27|1.00 &1.10|0.79 &1.12|0.78 &  \\\hline
	\end{tabular}
	
		\vspace{0.6cm}
	
		Open min value case 2, $imp\leq10^{-4}$: average = 1.1636021206194367
	
	\vspace{0.6cm}
	
	\begin{tabular}{|r||c|c|c|c|c|c|c|c|c|c|}\hline
		{\bf "2v"} & 0 & 1 & 2 & 3 & 4 & 5 & 6 & 7 & 8 & 9 \\\hline\hline
	0 & &1.19|0.97 &1.00|0.76 &1.23|0.91 &1.19|0.88 &1.23|0.95 &1.00|0.75 &1.18|0.98 &1.10|0.80 &0.93|0.67  \\\hline
	1 &1.19|0.97 & &1.13|0.82 & & & & & & &  \\\hline
	2 &1.00|0.76 &1.13|0.82 & & & & & & & &  \\\hline
	3 &1.23|0.91 & & & & & & & & &  \\\hline
	4 &1.19|0.88 & & & & & & & & &  \\\hline
	5 &1.23|0.95 & & & & & &0.99|0.74 &1.35|1.10 &1.23|0.92 &0.98|0.72  \\\hline
	6 &1.00|0.75 & & & & &0.99|0.74 & &1.54|1.21 &1.16|0.86 &1.27|1.00  \\\hline
	7 &1.18|0.98 & & & & &1.35|1.10 &1.54|1.21 & &1.42|1.16 &1.08|0.79  \\\hline
	8 &1.10|0.80 & & & & &1.23|0.92 &1.16|0.86 &1.42|1.16 & &1.09|0.78  \\\hline
	9 &0.93|0.67 & & & & &0.98|0.72 &1.27|1.00 &1.08|0.79 &1.09|0.78 &  \\\hline
	\end{tabular}
\end{table}

\clearpage


\begin{table}
		\centering
	
	Open 300 with 75x2+150: average = 1.1890592409097878
	
	
	
	\begin{tabular}{|r||c|c|c|c|c|c|c|c|c|c|}\hline
		{\bf "2v"} & 0 & 1 & 2 & 3 & 4 & 5 & 6 & 7 & 8 & 9 \\\hline\hline
		0 & &1.24|0.81 &0.98|0.61 &1.25|0.77 &1.20|0.74 &1.25|0.79 &1.01|0.62 &1.22|0.80 &1.17|0.64 &0.93|0.56  \\\hline
		1 &1.24|0.81 & &1.17|0.66 & & & & & & &  \\\hline
		2 &0.98|0.61 &1.17|0.66 & & & & & & & &  \\\hline
		3 &1.25|0.77 & & & & & & & & &  \\\hline
		4 &1.20|0.74 & & & & & & & & &  \\\hline
		5 &1.25|0.79 & & & & & &1.02|0.58 &1.36|0.87 &1.32|0.71 &0.96|0.58  \\\hline
		6 &1.01|0.62 & & & & &1.02|0.58 & &1.53|1.01 &1.18|0.67 &1.30|0.84  \\\hline
		7 &1.22|0.80 & & & & &1.36|0.87 &1.53|1.01 & &1.45|0.85 &1.13|0.64  \\\hline
		8 &1.17|0.64 & & & & &1.32|0.71 &1.18|0.67 &1.45|0.85 & &1.13|0.63  \\\hline
		9 &0.93|0.56 & & & & &0.96|0.58 &1.30|0.84 &1.13|0.64 &1.13|0.63 &  \\\hline
	\end{tabular}
	
	\vspace{0.6cm}
	
	
	Open 600 with 150x2+300, average = 1.2063095408033964.
	
		\begin{tabular}{|r||c|c|c|c|c|c|c|c|c|c|}\hline
		{\bf "2v"} & 0 & 1 & 2 & 3 & 4 & 5 & 6 & 7 & 8 & 9 \\\hline\hline
		0 & &1.23|0.30 &0.95|0.26 &1.29|0.33 &1.16|0.21 &1.25|0.25 &1.02|0.17 &1.21|0.25 &1.23|0.17 &0.90|0.15  \\\hline
	1 &1.23|0.30 & &1.19|0.18 & & & & & & &  \\\hline
	2 &0.95|0.26 &1.19|0.18 & & & & & & & &  \\\hline
	3 &1.29|0.33 & & & & & & & & &  \\\hline
	4 &1.16|0.21 & & & & & & & & &  \\\hline
	5 &1.25|0.25 & & & & & &1.00|0.16 &1.44|0.23 &1.33|0.14 &0.97|0.17  \\\hline
	6 &1.02|0.17 & & & & &1.00|0.16 & &1.57|0.44 &1.26|0.19 &1.35|0.24  \\\hline
	7 &1.21|0.25 & & & & &1.44|0.23 &1.57|0.44 & &1.53|0.25 &1.13|0.21  \\\hline
	8 &1.23|0.17 & & & & &1.33|0.14 &1.26|0.19 &1.53|0.25 & &1.10|0.20  \\\hline
	9 &0.90|0.15 & & & & &0.97|0.17 &1.35|0.24 &1.13|0.21 &1.10|0.20 &  \\\hline
	\end{tabular}
	
	
		\vspace{0.6cm}
	
		Open 600 using max and no Imp<0, average = 
	
	\begin{tabular}{|r||c|c|c|c|c|c|c|c|c|c|}\hline
		{\bf "2v"} & 0 & 1 & 2 & 3 & 4 & 5 & 6 & 7 & 8 & 9 \\\hline\hline
0 & &1.22|0.48 &0.98|0.36 &1.23|0.47 &1.18|0.37 &1.24|0.41 &1.05|0.32 &1.24|0.38 &1.08|0.32 &0.88|0.28  \\\hline
1 &1.22|0.48 & &1.11|0.37 & & & & & & &  \\\hline
2 &0.98|0.36 &1.11|0.37 & & & & & & & &  \\\hline
3 &1.23|0.47 & & & & & & & & &  \\\hline
4 &1.18|0.37 & & & & & & & & &  \\\hline
5 &1.24|0.41 & & & & & &1.01|0.37 &1.38|0.44 &1.36|0.33 &0.97|0.36  \\\hline
6 &1.05|0.32 & & & & &1.01|0.37 & &1.51|0.62 &1.17|0.39 &1.30|0.49  \\\hline
7 &1.24|0.38 & & & & &1.38|0.44 &1.51|0.62 & &1.46|0.51 &1.16|0.39  \\\hline
8 &1.08|0.32 & & & & &1.36|0.33 &1.17|0.39 &1.46|0.51 & &1.08|0.33  \\\hline
9 &0.88|0.28 & & & & &0.97|0.36 &1.30|0.49 &1.16|0.39 &1.08|0.33 &  \\\hline
	\end{tabular}
	
	

\end{table}


\clearpage

\begin{table}
	\begin{tabular}{|r||c|c|c|c|c|c|c|c|c|c|}
\hline
Target & 3v 1000 & 2v 1000 & 2v I>0 & New 2v & New 3v & $Imp\leq 2\cdot 10^{-4}$ & $Imp\leq 10^{-4}$ & M150I300  & Max600  \\\hline
Avg. & 1.1486 & 1.2264 & 1.2291 & 1.1671 & 1.1496 & 1.1771 & 1.1625 & 1.2056 & 1.1689  \\\hline
0 &1.20| &1.24| &1.26|0.26 &1.22|0.56 &1.18|0.32 &1.21|0.63 &1.19|0.49 &1.23|0.30 &1.20|0.48  \\\hline
1 &0.96| &0.99| &0.96|0.29 &0.97|0.43 &0.95|0.28 &0.96|0.55 &1.00|0.43 &0.95|0.26 &0.97|0.36  \\\hline
2 &1.22| &1.23| &1.24|0.38 &1.25|0.52 &1.20|0.32 &1.23|0.69 &1.23|0.55 &1.29|0.33 &1.22|0.47  \\\hline
3 &1.16| &1.24| &1.22|0.30 &1.16|0.46 &1.15|0.26 &1.21|0.58 &1.19|0.39 &1.16|0.21 &1.17|0.37  \\\hline
4 &1.24| &1.31| &1.36|0.36 &1.24|0.47 &1.25|0.32 &1.25|0.68 &1.23|0.47 &1.25|0.25 &1.23|0.41  \\\hline
5 &1.01| &1.02| &1.10|0.21 &1.02|0.39 &1.03|0.20 &1.04|0.50 &1.00|0.35 &1.02|0.17 &1.04|0.32  \\\hline
6 &1.26| &1.33| &1.29|0.27 &1.22|0.48 &1.21|0.23 &1.22|0.61 &1.18|0.42 &1.21|0.25 &1.22|0.38  \\\hline
7 &1.16| &1.17| &1.18|0.20 &1.11|0.43 &1.11|0.21 &1.14|0.48 &1.10|0.33 &1.23|0.17 &1.06|0.32  \\\hline
8 &0.90| &1.04| &0.92|0.20 &0.91|0.34 &0.88|0.17 &0.95|0.46 &0.93|0.32 &0.90|0.15 &0.87|0.28  \\\hline
9 &1.17| &1.20| &1.13|0.25 &1.08|0.45 &1.09|0.25 &1.14|0.57 &1.12|0.41 &1.19|0.18 &1.10|0.37  \\\hline
35 &1.31| &1.03| &0.99|0.26 &0.96|0.45 &0.99|0.24 &1.02|0.53 &0.99|0.41 &1.00|0.16 &1.00|0.37  \\\hline
36 &1.23| &1.40| &1.46|0.31 &1.41|0.65 &1.32|0.42 &1.32|0.55 &1.35|0.40 &1.44|0.23 &1.37|0.44  \\\hline
37 &0.94| &1.40| &1.40|0.16 &1.27|0.53 &1.26|0.18 &1.32|0.45 &1.23|0.32 &1.33|0.14 &1.34|0.33  \\\hline
38 &0.91| &0.98| &1.04|0.27 &0.97|0.48 &0.96|0.31 &0.96|0.57 &0.98|0.42 &0.97|0.17 &0.96|0.36  \\\hline
39 &1.43| &1.56| &1.55|0.53 &1.47|0.70 &1.44|0.52 &1.50|0.84 &1.54|0.66 &1.57|0.44 &1.50|0.62  \\\hline
40 &1.14| &1.18| &1.38|0.28 &1.18|0.51 &1.16|0.26 &1.16|0.56 &1.16|0.43 &1.26|0.19 &1.16|0.39  \\\hline
41 &1.23| &1.37| &1.32|0.37 &1.28|0.59 &1.27|0.37 &1.27|0.70 &1.27|0.54 &1.35|0.24 &1.29|0.49  \\\hline
42 &1.34| &1.44| &1.50|0.39 &1.40|0.66 &1.39|0.28 &1.42|0.67 &1.41|0.53 &1.53|0.25 &1.45|0.51  \\\hline
43 &1.07| &1.15| &1.20|0.29 &1.08|0.41 &1.09|0.27 &1.10|0.58 &1.08|0.47 &1.13|0.21 &1.15|0.39  \\\hline
44 &1.09| &1.24| &1.10|0.23 &1.15|0.45 &1.08|0.29 &1.12|0.51 &1.09|0.40 &1.10|0.20 &1.07|0.33  \\\hline
	\end{tabular}
\end{table}

\clearpage

Tables for pipe system, formal outputs: The first and second lines are bounds and solutions the test for open all nodes, the 3rd, 4th lines are bounds and solutions for the test of open all nodes that $Imp > 2\cdot10^{-5}$.

TO = 1200 for first 3 tests, and the last two tests are with TO = 3600s.



\begin{table}
	\begin{tabular}{|r||c|c|c|c|c|c|c|c|c|c|c|c|c|c|c|c|c|c|c|c|c|}
		\hline
	Case& 0 & 1 & 2 & 3 & 4 & 5 & 6 & 7 & 8 & 9 & 10 & 11 & 12 & 13 & 14 & 15 & 16 & 17 & 18 & 19 \\\hline\hline
All & 9.4 &9.2 &9.7 &10.7 &8.9 &11.9 &9.8 &9.7 &7.5 &8.9 &8.8 &8.1 &9.3 &6.7 &6.5 &5.0 &4.7 &5.4 &4.8 &4.3  \\\hline
All & 3.7 &4.7 &4.7 &6.2 &3.0 &4.5 &1.4 &3.3 &1.7 &1.5 &1.3 &1.7 &2.8 &1.7 &1.4 &1.5 &inf &0.8 &1.2 &0.8  \\\hline\hline

$2\cdot10^{-5}$&8.7 &8.8 &10.2 &10.1 &8.4 &11.3 &8.8 &7.9 &7.1 &7.4 &8.3 &6.9 &8.6 &6.1 &6.0 &4.6 &4.5 &4.6 &4.1 &4.3  \\\hline
$2\cdot10^{-5}$&5.7 &6.0 &4.8 &7.4 &4.9 &6.7 &3.4 &4.2 &3.0 &3.2 &3.2 &2.7 &4.3 &3.5 &3.2 &2.4 &1.7 &1.8 &2.0 &1.8  \\\hline\hline

$4\cdot10^{-5}$&8.8 &8.9 &9.5 &10.0 &8.2 &11.4 &9.0 &8.1 &7.4 &7.5 &7.9 &6.9 &8.6 &6.1 &6.3 &4.5 &4.4 &4.6 &4.4 &4.3  \\\hline
$4\cdot10^{-5}$&5.7 &5.9 &5.4 &7.6 &4.8 &6.8 &2.7 &4.2 &2.9 &3.2 &3.3 &2.7 &4.3 &3.5 &3.2 &2.5 &1.7 &1.7 &2.0 &1.8  \\\hline\hline

$2\cdot10^{-5}$&8.0 &7.9 &8.7 &9.3 &7.3 &10.5 &7.8 &7.0 &6.4 &6.7 &6.6 &6.1 &8.0 &5.0 &5.2 &4.1 &3.9 &4.0 &3.9 &3.7  \\\hline
$2\cdot10^{-5}$&5.7 &6.1 &5.4 &7.6 &4.9 &6.9 &3.7 &4.4 &3.2 &3.0 &3.1 &3.0 &4.2 &3.5 &3.4 &2.5 &1.8 &2.0 &2.0 &1.8  \\\hline\hline

$4\cdot10^{-5}$&7.9 &8.0 &8.8 &9.2 &7.3 &10.5 &7.8 &7.0 &6.4 &6.7 &6.9 &6.1 &8.0 &5.4 &5.1 &4.1 &3.9 &4.0 &3.9 &3.7  \\\hline
$4\cdot10^{-5}$&5.7 &6.1 &5.4 &7.6 &4.8 &7.0 &3.5 &4.4 &3.2 &3.2 &3.2 &2.9 &4.3 &3.5 &3.4 &2.5 &1.7 &2.0 &2.1 &1.8  \\\hline

&8.5 &8.7 &9.4 &9.9 &8.1 &11.0 &8.5 &7.4 &6.9 &7.3 &7.6 &6.7 &8.4 &6.0 &5.8 &4.5 &4.3 &4.5 &4.0 &4.2  \\\hline
&5.6 &6.1 &5.4 &7.4 &4.9 &6.8 &2.8 &4.2 &3.0 &3.2 &3.0 &2.9 &4.2 &3.5 &3.0 &2.4 &1.7 &2.0 &2.0 &1.8  \\\hline
	\end{tabular}
	
	

	
\end{table}
	
\end{document}
