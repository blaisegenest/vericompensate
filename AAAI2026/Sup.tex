\documentclass[letterpaper]{article} % DO NOT CHANGE THIS
\usepackage[submission]{aaai2026}  % DO NOT CHANGE THIS
\usepackage{times}  % DO NOT CHANGE THIS
\usepackage{helvet}  % DO NOT CHANGE THIS
\usepackage{courier}  % DO NOT CHANGE THIS
\usepackage[hyphens]{url}  % DO NOT CHANGE THIS
\usepackage{graphicx} % DO NOT CHANGE THIS
\urlstyle{rm} % DO NOT CHANGE THIS
\def\UrlFont{\rm}  % DO NOT CHANGE THIS
\usepackage{natbib}  % DO NOT CHANGE THIS AND DO NOT ADD ANY OPTIONS TO IT
\usepackage{caption} % DO NOT CHANGE THIS AND DO NOT ADD ANY OPTIONS TO IT
\frenchspacing  % DO NOT CHANGE THIS
\setlength{\pdfpagewidth}{8.5in} % DO NOT CHANGE THIS
\setlength{\pdfpageheight}{11in} % DO NOT CHANGE THIS
%
% These are recommended to typeset algorithms but not required. See the subsubsection on algorithms. Remove them if you don't have algorithms in your paper.
\usepackage{algorithm}
\usepackage{algorithmic}
\pagestyle{plain}
\usepackage{threeparttable}
\input{math_commands.tex}
\usepackage{lineno}
\usepackage{subcaption}
\usepackage{tabularx}
\usepackage{cases}
\captionsetup{compatibility=false}
\usepackage{epstopdf}
\usepackage{placeins}
\usepackage{pgfplots}
\usepackage{tikz}
\usepackage{calc}
\usepackage{array}
%\usepackage[linesnumbered,ruled,vlined]{algorithm2e}
\usetikzlibrary{positioning, arrows.meta,calc}
\usepackage{newfloat}
\usepackage{listings}
% allow to type " and get latex to handle it
\usepackage[autostyle=false, style=english]{csquotes}
\MakeOuterQuote{"}
\DeclareCaptionStyle{ruled}{labelfont=normalfont,labelsep=colon,strut=off} % DO NOT CHANGE THIS
\lstset{%
	basicstyle={\footnotesize\ttfamily},% footnotesize acceptable for monospace
	numbers=left,numberstyle=\footnotesize,xleftmargin=2em,% show line numbers, remove this entire line if you don't want the numbers.
	aboveskip=0pt,belowskip=0pt,%
	showstringspaces=false,tabsize=2,breaklines=true}
\floatstyle{ruled}
\newfloat{listing}{tb}{lst}{}
\floatname{listing}{Listing}
%
% Keep the \pdfinfo as shown here. There's no need
% for you to add the /Title and /Author tags.
\pdfinfo{
	/TemplateVersion (2026.1)
}

\title{Order Reduction and MILP Models for Real-Time DNN Robustness Certification \\ Supplementary Material}
\date{}
\author{
	%Authors
	% All authors must be in the same font size and format.
	Written by AAAI Press Staff\textsuperscript{\rm 1}\thanks{With help from the AAAI Publications Committee.}\\
	AAAI Style Contributions by Pater Patel Schneider,
	Sunil Issar,\\
	J. Scott Penberthy,
	George Ferguson,
	Hans Guesgen,
	Francisco Cruz\equalcontrib,
	Marc Pujol-Gonzalez\equalcontrib
}
\affiliations{
	%Afiliations
	\textsuperscript{\rm 1}Association for the Advancement of Artificial Intelligence\\
	% If you have multiple authors and multiple affiliations
	% use superscripts in text and roman font to identify them.
	% For example,
	
	% Sunil Issar\textsuperscript{\rm 2},
	% J. Scott Penberthy\textsuperscript{\rm 3},
	% George Ferguson\textsuperscript{\rm 4},
	% Hans Guesgen\textsuperscript{\rm 5}
	% Note that the comma should be placed after the superscript
	
	1101 Pennsylvania Ave, NW Suite 300\\
	Washington, DC 20004 USA\\
	% email address must be in roman text type, not monospace or sans serif
	proceedings-questions@aaai.org
	%
	% See more examples next
}

%Example, Single Author, ->> remove \iffalse,\fi and place them surrounding AAAI title to use it
\iffalse
\title{My Publication Title --- Single Author}
\author {
	Author Name
}
\affiliations{
	Affiliation\\
	Affiliation Line 2\\
	name@example.com
}
\fi

\iffalse
%Example, Multiple Authors, ->> remove \iffalse,\fi and place them surrounding AAAI title to use it
\title{My Publication Title --- Multiple Authors}
\author {
	% Authors
	First Author Name\textsuperscript{\rm 1},
	Second Author Name\textsuperscript{\rm 2},
	Third Author Name\textsuperscript{\rm 1}
}
\affiliations {
	% Affiliations
	\textsuperscript{\rm 1}Affiliation 1\\
	\textsuperscript{\rm 2}Affiliation 2\\
	firstAuthor@affiliation1.com, secondAuthor@affilation2.com, thirdAuthor@affiliation1.com
}
\fi


\newtheorem{proposition}{Proposition}
\newtheorem{definition}{Definition}
\newcommand{\vW}{\boldsymbol{W}}
\newcommand{\val}{{\textrm{value}}}
\newcommand{\Val}{{\textrm{value}}}
\newcommand{\MILP}{{\textrm{MILP}}}
\newcommand{\LP}{{\textrm{LP}}}
\newcommand{\Improve}{\mathrm{Improve}}
\newcommand{\Utility}{\mathrm{SAS}}
\newcommand{\Sol}{\mathrm{Sol}}
\newcommand{\sol}{\mathrm{sol}}
\newcommand{\UB}{\mathrm{UB}}
\newcommand{\LB}{\mathrm{LB}}
\newcommand{\ub}{\mathrm{ub}}
\newcommand{\lb}{\mathrm{lb}}
\newcommand{\B}{\mathrm{B}}
\usepackage{amsmath, amssymb, amsfonts}
\newcommand{\ReLU}{\mathrm{ReLU}}
\newcommand{\CMP}{{\textrm{CMP}}\ }
\newcommand{\fix}{\marginpar{FIX}}
\newcommand{\new}{\marginpar{NEW}}
\newcommand{\toolname}{Hybrid MILP}

% REMOVE THIS: bibentry
% This is only needed to show inline citations in the guidelines document. You should not need it and can safely delete it.
\usepackage{bibentry}
% END REMOVE bibentry

\begin{document}
	
\maketitle


\section{Reproducibility Checklist}

This paper:
\begin{itemize}
\item Includes a conceptual outline and/or pseudocode description of AI methods introduced {\bf yes}
\item Clearly delineates statements that are opinions, hypothesis, and speculation from objective facts and results {\bf yes}
\item Provides well marked pedagogical references for less-familiar readers to gain background necessary to replicate the paper {\bf yes}
\item Does this paper make theoretical contributions? {\bf yes}
\end{itemize}

if yes, then:

\begin{itemize}
\item All assumptions and restrictions are stated clearly and formally. {\bf yes}
\item All novel claims are stated formally (e.g., in theorem statements). {\bf yes}
\item Proofs of all novel claims are included. {\bf yes} (in the supplementary material)
\item Proof sketches or intuitions are given for complex and/or novel results. {\bf yes}
\item Appropriate citations to theoretical tools used are given. {\bf yes}
\item All theoretical claims are demonstrated empirically to hold. {\bf yes}
\item All experimental code used to eliminate or disprove claims is included. {\bf yes}
\end{itemize}




\section{Proof of Proposition 2}

Recall first that we use the classical MILP encoding 
for $\hat{x}'_i=\ReLU(x'_i)$, using one binary variable $a'$. We reuse 
$a'$ (with its value already settled, see Proposition 1 in the main paper) 
in the encoding of $\hat{y}_i = \ReLU(x_i)-\ReLU(x'_i)$:

\addtocounter{proposition}{1}
\addtocounter{table}{1}

\begin{proposition}
Assuming that $y_i \in [-\gamma_i, \gamma_i]$,
and that $x'_i \in [\alpha_i,\beta_i]$,
we have that $\hat{y}_i = \ReLU(x_i=x'_i+y_i)-\ReLU(x'_i)$ is the solution of:
	\begin{align*}
		& \begin{aligned}
			y_i + x'_i &\leq a\beta_i        &
			y_i &+ x'_i \geq (1-a)\alpha_i \\
			x'_i       &\leq a'\beta_i       & 
			x'_i       &\geq (1-a')\alpha_i \\
			\hat{y}_i  &\leq a\gamma_i       &
			\hat{y}_i  &\geq -a'\gamma_i \\
			\hat{y}_i  &\leq y_i + (1-a)\gamma_i  &
			\hat{y}_i  &\geq y_i - (1-a')\gamma_i \\
			\hat{y}_i  &\leq -x'_i + a\beta_i &
			\hat{y}_i  &\geq -x'_i + (1-a')\alpha_i \\
			\hat{y}_i  &\leq y_i + x'_i + (1-a)(-\alpha_i) &
			\hat{y}_i  &\geq y_i + x'_i + a'(-\beta_i)
		\end{aligned}
	\end{align*} 
    where $a,a' \in \{0,1\}$ are binary variables, 
    and $a'$ is shared with the classical MILP
    encoding for $\hat{x}'_i=\ReLU(x'_i)$.
\end{proposition}



	We do a case analysis depending on the value of both binary variables. 
First, we know the constraints for $\hat{x_i}'=\ReLU(x_i')$ are exact. 

We have 2 binary variables and 4 cases in total. We only need to check that, in all 4 cases, $$\hat{y}_i = \ReLU(x'_i+y_i)-\ReLU(x'_i).$$

\textbf{Case 1:} if $a = 1$ and $a' = 1$, then $x'_i \geq 0 $ and $y_i+x'_i\geq 0$, then we need to show $\hat{y}_i = y_i$ based on  $\hat{x'_i} = x'_i$. This is true by the two inequalities in line 4.

\textbf{Case 2:}  if $a = 1$ and $a' = 0$, then $x'_i \leq 0 $ and $y_i+x'_i\geq 0$, then we need to show $\hat{y}_i = y_i+x'_i$ based on  $\hat{x'_i} = 0$. This is true by the two inequalities in line 6.

\textbf{Case 3:} if $a = 0$ and $a' = 0$, then $x'_i \leq 0 $ and $y_i+x'_i\leq 0$, then we need to show $\hat{y}_i = 0$ based on  $\hat{x'_i} = 0$. This is true by the two inequalities in line 3.

\textbf{Case 4:} if $a = 0$ and $a' = 1$, then $x'_i \geq 0 $ and $y_i+x'_i\leq 0$, then we need to show $\hat{y}_i = -x'_i$ based on  $\hat{x'_i} = x'_i$. This is true by the two inequalities in line 5.

\section{PCA Model Order Reduction for Bound Computation}

%Some words on PCA, learn of 2 Matrix full-dim to reduced-dim and back.
%Then, how we use it (pictures and text).

\begin{figure*}[h!]
    \centering
    \includegraphics[scale=0.9]{MNIST.pdf} \hspace{1.5cm}
    \caption{Training, exploitation, and bound computation on the MNIST dataset.}
    \label{fig.MNIST}
\end{figure*}	

\begin{figure*}[h!]
    \centering
    \includegraphics[scale=0.9]{PIPE.pdf} \hspace{1.5cm}
    \caption{Training and bound computation on the pipe use case. 
    Note that two PCAs are used, one for the input space (deformation) and another for the output space (plastic strain). The exploitation and bound computation must then use a pipeline that: (1) reduces the deformation; (2) obtains a reduced strain with the surrogate, and (3) decodes the reduced strain from the reduced to the full dimension.}
    \label{fig.PIPE}
\end{figure*}	


PCA (Chinesta et al. 2017) is a linear technique to reduce the dimension of a dataset while keeping its main information.
From a dataset, the eigenvectors are computed, defining the most important linear components of this dataset.
Projecting over the first few main eigenvectors is a powerful model order reduction technique ({\em encoding} / {\em reducing from the full dimension}). 
It is easy to go from the reduced dimension back to the full dimension
by making the product between the reduced vector and the first few eigenvectors ({\em decoding}).

Figure~\ref{fig.MNIST} depicts the MNIST pipeline on reduced dimensions, computing:

\begin{enumerate}
\item the PCA encoding and decoding as well as the MNIST DNN from the MNIST Training DataSet ({\em Training}).
We choose to reduce the dimension from 784 to 20, using the first 20 eigen vectors from PCA, with a PCA encoder and a PCA decoder.

\item {\em In exploitation}, we check whether a test dataset has the same accuracy (class predicted matches the ground truth) 
going directly through the MNIST DNN rather than being encoded and decoded by the PCA decoding first.
The number of retained PCA dimensions (20) is selected to ensure that the MNIST DNN's classification accuracy on reconstructed images (i.e., after PCA encoding and decoding) remains the same $97$\% as the MNIST DNN used directly on the full-dimensional MNIST images.
It ensures no loss of accuracy.
%
\item {\em In Bound Computation}, we consider as input the 20 ($\times 2$: input and perturbed input') reduced PCA dimensions, from which all other variables are fixed.
However, we can define 784 $\times 2$ linear variables corresponding to the original image $I$ and its perturbation $I'$ before PCA encoding, as the constraints between these and the input are linear.
We fix the $L_1$-perturbation $\varepsilon$ on the image $I$ and its deformation $I'$, rather than on the inputs, as that is where it is meaningful.
Finally, the goal is to optimize the value of $o_6 -o_8+ o'_8 - o'_6$,  where $o_C$ is the output neuron of the DNN corresponding to class $C \in \{0,\ldots, 9\}$.
\end{enumerate}

%
%It was also used to obtain the PCA encoding and decoding, transforming the images into a reduced basis. The number of PCA dimensions was chosen such that the exploitation accuracy remains constant. That is, encoding into a PCA basis and decoding it results in the same accuracy, as illustrated in ``Exploitation''. 
%
%Last, for the ``bound computation'', we determine an $L_1$-perturbation $\varepsilon$. The search space for bound computation is 20-dimensional. However, the bound itself is computed on the full space. This is possible since PCA is a \emph{linear operation} based on the eigenvectors of the covariance matrix and its transpose for the PCA decoding (inverse).

\newpage


Figure~\ref{fig.PIPE} illustrates the pipeline used in the pipe strain case. 
Two different PCA reductions are used. The first one is on the deformation training dataset, 
and the second is on the plastic strain one.
%
The surrogate is then learned from the reduced deformation to the reduced plastic strain. 
%

To predict the plastic strain (which is not directly observable), we thus consider the deformation (which is directly observable), reduce it using the PCA reduction to obtain the reduced deformation, call the surrogate model, obtain the associated predicted reduced strain, and decode the reduced strain, obtaining the strain of the pipe in the full 3000 dimensions. Similarly to the MNIST case, we consider as input the 10 $\times 2$ dimensions of reduced deformation, but also consider 3000 $\times 2$ linear variables corresponding to the deformation and its perturbation, with linear constraints between them and the inputs (provided by the PCA encoder, which is given under the form of a matrix), and set the $L_1$ perturbation $\varepsilon$ over the deformation and its perturbation, where it is physically meaningful.
The surrogate outputs 26 output dimensions ($\times 2$) of reduced strain, but again, as we have the linear PCA decoder explicitly, we can define the optimization goal in terms of optimizing the difference over 10 specific points in the full geometry (where it is are physically relevant), rather than on the reduced strain space.

\newpage



	\section{Additional Experimental Evaluations}
	

\iffalse

\subsection{Classical vs our "2v" model vs ITNE}


\begin{table}[h!]
	\centering
	\begin{tabular}{||l|c|c|c||}\hline\hline
		model &        Bound $\downarrow$ &  Sol. &      Worst-Case $\uparrow$ \\\hline \hline
	Classical, $0 \times 2$ (LP)&  ? & ? & ?
    \\\hline
	ITNE, $0 \times 2$ (LP) &    ? & ? & ?
    \\\hline
	2v model, $0 \times 2$ (LP) &    ? & ? & ?
    \\\hline \hline
	
		Classical, $50 \times 2$ &    $.320$ &  $.320$ & $.017$ 
    \\\hline
	ITNE, $50 \times 2$ &    $.042$ &  $.037$ & $.022$
	\\ \hline
    2v model, $50 \times 2$ &    {\bf .040} &  $.037$ &  $.018$ 
    \\\hline \hline
    Classical, $100 \times 2$ &  .186  &  $.022$ & $.022$ 
    \\\hline
	ITNE, $100 \times 2$ &    $.045$ &  $.023$ & .023
    \\\hline
	2v model, $100 \times 2$&     {\bf .042} &  $.023$ &   .023 
    \\\hline \hline
	\end{tabular}
	\caption{Comparison of the classical encoding, ITNE and our "2v" model on the pipe system 
	with a fixed timeout of 1000s, with either 0 (LP), $50 \times 2$, 
    or the full $100 \times 2$ binary variables.}
    \label{table.classical}
\end{table}

\fi



	\subsection{Experimental results for robustness (MNIST)}
	
	
	\begin{table}[h!]
		\centering
	\begin{tabular}{||l||c|c|c||}\hline\hline
		model &        Bound $\downarrow$ &  Sol. &      Worst-Case $\uparrow$ \\\hline \hline
		1v, $0 \times 1$ (LP) & 20.2612  & 20.2612  & .028 \\\hline 
		3v, $0 \times 3$ (LP) & {\bf 20.2614}  & 20.2614  & .125 \\\hline 
	    2v, $0 \times 2$ (LP) & {\bf 20.2614}  & 20.2614  & {\bf .145} \\\hline\hline	 

		1v, $300 \times 1$ & ? & ? & ? \\\hline 
		3v, $300 \times 3$ & ? & ? & ? \\\hline 
	    2v, $300 \times 2$ & ? & ? & ? \\\hline\hline	 

		1v, $400 \times 1$ & {\bf 14.87} & 6.456 & .032 \\\hline 
		3v, $400 \times 3$ & 16.62 & 6.343 & .254 \\\hline 
	    2v, $400 \times 2$ & 16.33 & 5.777 & .371 \\\hline \hline

		1v, $500 \times 1$ & {\bf 14.97} & $.845$ & $.009$ \\\hline 
		3v, $500 \times 3$ & $17.66$ & $.813$ & {\bf .518} \\\hline 
	    2v, $500 \times 2$ & $16.49$ & n/a & n/a \\\hline\hline	 
	\end{tabular}
	\caption{Bounds on $\beta^{.5}_{6,8}$ 
	obtained by the "1v", "3v" and "2v" models 
	on the {\bf full dimension} MNIST DNN, 
	for timeouts of $14400$s, when 0, 300, 400 or 500 ($\times 1$, $\times 2$, $\times 3$) variables are binary.}
	\label{table.mnist}
\end{table}

In Table \ref{table.mnist}, we further tested different number of reduced variables. 
With 0 binary variables (full LP), the "1v" model converges in 20 seconds, and the "2v" and "3v" models in 200s.
Opening $400 (\times 1,2,3)$ variables shows a very slight improvement in the bound over 500 variables, 
far insufficient to prove that any image is robust in real-time. Notice that LP

\subsection{Reduced dimension}

\begin{table}[h!]
	\centering
	\begin{tabular}{||l||c|c|c||}\hline\hline
		model &        Bound$\downarrow$ &  Sol. &      Worst-Case$\uparrow$ \\\hline \hline
1v, $400 \times 1$ & $1.414$ &  $.691$ & $.010$ \\\hline 
3v, $400 \times 3$ & $1.186$ & $.600$ & $.003$ \\\hline 
2v, $400 \times 2$ & $1.274$ & $.566$ & $.002$ \\\hline\hline
	 
1v, $475 \times 1$ &  $1.408$ & $.301$ & $.008$  \\\hline 
3v, $475 \times 3$ &  $1.153$ & $.250$ & $.006$ \\ \hline 
2v, $475 \times 2$ &  $1.247$ & $.1957$ & $.019$ \\\hline\hline

1v, $500 \times 1$ & $1.412$ & $.161$ & .057 \\\hline 
3v, $500 \times 3$ & {\bf 1.137} & $.103$ & $.065$\\\hline 
2v, $500 \times 2$ &  $1.182$ & $.084$& {\bf .084}  \\\hline\hline
	 
	\end{tabular}
	\caption{Comparison of "1v", "3v" and "2v" models 
	to obtain bounds on $\beta^{.5}_{6,8}$ on the {\bf 20 dimension} reduced order MNIST DNN, for timeout of 14400s, 
	where 400, 475,  or 500 ($\times 1$, $\times 2$, $\times 3$) neurons use binary variables.}
	\label{table.reduced}
\end{table}


Notice that the MILP process is far from converging (large difference between the Bound and the Solution) after 4hours (14400s), and we could use much longer 
runtime in order to obtain better bounds. As this task is performed only once offline, this is not limiting.

Further, we report in table \ref{table.pair} the bounds obtained for every pair of class $C < D \in \{0, \ldots, 9\}$
using all $500 (\times 1,2,3) $ binary variables.





The tests with restricted number of binary variables do not improve the percentage of images certified robust (Table \ref{table.cert}). 

\begin{table}[h!]
	\begin{tabular}{||l||c|c|c|c||}\hline\hline
		model &    $L_1\leq 0.5$ & $L_1\leq 1$ & $L_1\leq 1.5$ &  $L_1\leq 2$ \\\hline \hline
		1v, $500\times1$ & $80 \%$ & $32\%$ & $7\%$ & $0\%$ \\\hline
		3v, $500 \times 3$ & {\bf 86 \%} & {\bf 53\%} & {\bf 20\%} & {\bf 4\%} \\\hline
		2v, $500 \times 2$ & 84\% & 51\% & 16\% & {\bf 4\%} \\\hline \hline
	\end{tabular}
	\caption{Percentage of images certified robust in real-time 
	using the computed $(\beta^{.5}_{i,j})_{i < j \leq 10}$ 
	by the "1v", "3v" and "2v" models, for different values of $L_1$-perturbations.}
    \label{table.cert}
\end{table}





	
	

\subsection{Experimental results for regression (Pipe strain)}


	For the pipe system in Table \ref{table.pipe} ...



	
	\iffalse
	\begin{table}[h!]
	\begin{tabular}{|l|l|l|l|l|}\hline
		$L_1\leq 0.83$ &        Bound $\downarrow$ &  Solution $\uparrow$ &      Real $\uparrow$ &  Time \\\hline
		1v,open 100 &     {\bf 0.035613} &  0.035613 &                       0.01288 & 10608 \\\hline
		3v,open 100 &     0.040074 &  0.028934 &                      0.021441 & 10922 \\\hline
		%3v,open 100 &     0.039824 &  0.028832 &                      0.022255 & 22153 \\\hline
		2v,open 100 &     0.046719 &  0.024364 &  {\bf 0.024436} & 10922 \\\hline
	\end{tabular}
	\caption{Comparison of 1v,2v and 3v models on the pipe system with a fixed timeout of 10.000s.}
\end{table}
\fi
	
		
	\begin{table}[h!]
	\begin{tabular}{||l||c|c|c|c||}\hline\hline
		model &        Bound$\downarrow$ &  Sol. &      Worst-Case$\uparrow$ &  Time(s) \\\hline \hline
		1v, $100 \times 1$ &     {\bf .0356} &  $.0356$ & $.0191$ &  1000 \\\hline
		3v, $100 \times 3$&     .0414 &  .0254 &  .0166 &  1000 \\\hline
		2v, $100 \times 2$&     .0418 &  .0229 &   {\bf .0229} &  1000 \\\hline \hline
	3v, $97 \times 3$&     .0352 &  .0284 &  .0196 & 14440 \\\hline
		3v, $100 \times 3$&      {\bf .0350} &  .0272 &  .0216 & 14440 \\\hline
	2v, $97 \times 2$&     .0393 &  .0240 &   {\bf .0237} & 14440 \\\hline
		2v, $100 \times 2$&     .0360 &  .0236 &   .0236 & 14440 \\\hline \hline
		3v, $100 \times 3$&     {\bf .0329} &  .0277 &  .0165 & 72000 \\\hline
		2v, $100 \times 2$&     .0337 &  .0245 &  {\bf .0245} & 72000 \\\hline\hline
	\end{tabular}
	\caption{Comparison of "1v", "3v" and "2v" models on the pipe system with timeouts of 1000s, 14440s and 72000s, where all 100 ($\times 1, \times 2,\times 3$) neurons use binary variables.}
	\label{table.pipe}
\end{table}



\begin{table*}
	\centering
	\begin{tabular}{|l||c|c|c|c|c|c|c|c|c|c|}\hline
		{\bf "1v"} & 0 & 1 & 2 & 3 & 4 & 5 & 6 & 7 & 8 & 9 \\\hline\hline
0 & &1.51 &1.21 &1.45 &1.45 &1.46 &1.31 &1.47 &1.44 &1.14  \\\hline
1 &1.51 & &1.44 &1.38 &1.16 &1.29 &1.37 &1.22 &1.51 &1.29  \\\hline
2 &1.21 &1.44 & &1.42 &1.60 &1.69 &1.53 &1.32 &1.52 &1.34  \\\hline
3 &1.45 &1.38 &1.42 & &1.58 &1.41 &1.64 &1.42 &1.52 &1.34  \\\hline
4 &1.45 &1.16 &1.60 &1.58 & &1.53 &1.38 &1.41 &1.64 &1.10  \\\hline
5 &1.46 &1.29 &1.69 &1.41 &1.53 & &1.16 &1.53 &1.61 &1.13  \\\hline
6 &1.31 &1.37 &1.53 &1.64 &1.38 &1.16 & &1.67 &1.41 &1.49  \\\hline
7 &1.47 &1.22 &1.32 &1.42 &1.41 &1.53 &1.67 & &1.67 &1.32  \\\hline
8 &1.44 &1.51 &1.52 &1.52 &1.64 &1.61 &1.41 &1.67 & &1.36  \\\hline
9 &1.14 &1.29 &1.34 &1.34 &1.10 &1.13 &1.49 &1.32 &1.36 &  \\\hline
	\end{tabular}

	\vspace{0.3cm}
	
	\begin{tabular}{|l||c|c|c|c|c|c|c|c|c|c|}\hline
		{\bf "3v"} & 0 & 1 & 2 & 3 & 4 & 5 & 6 & 7 & 8 & 9 \\\hline\hline
0 & &1.08 &0.85 &1.13 &1.06 &1.12 &0.90 &1.15 &1.03 &0.79  \\\hline
1 &1.08 & &1.01 &0.99 &0.84 &1.02 &1.09 &0.87 &1.14 &0.99  \\\hline
2 &0.85 &1.01 & &1.01 &1.16 &1.31 &1.26 &0.92 &1.14 &0.96  \\\hline
3 &1.13 &0.99 &1.01 & &1.26 &1.03 &1.27 &0.99 &1.12 &0.95  \\\hline
4 &1.06 &0.84 &1.16 &1.26 & &1.19 &1.01 &1.05 &1.25 &0.73  \\\hline
5 &1.12 &1.02 &1.31 &1.03 &1.19 & &0.86 &1.17 &1.15 &0.84  \\\hline
6 &0.90 &1.09 &1.26 &1.27 &1.01 &0.86 & &1.31 &1.04 &1.13  \\\hline
7 &1.15 &0.87 &0.92 &0.99 &1.05 &1.17 &1.31 & &1.26 &0.97  \\\hline
8 &1.03 &1.14 &1.14 &1.12 &1.25 &1.15 &1.04 &1.26 & &1.01  \\\hline
9 &0.79 &0.99 &0.96 &0.95 &0.73 &0.84 &1.13 &0.97 &1.01 &  \\\hline
	\end{tabular}
	
\vspace{0.3cm}

	\begin{tabular}{|l||c|c|c|c|c|c|c|c|c|c|}\hline
		{\bf "2v"} & 0 & 1 & 2 & 3 & 4 & 5 & 6 & 7 & 8 & 9 \\\hline\hline
	0 & &1.24 &0.99 &1.23 &1.24 &1.31 &1.02 &1.33 &1.17 &1.04  \\\hline
	1 &1.24 & &1.20 &1.09 &0.92 &1.13 &1.15 &0.97 &1.27 &1.16  \\\hline
	2 &0.99 &1.20 & &1.20 &1.29 &1.50 &1.46 &1.08 &1.30 &1.10  \\\hline
	3 &1.23 &1.09 &1.20 & &1.41 &1.21 &1.35 &1.19 &1.26 &1.05  \\\hline
	4 &1.24 &0.92 &1.29 &1.41 & &1.37 &1.23 &1.31 &1.50 &0.92  \\\hline
	5 &1.31 &1.13 &1.50 &1.21 &1.37 & &1.03 &1.40 &1.40 &0.98  \\\hline
	6 &1.02 &1.15 &1.46 &1.35 &1.23 &1.03 & &1.56 &1.18 &1.37  \\\hline
	7 &1.33 &0.97 &1.08 &1.19 &1.31 &1.40 &1.56 & &1.44 &1.15  \\\hline
	8 &1.17 &1.27 &1.30 &1.26 &1.50 &1.40 &1.18 &1.44 & &1.24  \\\hline
	9 &1.04 &1.16 &1.10 &1.05 &0.92 &0.98 &1.37 &1.15 &1.24 &  \\\hline
	\end{tabular}
	
	\caption{Bounds $(\beta^{.5}_{C,D})_{C < D \leq 9}$ 
	for MNIST on {\bf reduced 20 dimensions}, as reached by the "1v", "3v" and "2v" models
	with 500 $(\times 1,2,3)$ binary variables.}
    \label{table.pair}
\end{table*}

	
\end{document}
