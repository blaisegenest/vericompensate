\section{Solution-Aware Scoring.}

\label{sec4}

In this section, we propose a novel {\em Solution-Aware Scoring} (SAS),
to evaluate accurately how opening a ReLU impacts the accuracy.
To do so, SAS considers explicitly the {\em difference} $\Delta(z)$ on the value of target node $z$ from opening $\ReLU(a)$, that is the difference for the value of $z$ between considering $\ReLU(a)$ exactly or using its LP relaxation $\LP(a)$, using the solution to a unique full LP call, which is resonnably fast to obtain. To compute $\Delta(z)$, intermediate $\Delta(b),\Delta(\hat{b})$ are computed inductively.
%Indeed, it is not rare that $z$ is sensitive to ReLU node $n$, and yet $\LP(n)$ already provides an accurate approximation of $\ReLU(n)$.
%In this case, usual heuristics would open $n$, while it would only improve the value of $z$ in a limited way.

Assume that we want to compute an upper bound for neuron $z$ on layer $\ell_z$.
We write $n < z$ if neuron {\color{blue} $n$ is} on a layer before $\ell_z$, and $n \leq z$ if $n< z$ or $n=z$. We denote ($\Sol\_\max_X^z(n))_{n \leq z}$ a solution of $\mathcal{M}_X$ maximizing $z$. In particular, $\Sol\_\max_X^z(z)$ is the maximum of $z$ under $\mathcal{M}_X$.

Consider $(sol(n))_{n \leq z} = (\Sol\_\max_\emptyset^z(n))_{n \leq z}$, a solution maximizing the value for $z$ when all ReLU use the LP relaxation.
%(this can be obtained very efficiently by {\em one} call to an LP solver).
Function
$\Improve\_\max^z(n)=$ $\sol(z) - \Sol\_\max_{\{n\}}^z(z)$, 
accurately represents how much opening neuron $n < z$ reduces the maximum computed for $z$
compared with using only LP. 
We have $\Improve\_\max^z(n)\geq 0$ as $\Sol\_\max_{\{n\}}^z$ fulfills all the constraints of 
$\mathcal{M}_\emptyset$, so $\Sol\_\max_{\{n\}}^z(z) \leq \sol(z)$.
Similarly, we define ($\Sol\_\min_\emptyset^z(n))_{n \leq z}$ and 
$\Improve\_\min^z(n)$. Calling MILP on $\mathcal{M}_{\{n\}}$ for every neuron $n \leq z$
would however be very time consuming when the number of neurons $a$ to evaluate is large.
{\color{blue} 
The main novelty of our SAS function is that it uses a (single) LP call to compute $(sol(n))_{n \leq z}$, with negligible runtime wrt the forthcoming  $\MILP_X$ call, and yet accurately approximates $\Improve\_\max^z(n)$ to choose a meaningful set $X$ of open nodes (Table \ref{tab:example1}).


\iffalse
That's why as far as we know, in competing heuristics to rank important nodes (e.g. \cite{BaB,huang2017safety,ferrari2022complete}), no call to solvers are made.
%Instead, we focus on the following $\Utility\_\max\nolimits^z(n)$ function upper bounding $\Improve\_\max^z(n)$. 
\fi





%\subsubsection*{Observation}

%The key of our formula is based on the following observation:



%Our observation (by experiments) is that \begin{align}
%	I_X \approx \sum_{b\in X} I_b.
%\end{align} Especially, if all neurons in $X$ are from one layer before $a$, then in %experiments, we observe that 

\iffalse
\begin{align*}
	|(I_X - \sum_{b\in X} I_b)/I_X| < 1\%. \ (\text{in experiments})
\end{align*} Even $X$ contains neurons from 3 layers before the target layer, in experiments, $I_X$ is still close to $\sum_{b\in X} I_b$.

Therefore, based on this observation, the question to choose $X$ is converted to compute $I_b$ for neurons $b$ in layers before the target layer. Our formula is to estimate the improvement of different individual neurons in different layers. For different layers, the formula will be different.  However, neither the observation in this subsection nor the formula in the next subsection has solid theoretical proof to show that they are very accurate. They are all based on experiments. 


In our algorithm, we will open neurons at most 3 layer3 before the target layer. So the formula will consists of three parts.


\subsubsection*{Compute the improvement of a single neuron}

\subsection*{One Layer before $z$}

\fi

%For all neurons $n$, let $\sol(n)=$$\Sol\_\max_\emptyset^z(n)$ be the value of neuron $a$
%in the solution of the LP instance $\Sol\_\max_\emptyset^z$ to maximize $z$.
%For one layer before the target layer, the formula is simple and most accurate. 
%To estimate $\Improve\_\max^z(a)$, 



%we define $\Utility\_\max^z(a)$, first for neurons $a$ one layer before $z$, by computing by how much the value of $z$ will change if $a$ is opened
%and other values remain the same - in particular, $value(\hat{a})=\ReLU(sol(a))$. We define:
%we first need to run $M^a_{\emptyset}$ to compute the upper bound of $a$  to obtain the solution data. Especially, we will read the values of $b$, before $\ReLU$ function and after $\ReLU$ function.

For a neuron $b$ on the layer before the layer $\ell_z$, we define:


\vspace{-0.4cm}
	$$\Utility\_\max\nolimits^z(b) = W_{bz} \times (\sol(\hat{b})- \ReLU(\sol(b)))$$
\vspace{-0.4cm}
	
	%In particular, if $\sol(\hat{a})=\ReLU(\sol(a))$, then we will have 
	%\begin{align*}
%		Utility^z(a) = 0.
%	\end{align*}

Consider $b$ with $W_{bz}<0$: to maximize $z$, the value of $\sol(\hat{b})$ is minimized, 
which is $\sol(\hat{b})=\ReLU(\sol(b))$ thanks to Proposition~\ref{LP}. 
Even if $z$ is sensitive to this ReLU $b$, the improvement of $b$ is 0.
Utility does not open it as $\Utility\_\max^z(b)=0$, whereas usual heuristics would.

Recall the rate $r(b)=\frac{\max(0,\UB(b))}{\max(0,\UB(b))-\min(0,\LB(b))} \in [0,1]$.
For a neuron $a$ two layers before $\ell_z$, 
$b$ denoting neurons in the layer $\ell$ just before $\ell_z$, 
we define:

\begin{align*}
	\Delta(\hat{a}) &= \ReLU(\sol(a))-\sol(\hat{a})\\
	\forall b \in \ell, \Delta(b) &= W_{ab}\Delta(\hat{a})\\
	\forall b \in \ell, \Delta(\hat{b}) &=
	\begin{cases}
		%\Delta(b),  & \text{if } W_{bz} > 0 \text{ and } \LB(b)\geq 0\\
		r(b)\Delta(b),%\frac{\UB(b)}{\UB(b)-\LB(b)}\Delta(b),  
		&\text{if }W_{bz} > 0 \\ % \text{ and } \LB(b)<0\\
		\max(\Delta(b),-\sol(b)),  &\text{if }  W_{bz} < 0 \text{ and } \sol(b)\geq0\\
		\max(\Delta(b)+\sol(b),0),  &\text{if }  W_{bz} < 0 \text{ and } \sol(b)<0		 
	\end{cases}\\
	\Utility\_\max\nolimits^z(a) &= -\sum_{b \in \ell} W_{bz} \Delta(\hat{b})
\end{align*}

%\begin{tikzpicture}[scale=1, >=stealth]
%	
%	% Draw axes
%	\draw[->] (-5,0) -- (4,0) node[right] {$x$};
%	\draw[->] (0,-1) -- (0,3) node[above] {$y$};
%	
%	% Draw ReLU function
%	\draw[line width=0.4mm, blue] (-3,0) -- (0,0);
%	\draw[thick, blue] (0,0) -- (2.5,2.5) node[below, shift={(0.5,-0.4)}] {$y = \ReLU(x)$};
%	\draw[thick, blue] (-3,0) -- (2.5,2.5) node[above, shift={(-0.5,0.4)}] {$y = \frac{\UB}{\UB-\LB} x-\frac{\UB\LB}{\UB-\LB}$};
%	
%%	% Add labels
%%	\draw[dashed] (2,0) -- (2,2) -- (0,2); % Optional grid
%%	\node[below left] at (0,0) {$0$};
%%	
%%	% Add tick marks
%%	\foreach \x in {1,2}
%%	\draw[shift={(\x,0)}] (0,0.1) -- (0,-0.1) node[below] {\x};
%%	\foreach \y in {1,2}
%%	\draw[shift={(0,\y)}] (0.1,0) -- (-0.1,0) node[left] {\y};
%	
%\end{tikzpicture}

%We will show with a more general definition that $0 \leq \Improve\_\max^z(a) \leq \Utility\_\max^z(a)$ in Prop.~\ref{prop2}. 

%Informally, $\Delta(\hat{a}), \Delta(b), \Delta(\hat{b})$ approximate the improvement on the accuracy of $\hat{a}, b, \hat{b}$ when computing $\ReLU(a)$ using the exact MILP encoding instead of LP. 

Notice that $\Utility$ does not need to consider bias explicitly, unlike $s_{FSB},s_{SR}$,
as they are already accounted for in the solution considered. 
There are two main differences with $s_{FSB}$: 
First, $\Delta(\hat{a})$ is defined exactly, with $\ReLU(\sol(a))-\sol(\hat{a})$,
whereas the corresponding $\Delta_{FSB}(\hat{a})$ is approximated as $\LB(a) r(a)$, 
which is only an upper bound of $\ReLU(\sol(a))-\sol(\hat{a})$.
More importantly, the corresponding $\Delta_{FSB}(\hat{b})$ is always $r(b) \Delta(b)$, whereas SAS adapts to the case when $W_{bz}<0$, correctly considering the 0 phase of $ReLU(b)$ when $\sol(b)\geq 0$, and the ID phase when $\sol(b)<0$.
Assuming that the solution to maximize $z$ does not change when opening $\ReLU(a)$
on neurons independant of $a$, then $\Utility$ is exact. This is not the case for $s_{FSB}$ because of the two previously mentionned cases. Further, even when the global solution changes, we can show that $\Utility$ is a safe overapproximation, which does not hold for $s_{FSB},s_{SR}$ (because of the case $sol(b) > 0$):
}

\begin{proposition}
	\label{prop2}
		$0 \leq \Improve\_\max^z(a) \leq \Utility\_\max^z(a)$. 
\end{proposition}


Thus, $\Utility\_\max^z(a)$ can be used to approximate $\Improve\_\max^z(a)$. 
In particular, for all nodes $a$ with $\Utility\_\max\nolimits^z(a)=0$, 
we are sure that this node is not having any impact on $\Sol\_\max_{\{a\}}^z(z)$. 
%This is one striking difference (but not the only one) with choosing utility based on 
%$|W_{az}|$ \cite{DivideAndSlide}.

	
%	
%	Similarly, let $\sol(b)$ be the value of $b$ in the LP solution of lower bound of $a$, and $\sol(\hat{b})$ be the value of $\hat{b}$. Then the formula to estimate improvement of lower bound of $b$ is: \begin{align*}
%		Improve\_min^z(b) \approx -W_{ba}(\sol(\hat{b})-\ReLU(\sol(b))).
%	\end{align*}
	
%To explain the formula, we use upper bound and the case that $W_{ba} > 0$ as an example. To compute the upper bound of $a$, $\hat{b}$ should be as large as possible. In the LP model, for fixed $\sol(b)$, the upper bound of $\hat{b}$ may be larger than $\ReLU(\sol(b))$. This is because in LP model, the upper bound of $\sol(\hat{b})$ is decided by the linear approximation rather than $\ReLU$ function. So, when neuron $b$ is open, if $\sol(b)$ do not change, then the upper bound of $a$ will be improved because the value of $\sol(\hat{b})$ will be lower to $\ReLU(\sol(b))$.
% 			
%Of course changing other variables may also effect the upper bound, but our experiments show that, the change from $\sol(\hat{b})$ to $\ReLU(\sol(b))$ is the major part of improvement. 


 



	
	\begin{proof}
    Consider $\sol'(n)_{n \leq z}$ with
	$\sol'(n)=\sol(n)$ for all $n \notin \{z,\hat{a}\} \cup \{b,\hat{b} \mid b \in \ell\}$. In particular,  $\sol'(a) = \sol(a)$.
	Now, define $\sol'(\hat{a}) = \ReLU(\sol(a))$. 
	That is, $\sol'(\hat{a})$ is the correct value for $\hat{a}$, obtained if we open neuron $a$, compared to the LP abstraction for $\sol(\hat{a})$.
	We define $\sol'(b)=\sol(b)+\Delta(b)$ and 
	$\sol'(\hat{b})=\sol(\hat{b}) + \Delta(\hat{b})$.
	Last, $\sol'(z)=\sol(z) + \sum_{b \in \ell} W_{bz} \Delta(\hat{b})$.
	We will show:
	\begin{equation}
		\label{eq12}
		(\sol'(n))_{n \leq z} \text{ satisfies the constraints in } \mathcal{M}_{\{a\}}
	\end{equation} 
	This suffices to conclude: as
	$\sol'(z)$ is a solution of $\mathcal{M}_{\{a\}}$, it is smaller or equal to the maximal solution: $\sol'(z) \leq$ $\Sol\_\max_{\{a\}}^z(z)$. That is, 
	$\sol(z)-\sol'(z) \geq \sol(z) -$ $\Sol\_\max_{\{a\}}^z(z)$, i.e. 
	$ \Utility\_\max^z(a) \geq \Improve\_\max^z(a)$.
	In particular, we have that $\Utility\_\max^z(a) \geq 0$, which was not obvious from the definition.

	Finally, we show (\ref{eq12}). First, opening $a$ changes the value of $\hat{a}$ from
	$\sol(\hat{a})$ to $\ReLU(\sol(a)) = sol(\hat{a}) + \Delta(a)$, 
	and from $sol(b)$ to $sol(b) + \Delta(b)$.
	The case of $\Delta(\hat{b})$ is the most interesting:
	If $W_{bz}>0$, then according to Proposition \ref{LP}, the LP solver
sets $\sol(\hat{b}) = \sol(b) \frac{\UB(b)}{\UB(b)-\LB(b)} +$ Cst to maximize $z$.
Changing $b$ by $\Delta(b)$ thus results in changing $\sol(\hat{b})$ by 
$\frac{\UB(b)}{\UB(b)-\LB(b)}\Delta(b)$.
If $W_{bz}\leq0$, then the LP solver sets $\sol(\hat{b})$ to the lowest possible value to maximize $z$, which happens to be $\ReLU(b)$ according to Proposition \ref{LP}.
If $\sol(b) < 0$, then we have $\sol(\hat{b})=\ReLU(b)=0$ and opening $a$ change the 0 value only if $\sol(b)+\Delta(b)>0$. If $\sol(b) > 0$, then 
$\sol(\hat{b})=\ReLU(\sol(b))=\sol(b)$, and the change to $\hat{b}$ will be 
the full $\Delta(b)$, unless $\Delta(b) < -\sol(b) < 0$ in which case it is 
$-\sol(b)$.
		\end{proof}


	\iffalse

	If we open $a$ without changing its value $\sol(a)$, then the change $\Delta(\hat{a})$ in the weight of $\hat{a}$ is 
$\Delta(\hat{a})=\ReLU(\sol(a)) - \sol(\hat{a}) \leq 0$ as above. Its impact on $z$ is no more direct with $W_{az}$, but it is through $\ell$. 
We let $\Delta(b) = W_{ab}\Delta(\hat{a})$ for all $b \in \ell$.
Based on Proposition \ref{LP}, we can evaluate the impact 
$\Delta(\hat{b})$ of opening $a$ on the value of each $\hat{b}$, by using the upper and lower bound $\UB(b),\LB(b)$:

	\begin{align*}
		&\Delta(\hat{b}) =
		\begin{cases}
			\frac{\UB(b)}{\UB(b)-\LB(b)}\Delta(b),  &\text{if }W_{bz} > 0\\
			\max(\Delta(b),-\sol(b)),  &\text{if }  W_{bz} < 0 \text{ and } \sol(b)\geq0\\
			\max(\Delta(b)+\sol(b),0),  &\text{if }  W_{bz} < 0 \text{ and } \sol(b)<0		 
		\end{cases}
		\end{align*}


Indeed, if $W_{bz}>0$, then according to Proposition \ref{LP}, the LP solver
sets $\sol(\hat{b}) = \sol(b) \frac{\UB(b)}{\UB(b)-\LB(b)} +$ Cst to maximize $z$.
Changing $b$ by $\Delta(b)$ thus results in changing $\sol(\hat{b})$ by 
$\frac{\UB(b)}{\UB(b)-\LB(b)}\Delta(b)$.
If $W_{bz}\leq0$, then the LP solver sets $\sol(\hat{b})$ to the lowest possible value to maximize $z$, which happens to be $\ReLU(b)$ according to Proposition \ref{LP}.
If $\sol(b) < 0$, then we have $\sol(\hat{b})=\ReLU(b)=0$ and opening $a$ change the 0 value only if $\sol(b)+\Delta(b)>0$. If $\sol(b) > 0$, then 
$\sol(\hat{b})=\ReLU(\sol(b))=\sol(b)$, and the change to $\hat{b}$ will be 
the full $\Delta(b)$, unless $\Delta(b) < -\sol(b) < 0$ in which case it is 
$-\sol(b)$. We then set:

%We then define 
%\begin{align*}
%	Utility\_max^z(a) = (\sol(\hat{a})-\ReLU(\sol(a)))\sum_b k(b).
%\end{align*}

$$ \Utility\_\max\nolimits^z(a) = -\sum_{b \in \ell} W_{bz} \Delta(\hat{b})$$
\fi

We can proceed inductively in the same way to define $\Utility\_\max^z(a)$ for deeper neurons $a$.

While $\Utility$ is a priori a more accurate way to select neurons to open in partial MILP (that will be verified in Table \ref{tab:example}), global scoring functions have two advantages: first, they are faster to compute (a unique backpropagation is sufficient, while SAS needs one full LP call to recover the solution, plus a forward propagation for each node $a$). This is not an issue for selecting nodes for partial MILP, as there is a single selection step. But this would be potentially more problematic for choosing branching nodes. A trade off could be to precompute $s_{FSB}$ and only on the most promising nodes run $\Utility$. 
The second shortcoming is the other side of the coin to consider improvement local to a solution: in the case of branching, one of the branch imposes to be far from the solution, 
while global scoring is as (in)accurate for both branches. We will verify that using GS for ordering nodes after the SAS selection is actually more efficient, for this very reason.

\iffalse
We use the following formula for general cases, i.e. possibly more than 3 layers. We use $a$ to denote the source node, and use $b,c$ to denote that $b$ is in one layer before $c$. 

\begin{align*}
	\Delta(\hat{a}) &= \ReLU(\sol(a))-\sol(\hat{a})\\
	\Delta_0(b) &= \sum_{b} W_{bc}\Delta(\hat{b})\\
		\Delta(c) &= \min(\max((\sol(c)+\Delta_0(c),\LB(c))),\UB(c))-\sol(c)\\
		\Delta(\hat{c}) &=
		\begin{cases}
		\Delta(c)\frac{\LB(c) - \sol(\hat{c})}{\LB(c) - \sol(c)},  &\text{if } \LB(c)< \sol(c) < 0\\
		\Delta(c)\frac{\UB(c) - \sol(\hat{c})}{\UB(c) - \sol(c)},  &\text{if }  0< \sol(c) < \UB(c)\\
			\Delta(c),  &\text{else } 	 
		\end{cases}
\end{align*}


\subsubsection*{Three Layer before  $z$} 

Suppose $a$ is a neuron in three layers before $z$, we use $b$ to denote neurons in two layer before $z$ and $c$ to denote neurons in one layer before $z$ and. 

This formula is based on previous subsection but more complex. In some network, running this formula may cost too much time. 

The key problem is how to compute the coefficient $k$ for neurons in two layers before the target layer. To do this, we may use the values in the solution of LP model as follows:

\begin{definition}\label{3layer}
Let $\UB$ and $\LB$ denote the precomputed upper bounds and lower bounds used in building MILP models. We define the following function $h$ for all neurons $b$ in two layers before the target neuron $z$ as follows:
	\begin{align}
		&v_0 = \sol(\hat{b}), v_1 = \ReLU(\sol(b)), v_2 = \frac{\UB(b)\sol(b)-\UB(b)\LB(b)}{\UB(b)-\LB(b)}\\
		&h(b) =
		\begin{cases}
			\frac{v_0-v_1}{v_2-v_1}, & \text{if } v_2-v_1 > 0\\
			0.5, & \text{otherwise.}
		\end{cases}
	\end{align} 
\end{definition} 

\begin{definition}
	Continue the assumption in Definition \ref{3layer}. We define function $D$ layer by layer.
	
	First, $\Delta(a) = \ReLU(\sol(a))-\sol(\hat{a})$.
	
To compute $\Delta(b)$ for neurons $b$ in two layer before $z$, we define \begin{align}
	&u_0 = \max(\LB(b),\min(\UB(b),  \sol(b)+\Delta(a)W_{ab}))\\
	&u_1 = \begin{cases}
		\ReLU(u_0)+h(b)(\frac{\UB(c)u_0-\UB(b)\LB(b)}{\UB(b)-\LB(b)}-\ReLU(u_0)), & \text{if }\LB(b) < 0\\
	u_0, & \text{if }  \LB(b) \geq 0
	\end{cases}\\
	&\Delta(b) = u_1-\sol(\hat{b})
\end{align}
	
	To compute $\Delta(c)$ for neurons $c$ in one layer before $z$, we define 
	\begin{align}
		&w_0 = \sum_b \Delta(b)W_{bc}\\
		&w_1 = \min(\UB(c),\sol(c)+w_0)\\		
		&\Delta(c) =
		\begin{cases}
			w_1-\sol({c}), & \text{if }W_{cz} > 0 \text{ and } \LB(c)\geq 0\\
		k(c)(w_1-\sol({c})), & \text{if }W_{cz} > 0 \text{ and } \LB(c)< 0\\
		\ReLU(w_1)-\sol(\hat{c})	, & \text{if }  W_{cz} < 0
		\end{cases}\\
		&\Delta(z) = \sum_c \Delta(c)W_{cz}\\
		&\Utility\_\max^z(a) = -\Delta(z)
	\end{align}
\end{definition}
		
\fi