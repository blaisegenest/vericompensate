\section{Value Abstraction for DNN verification}

In this section, we describe different value (over-)abstractions on $\vz$ that are used by efficient algorithms to certify robustness around an input $\vx$. Over-abstractions of values include all values for $\vz$ in the neighbourhood of $\vx$, and thus a certificate for safety in the over-abstraction is a proof of safety for the original input $\vx$.

\subsection{The Box Abstractions}


\definecolor{applegreen}{rgb}{0.55, 0.71, 0.0}



The concept of value abstraction involves calculating upper and lower bounds for the values of certain neurons in a Deep Neural Network (DNN) when inputs fall within a specified range. This approach aims to assess the network's robustness without precisely computing the values for every input within that range.

Firstly, it's important to note that weighted sums represent a linear function, which can be explicitly expressed with relative ease. However, the ReLU (Rectified Linear Unit) function presents a challenge in terms of accurate representation. Although ReLU is a relatively straightforward piecewise linear function with two modes (one for $x<0$ and another for $x \geq 0$), it is not linear. The complexity arises when considering the compounded effects of the ReLU function across the various layers of a ReLU DNN. It's worth noting that representing $\ReLU(x)$ precisely is feasible when $x$ is "{\em stable}", meaning it's consistently positive or consistently negative, as there's only one linear mode involved in each scenario. Consequently, the primary challenge lies in addressing "{\em unstable}" neurons, where the linearity of the function does not hold consistently.


\begin{figure}[t!]
	\centering
	\begin{tikzpicture}
		
		\node[circle, draw= applegreen, thick, minimum width = 20,
		minimum height = 20] (input1) {$n_1$};

		
		\node[circle, draw= applegreen, thick, minimum width = 20,
		minimum height = 20] (input2) at ($(input1) + (0,-1.5)$) {$n_2$};
		
		% Hidden layers
		
		\node (hidden10) at ($(input1) + (2.5,0.6)$) {$0$};
		
		\node (hidden20) at ($(input1) + (2.5,-1.5-0.6)$) {$0$};
		
		\node (hidden50) at ($(input1) + (7.5,0.6)$) {$0$};
		
		\node (hidden60) at ($(input1) + (7.5,-1.5-0.6)$) {$1.5$};
		
		
		\node[circle, draw= blue, thick, minimum width = 20,
		minimum height = 20] (hidden1) at ($(input1) + (2.5,0)$) {$n_3$};
		\node[circle, draw= blue, thick] (hidden2) at ($(input1) + (2.5,-1.5)$) {$n_4$};
		
		\node[circle, draw= blue, thick, minimum width = 20,
		minimum height = 20] (hidden3) at ($(input1) + (5,0)$){$\hat{n}_3$};
		\node[circle, draw= blue, thick] (hidden4) at ($(input1) + (5,-1.5)$) {$\hat{n}_4$};
		
		
		\node[circle, draw= red, thick, minimum width = 20,
		minimum height = 20] (hidden5) at ($(input1) + (7.5,0)$){$n_5$};
		\node[circle, draw= blue, thick] (hidden6) at ($(input1) + (7.5,-1.5)$) {$n_6$};
		
		
		% Output layer
		\node[circle, draw= blue, thick, minimum width = 20,
		minimum height = 20] (output1) at ($(input1) + (10,0)$){$\hat{n}_5$};
		
		\node[circle, draw= blue, thick, minimum width = 20,
		minimum height = 20] (output2) at ($(input1) + (10,-1.5)$){$\hat{n}_{6}$};
		
		

		\draw[->,thick] ($(input1) + (-1.5,0)$) -- (input1) node[midway, above] {$[-1,1]$};
		
		\draw[->,thick] ($(input1) + (-1.5,-1.5)$) -- (input2) node[midway, above] {$[-1,1]$};
		
		
		
		\draw[->,thick] (input1) -- (hidden1) node[near start, above] {$1$};
		\draw[->,thick] (input1) -- (hidden2)node[near start, above] {$1$};
		
		\draw[->, thick] (input2) -- (hidden1) node[near start, below] {$1$};
		\draw[->, thick] (input2) -- (hidden2)node[near start, below] {$-1$};
		
		
		
		
		
		\draw[->, thick] (hidden1) -- (hidden3) node[midway, above] {$\max(n_1,0)$};
		\draw[->, thick] (hidden2) -- (hidden4) node[midway, above] {$\max(n_2,0)$};
		
		
		
		
		
		\draw[->,thick] (hidden3) -- (hidden5) node[near start, above] {$1$};			
		\draw[->,thick] (hidden3) -- (hidden6) node[near start, above] {$0$};
		
		\draw[->,thick] (hidden4) -- (hidden5)node[near start, below] {$2$};
		\draw[->,thick] (hidden4) -- (hidden6)node[near start, below] {$1$};
		
		
		
		
		\draw[->,thick] (hidden5) -- (output1) node[midway, above] {$\max(n_5,0)$};
		\draw[->,thick] (hidden6) -- (output2) node[midway, above] {$\max(n_6,0)$};
		

		% Connections
		
		
	\end{tikzpicture}
	\caption{A DNN. Every neuron is separated into 2 nodes, 
	$n$ pre- and $\hat{n}$ post-ReLU activation.} 
	%The pair of paths 	$({\color{applegreen}n_2 n_3 n_5},{\color{red}n_2 n_4 n_5})$ is {\em compensating} 	(weights ${\color{applegreen}1},{\color{red}-2}$).}
	\label{fig1}
\end{figure}




Consider the simpler abstraction, termed ``Box abstraction", {\color{blue} recalled e.g. in \cite{deeppoly}}: it inductively computes the bounds for each neuron in the subsequent layer independently. This is achieved by considering the weighted sum of the bounds from the previous layer, followed by clipping the lower bound at $\max(0,$ lower bound$)$ to represent the ReLU function, and so forth. 
For all $i \geq 3$, define $x_i=\val_{\vx}(n_i)$, where $\vx=(x_1,x_2)$.
Taking the DNN example from Fig \ref{fig1}, assume $x_1,x_2 \in [-1,1]$. This implies that $x_3,x_4 \in [-2,2]$. After applying the ReLU function, $\hat{x}_3,\hat{x}_4$ are constrained to $[0,2]$, leading to $x_5 \in [0,6]$ and $x_6 \in [0,2]$. 
The bounds for $n_1, \ldots, n_4$ are exact, meaning for every $\alpha$ within the range, an input $\vy$ can be found such that $\val_{\vy}(n_i)=\alpha$. However, this precision is lost from the next layer (beginning with $n_5, n_6$) due to potential dependencies among preceding neurons. For example, it is impossible for $x_5=\Val_{\vx}(n_5)$ to reach $6$, as it would necessitate both $x_3=2$ and $x_4=2$, which is not possible at the same time as 
$x_3=2$ implies $x_1=x_2=1$ and $x_4=2$ implies $x_2=-1$ (and $x_1=1$), a contradiction.

{\color{blue} In \cite{DBLP_Ehlers17, deeppoly} and others, the {\em triangular abstraction} was proposed:
\begin{align}
	\ReLU(x) = max (0,x) \leq \hat{x} \leq \UB(n) \frac{x-\LB(n)}{\UB(n)-\LB(n)} 
	\label{eq:deeppoly}
\end{align} 
It has two lower bounds (the 0 and identity functions), and one upper bound. 
DeepPoly \cite{deeppoly} chooses one of the two lower bounds for each neuron $x$, giving rise to a greedy quadratic-time algorithm to compute very fast an abstraction of the value of $\hat{x}$ (but not that accurately).
}




\subsection{MILP, LP and partial MILP encodings for DNNs}

At the other end of the spectrum, we find the Mixed Integer Linear Programming (MILP) value abstraction, which is a complete (but inefficient) method. 
Consider an unstable neuron $n$, whose value $x \in [\LB(n),\UB(n)]$ with $\LB(n)<0<\UB(n)$.
The value $\hat{x}$ of $\ReLU(x)$ can be encoded exactly in an MILP formula with one 
integer (actually even binary) variable $a$ valued in $\{0,1\}$, using constants $\UB(n),\LB(n)$ with 4 constraints \cite{MILP}:

\vspace{-0.4cm}
\begin{equation}\quad \hat{x} \geq x \quad \wedge \quad \hat{x} \geq 0, \quad \wedge \quad \hat{x} \leq \LB(n) \cdot a \quad \wedge \quad \hat{x} \leq x-\UB(n) \cdot (1-a)
\label{eq11}
\end{equation}

For all $x \in [\LB(n),\UB(n)] \setminus 0$, there exists a unique solution $(a,\hat{x})$ that meets these constraints, with $\hat{x}=\ReLU(x)$ \cite{MILP}. The value of $a$ is 0 if $x < 0$, and 1 if $x>0$, and can be either if $x=0$. This encoding approach can be applied to every (unstable) ReLU node, and optimizing its value can help getting more accurate bounds. However, for networks with hundreds of {\em unstable} nodes, the resulting MILP formulation will contain numerous integer variables and generally bounds obtained will not be accurate, even using powerful commercial solvers such as Gurobi.

MILP instances can be linearly relaxed into LP over-abstraction, where variables originally restricted to integers in $\{0,1\}$ (binary) are relaxed to real numbers in the interval $[0,1]$, while maintaining the same encoding. As solving LP instances is polynomial time, this optimization is significantly more efficient. However, this efficiency comes at the cost of precision, often resulting in less stringent bounds. This approach is termed the {\em LP abstraction}.

In this paper, we propose to use {\em partial MILP}, to get interesting trade-offs between accuracy and runtime: 
for a set of unstable neurons $X$, we denote by MILP$_X$ the MILP encoding where variables encoding $X$ are binary, and other variables are linear variables using the LP relaxation. We say that nodes in $X$ are {\em opened}. 
To further limit the number of binary variables needed for a given accuracy, we devise the same iterative approach as the box abstraction or DeepPoly \cite{deeppoly}, computing lower 
and upper bounds $\LB(n),\UB(n)$ for neurons $n$ of a layer, that are used when computing values of the next layer, thus necessitating less variables from previous layers. 

The crucial factor in such an approach is to {\em select} few opened ReLU nodes in $X$ which are the most important for the accuracy. An extreme strategy was adopted in \cite{DivideAndSlide}, where only ReLU nodes of the immediate previous layer can be opened, and the measure to choose ReLU $a$ when computing the bounds for neuron $b$ was to consider $|W_{ab}| (\UB(a)-\LB(a))$. 
To obtain a more accurate measure, that 
%can also scale to deeper layers and 
is not limited to open nodes from the immediate previous layer, 
{\color{blue} we invoke a folklore result \cite{alessandro} on the LP relaxation of (\ref{eq11}), for which we provide a direct and explicit proof:}

%Consider an unstable neuron $n$, that is $\LB(n) < 0 < \UB(n)$. 
%Let $x$ represent the input value of $n$, hence with 
%$\LB(n) < x < \UB(n)$, and $\hat{x}$ representing $\ReLU(x)$. 
%We have:




\begin{proposition}
	\label{LP}
	The LP relaxation of (\ref{eq11}) is equivalent with the triangular abstraction 	(\ref{eq:deeppoly}).
\end{proposition}
 
%{\color{blue}
%To the best of our knowledge, we could not find such an equivalence stated in the bibliography, although the proof is simple. This result will be instrumental for our Utility function (Section \ref{sec4}).}




\begin{proof}
	Consider an unstable neuron $n$, that is $\LB(n) < 0 < \UB(n)$.
The lower bound on $\hat{x}$ is simple, as $\hat{x} \geq 0 \wedge \hat{x} \geq x$ is immediately equivalent with $\hat{x} \geq \ReLU(x)$.

We now show that the three constraints 
$\hat{x} \leq \UB(n) \cdot a \, \wedge \, \hat{x} \leq x-\LB(n) \cdot (1-a) \, \wedge \, a \in [0,1]$ translates into $\hat{x} \leq \UB(n) \frac{x-\LB(n)}{\UB(n)-\LB(n)}$. 
We have $\hat{x}$ is upper bounded by $max_{a \in [0,1]} (min(\UB(n) \cdot a, x - \LB(n) (1-a)))$, and this bound can be reached. Furthermore, using standard function analysis tools (derivative...), we can show that the function $a \mapsto \min(\UB(n) \cdot a, x - \LB(n) (1-a))$ attains its maximum when $\UB(n) \cdot a = x - \LB(n) (1-a)$, leading to the equation $(\UB(n) - \LB(n)) a = x - \LB(n)$ and consequently $a = \frac{x - \LB(n)}{\UB(n)-\LB(n)}$. This results in an upper bound $\hat{x} \leq \UB(n) \frac{x - \LB(n)}{\UB(n)-\LB(n)}$, which can be reached, hence the equivalence.
\end{proof}


