

Deep neural networks (DNNs for short) have demonstrated remarkable capabilities, achieving human-like or even superior performance across a wide range of tasks. However, their robustness is often compromised by their susceptibility to input perturbations \cite{szegedy}. This vulnerability has catalyzed the verification community to develop various methodologies, each presenting a unique balance between completeness and computational efficiency \cite{Marabou,Reluplex,deeppoly}. This surge in innovation has also led to the inception of competitions such as VNNComp \cite{VNNcomp}, which aim to systematically evaluate the performance of neural network verification tools. While the verification engines are generic, the benchmarks usually focus on local robustness, i.e. given a DNN, an image and a small neighbourhood around this image, is it the case that all the images in the neighbourhood are classified in the same way. 
For the past 5 years, VNNcomp has focused on rather easy instances, that can be solved within tens of seconds (the typical hard time-out is 300s). For this reason, DNN verifiers in the past years have mainly focused on optimizing for such easy instances. Among them, NNenum \cite{nnenum}, Marabou \cite{Marabou,Marabou2}, and PyRAT 
\cite{pyrat}, respectively 4th, 3rd and 2sd of the last VNNcomp'24 \cite{VNNcomp24}
and 5th, 2sd and 3rd  of the VNNcomp'23 \cite{VNNcomp23}; MnBAB \cite{ferrari2022complete}, 2sd in VNNcomp'22 \cite{VNNcomp22}, built upon ERAN \cite{deeppoly} and PRIMA \cite{prima}; and importantly, $\alpha,\beta$-CROWN \cite{crown,xu2020fast}, the winner of the last 4 VNNcomp, benefiting from branch-and-bound based methodology \cite{cutting,BaB}.
We will thus compare mainly with $\alpha,\beta$-CROWN experiments as gold standard in the following\footnote{GCP-CROWN \cite{cutting} is slightly more accurate than $\alpha,\beta$-CROWN on the DNNs we tested, but necessitates IBM CPLEX solver, which is not available to us}.

\subsection{Related Work} 
