\documentclass{llncs}
\usepackage{hyperref}
\usepackage{url}
\pagestyle{plain}
\usepackage{threeparttable}
\input{math_commands.tex}
%\usepackage[latin9]{inputenc}
%\usepackage[T1]{fontenc}
\usepackage{float}
\usepackage{wrapfig}
\usepackage{lineno}
\usepackage{amsmath}
\usepackage{amssymb}
\usepackage{graphicx}
\usepackage{subcaption}
\usepackage{tabularx}
\usepackage{cases}
\captionsetup{compatibility=false}
% \usepackage{esint}
\usepackage{array}
\usepackage{epstopdf}
\usepackage{placeins}
\usepackage{pgfplots}
\usepackage{url}
\usepackage{tikz}
\usepackage{calc}
\usepackage{array}
\usepackage[linesnumbered,ruled,vlined]{algorithm2e}
\usetikzlibrary{positioning, arrows.meta,calc}

%%%%%%%%%%%%%%%%%%%%%%%%%%%%%%%
%%%%%%%%%%%%%%%%%%%%%%%%%%%%%%%


\newcommand{\vW}{\boldsymbol{W}}


\newcommand{\val}{{\textrm{value}}}
\newcommand{\Val}{{\textrm{value}}}
\newcommand{\MILP}{{\textrm{MILP}}}
\newcommand{\LP}{{\textrm{LP}}}
\newcommand{\Improve}{\mathrm{Improve}}
\newcommand{\Utility}{\mathrm{SAS}}
\newcommand{\Sol}{\mathrm{Sol}}
\newcommand{\sol}{\mathrm{sol}}

\newcommand{\UB}{\mathrm{UB}}
\newcommand{\LB}{\mathrm{LB}}
\newcommand{\ub}{\mathrm{ub}}
\newcommand{\lb}{\mathrm{lb}}
\newcommand{\B}{\mathrm{B}}
\usepackage{amsmath, amssymb, amsfonts}


\newcommand{\ReLU}{\mathrm{ReLU}}

\newcommand{\CMP}{{\textrm{CMP}}\ }

\newcommand{\fix}{\marginpar{FIX}}
\newcommand{\new}{\marginpar{NEW}}
\newcommand{\toolname}{Hybrid MILP}


\title{Solution-aware vs global ReLU selection: \\
partial MILP strikes back for DNN verification}
\date{}


\begin{document}

\begin{definition}
	Suppose we have a function $f$ from $\mathbb{R}^n$ to $\mathbb{R}^m$ and $||$ is the $L_\infty$ norm. 
	
	For a number $\epsilon > 0$, $K_\epsilon$ is the smallest number such that for any $d\geq \epsilon$, any input $x,y$.
	\begin{align*}
	 |x-y| \leq d \implies |f(x)-f(y)| \leq dK_\epsilon 
	\end{align*}
\end{definition}



\begin{proposition}
	
	Continue above definition. Suppose for a number $\epsilon$ we have $K_\epsilon$, then for any number $ \epsilon N\leq d\leq \epsilon(N+1)$ for an integer $N \geq 1$, we have that $$K_d \leq K_\epsilon\frac{N+1}{N}.$$
\end{proposition}

\begin{proof}
By assumption, $d \in [\epsilon N  , \epsilon(N+1)]$ . 

	For any $x, y$ that $|x-y| \leq d$, we have $|x-y| \leq \epsilon(N+1) $. Then we can divide the line segment between $x, y$ into $N+1$ pieces: $x_0 = x, x_1, x_2, \cdots, x_{N+1} = y$ such that $|x_i-x_{i+1}| \leq \epsilon$. Then we can apply the definition of $K_\epsilon$ for each pieces. 
	
	Therefore, we have that, for any two inputs $x,y$:
		\begin{align*}
		|x-y| \leq d \implies |f(x)-f(y)| \leq  \epsilon(N+1) K_\epsilon
	\end{align*}
	
	Hence by the definition, we have that $$K_d <= \dfrac{\epsilon(N+1) K_\epsilon }{d}<=\dfrac{\epsilon(N+1)K_\epsilon}{\epsilon N} = K_\epsilon\frac{N+1}{N}.$$
	
	This is what we want to show.
	

\end{proof}


\begin{corollary}
	\begin{enumerate}
		\item Continue the definition. Suppose for a number $\epsilon$ we have $K_\epsilon$, then for any number $d\geq \epsilon N$ for an integer $N \geq 1$, we have that $$K_d \leq K_\epsilon\frac{N+1}{N}.$$
		
		\item 	If we care about $\epsilon$ and compute the Lipschitz constant of $0.5 \epsilon$ and get $K$, then we can use Lipschizt constant $\leq K\frac{2+1}{2}$ for general larger $\epsilon$.
		\item 	Similarly, if we compute Lipschitz constant for any $\epsilon$ and get $K$, then we can use Lipschitz constant $K\frac{1+1}{1}$ for any larger $\epsilon$.
	\end{enumerate}
\end{corollary}

\begin{proof}
	All three are simply application of the proposition.
\end{proof}

\begin{proposition}
	
\end{proposition}


\end{document}


