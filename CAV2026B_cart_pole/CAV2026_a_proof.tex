\documentclass[runningheads]{llncs}
\usepackage[T1]{fontenc}

\usepackage[hyphens]{url}  % DO NOT CHANGE THIS
\usepackage{graphicx} % DO NOT CHANGE THIS
\frenchspacing  % DO NOT CHANGE THIS
\usepackage{algorithm}
\usepackage{algorithmic}
\pagestyle{plain}
\usepackage{threeparttable}
\input{math_commands.tex}
\usepackage{lineno}
\usepackage{subcaption}
\usepackage{tabularx}
\usepackage{cases}
\captionsetup{compatibility=false}
\usepackage{epstopdf}
\usepackage{placeins}
\usepackage{pgfplots}
\usepackage{tikz}
\usepackage{calc}
\usepackage{array}
%\usepackage[linesnumbered,ruled,vlined]{algorithm2e}
\usetikzlibrary{positioning, arrows.meta,calc}
\usepackage{newfloat}
\usepackage{listings}
\DeclareCaptionStyle{ruled}{labelfont=normalfont,labelsep=colon,strut=off} % DO NOT CHANGE THIS
\lstset{%
	basicstyle={\footnotesize\ttfamily},% footnotesize acceptable for monospace
	numbers=left,numberstyle=\footnotesize,xleftmargin=2em,% show line numbers, remove this entire line if you don't want the numbers.
	aboveskip=0pt,belowskip=0pt,%
	showstringspaces=false,tabsize=2,breaklines=true}
\floatstyle{ruled}
\newfloat{listing}{tb}{lst}{}
\floatname{listing}{Listing}
%\newtheorem{proposition}{Proposition}
%\newtheorem{definition}{Definition}
\newcommand{\vW}{\boldsymbol{W}}
\newcommand{\val}{{\textrm{value}}}
\newcommand{\Val}{{\textrm{value}}}
\newcommand{\MILP}{{\textrm{MILP}}}
\newcommand{\LP}{{\textrm{LP}}}
\newcommand{\Improve}{\mathrm{Improve}}
\newcommand{\Utility}{\mathrm{SAS}}
\newcommand{\Sol}{\mathrm{Sol}}
\newcommand{\sol}{\mathrm{sol}}
\newcommand{\UB}{\mathrm{UB}}
\newcommand{\LB}{\mathrm{LB}}
\newcommand{\ub}{\mathrm{ub}}
\newcommand{\lb}{\mathrm{lb}}
\newcommand{\B}{\mathrm{B}}
\usepackage{amsmath, amssymb, amsfonts}
\newcommand{\ReLU}{\mathrm{ReLU}}
\newcommand{\CMP}{{\textrm{CMP}}\ }
\newcommand{\fix}{\marginpar{FIX}}
\newcommand{\new}{\marginpar{NEW}}
\newcommand{\toolname}{Hybrid MILP}

\usepackage{bibentry}


\begin{document}
	
	\begin{definition}
		A system $S$ is a 5-tuple $(X,C,succ, U, I)$ that \begin{itemize}
			\item $X\subseteq \mathbb{R}^n$ is the full space;
			\item The control set $C$ is a set and the successor $succ$ is a function from $X\times C$ to $2^X$;
			\item $U\subseteq X$ is the unsafe region, and nonempty set $I\subseteq X$ disjoint from $U$;
		\end{itemize}
		
		The system of $S$ under perturbation of at most $\delta$, $S_\delta$, is a system with the same $(X,U,C,I)$ but a $succ_\delta$ function defined as follows: \begin{align*}
		succ_\delta(x,c) &= B_\delta(succ(x,c))\\
		&= \{z:\exists y\in succ(x,c)\ \ d(y,z)<\delta\}.
		\end{align*} Here $d$ can be an arbitrary kind of distance ($L_2, L_\infty$ and so on).
	\end{definition}
	
	Definition of policy: 
	
	\begin{definition} 
		 \begin{itemize}
			\item For a system $S = (X,C,succ, U, I)$, a policy $p$ is a function $p: X\rightarrow C$ and the function $succ_p: X\rightarrow 2^X$ is defined by $succ_p(x) = succ(x,p(x))$.
			\item A policy $p$ is piecewise continuous (piecewise Lipschitz continuous) DT if there exists a decision tree (DT) $\mathcal{T}$ that defines $p$, i.e., each leaf $i$ of $\mathcal{T}$ is assigned a continuous (Lipschitz continuous) function $f_i(x)$ defined on $X$, and $p$ is defined by
			\[
			p(x) = f_i(x) \quad \text{if } \mathcal{T}(x)=i.
			\]
		\end{itemize}
	\end{definition}


Now we define the reachable set $R$ and $R_\delta$.

\begin{definition}
	For a system $S=(X,C,succ, U, I)$ with policy $p$, the reachable set $R(I)$ is the union of all sequence $\{x_n:n\in\mathbb{N}\}$ that is a run by $S$ and $p$, i.e., $\{x_n:n\in\mathbb{N}\}$ satisfies \begin{align*}
		x_0&\in I\\
		x_{n+1} &\in succ_p(x_n), n\in\mathbb{N}
	\end{align*}
	
	And the reachable set under perturbation of at most $\delta$ is the union of all sequence $\{x_n:n\in\mathbb{N}\}$ that is a run by $S_\delta$ and $p$, i.e., $\{x_n:n\in\mathbb{N}\}$ satisfies \begin{align*}
		x_0&\in I\\
		x_{n+1} &\in B_\delta(succ_p(x_n)), n\in\mathbb{N}
	\end{align*}
\end{definition}


Next we define safety.


\begin{definition}
	For a system $S=(X,C,succ, U, I)$, a policy $p$ is safe if $R(I)$ is disjoint from $U$. A policy $p$ is robustly safe if for some $\delta$, $R_\delta(I)$ is disjoint from $U$.
\end{definition}


Next we define certificate.

\begin{definition}
	For a system $S$ and policy $p$, a certificate is a function $f : A \rightarrow \{0,1\}$ such that:
	\begin{itemize}
		\item $f^{-1}(1)\supseteq I$, $f^{-1}(1)\cap U = \emptyset$.
		
		\item For any $x$ and $x'$ such that $f(x)=1$, we have $x' \in succ_p(x)$ $f(x')=1$.
	\end{itemize}
	
A certificate $f$ is DT if there exists a decision tree (DT) $\mathcal{T}$ that defines $f$, i.e., each leaf $i$ of $\mathcal{T}$ is assigned a number $a_i \in \{0,1\}$, and $f$ is defined by
\[
f(x) = a_i \quad \text{if } \mathcal{T}(x)=i.
\]

\end{definition}



\begin{theorem}
	If a piecewise Lipschitz continuous DT policy $p$ is robustly safe and $succ$ of the system is Lipschitz continuous, then
	there exists a DT certificate for safety. 
	
	Here, a set value function $f$ is Lipschitz if there exists a constant $C$ such that for any $x,y$ with $d(x,y)=d$, $f(y)\subseteq B_{Cd}(f(x))$.
\end{theorem}

From now, we fix a system $S$ and assume $p$ is a piecewise Lipschitz continuous DT policy which is robustly safe for $\delta>0$. 

 Without loss of generality, we assume the Lipschitz constant for $succ$ is 1.

The proof is in several steps: \begin{itemize}
	\item First, we will show, $R_{\delta/2}(I)$ has an open neighborhood contained in $R_\delta(I)$ and hence disjoint with $U$.
	
	\item Second, we will show $R_{\delta/2}(I)$ is attractive in a neighborhood. 
\end{itemize}


\begin{lemma}
	$B_{\delta/2}(R_{\delta/2}(I))\subseteq R_\delta(I)$.
\end{lemma}

\begin{proof}
  This is by definition.
\end{proof}


\begin{lemma}
	There exists a finite polytopes decompose of $X=\bigcup_{i\in I,\ I\text{ is finite}} Q_i$ (they may include or exclude their boundaries) and Lipschitz continuous functions $f_i$ assigned for each polytopes that can define $p$:\begin{align*}
		p(x) = f_i(x)\quad \text{if }x\in Q_i
	\end{align*}
\end{lemma}

\begin{proof}
	This is because $p$ is a piecewise  Lipschitz continuous DT policy.
\end{proof}


\begin{lemma}
	There exists $\varepsilon >0$ such that $R_{\delta/2}(I)$ is attractive in the region $\bigcup_{i\in I}B_\varepsilon(R_{\delta/2}(I)\cap Q_i)\cap Q_i$.
\end{lemma}

\begin{proof}
	We use $R_i$ to denote each $R_{\delta/2}(I)\cap Q_i$.	By assumption, each $Q_i$ is assigned with a Lipschitz continuous function $f_i$ with Lipschitz constant $K_i$. Let $K = \max_{i\in I} K_i$ and $\varepsilon = \frac{\delta}{2K+2} < \delta/2$ (for $L_\infty$ distance; for other distance as $L_2$, $L_1$, $\varepsilon$ should be smaller). 
	
	For a point $y$ in $B_\varepsilon(R_i)\cap Q_i$, there exists $x\in R_i\subseteq Q_i$ such that $d(x,y)<\varepsilon$. We consider $succ_p(y)$ and $succ_p(x)$. 
	
	Since $x,y\in Q_i$, $d(p(x),p(y))\leq K_i d(x,y) < K \varepsilon<\delta/2$.  By assumptions, we have $succ_p(y)\subseteq B_{\delta/2}(succ_p(x))$. Because $x\in R_i\subseteq R_{\delta/2}(I)$, by definition, we have that $succ_p(y)\subseteq R_{\delta/2}(I)$.
	
	This is what we want to show.
\end{proof}


\begin{lemma}
	There is a region $A$ satisfying the following:
	\begin{itemize}
			\item A is a union of polytopes.
		\item $A\supseteq R_{\delta/2}(I)$
		\item $A\subseteq\bigcup_{i\in I}B_\varepsilon(R_{\delta/2}(I)\cap Q_i)\cap Q_i$
	\end{itemize}
	Hence, from $A$ we can construct a DT certificate.
\end{lemma}


%\begin{lemma}
%	Let $S$ be a system with policy $p$ and initial region $I$. For any $x\in R_r(I)$ and any $\varepsilon$, we have
%	\[
%	B_\varepsilon(x) \subseteq R_{r+\varepsilon }(I).
%	\]
%	Hence $B_\varepsilon(R_r(I)) \subseteq R_{r+\varepsilon}(I)$.
%\end{lemma}
%
%\begin{proof}
%	To prove this lemma, fix $x \in R_r(I)$.
%	
%	We assume $x = x_{n+1}$ and $x_{n+1} = p(x_n)+d_n$ with $|d_n|<r$ and $x_n\in R_r(I)\cup I\subseteq R_{r+\varepsilon}(I)\cup I$. Then, we have \[
%	B_{\varepsilon}(x) \subseteq B_{|d_n| + \varepsilon}(p(x_n)) \subseteq B_{r + \varepsilon}(p(x_n)).
%	\]
%	
%	By definition, since $x_n\in R_{r+\varepsilon}(I)\cup I$, $B_{r + \varepsilon}(p(x_n))\subseteq R_{r+\varepsilon}(I)$.
%	So,
%	\[
%B_{\varepsilon}(x) \subseteq
%	B_{r + \varepsilon}(p(x_n))\subseteq R_{r+\varepsilon}(I).
%	\]
%	This completes the proof.
%\end{proof}
%
%\vspace*{2ex}
%
%Notice that if $B_\varepsilon(R_r(I)) \subseteq R_{r+\varepsilon}(I)$, then $\operatorname{cl}(R_r(I)) \subseteq \operatorname{int}(R_{r+\varepsilon}(I))$. 
%
%\begin{proof}[of Theorem 1]
%	
%	We assume that $S$ is $\delta$-robustly safe.
%	
%	From Lemma 1, we can see that \begin{itemize}
%		\item $\operatorname{cl}(R(I)) \subseteq \operatorname{int}(R_\delta(I))$.
%		
%		\item Every point in $R(I)$ is at least a distance $\delta$ away from the boundary of $R_\delta(I)$.
%	\end{itemize}   
%	
%	
%	Therefore, we can partition the space into hypercubes of side length $\frac{\delta}{2\sqrt{n}}$ where $n$ is the dimension, and cover $R(I)$ with these hypercubes. Those hypercubes are in $R_\delta(I)$, and even $R_{\delta/2}(I)$.
%	
%	Put them together, we can get the following conclusion:
%	\begin{itemize}
%		\item There is a union of polytopes $\bigcup_{P \in \mathcal{P}} P$ contained in   $R_{\delta}(I)$ that covers $R(I)$.
%
%		\item Any point start from $\bigcup_{P \in \mathcal{P}} P$ following $S$ without perturbation will never enter $U$, and has a minimal distance from $U$.
%	\end{itemize}
%\end{proof}





\end{document}
